%%%%%%%%%%%%%%%%%%%%%%%%%%%%%%%%%%%%%%%%%%%%%%%%%%%%%%%%%%%%%%%%%%%%%%%%%%%%%
%%%                              Data Variable                            %%%
%%%%%%%%%%%%%%%%%%%%%%%%%%%%%%%%%%%%%%%%%%%%%%%%%%%%%%%%%%%%%%%%%%%%%%%%%%%%%
\subsubsection{Data Variables}
\label{sec:uivariables}
Most of the GUI-objects defined in the \UIMANAGER{} are directly linked
with data items of the datapool.
These objects are used to enter or display values. \\

\input{diagrams/ui_field_data_reference}
\input{diagrams/ui_xfer}
\index{data item!ui\_manager}
\index{data item!data variable}
\index{data variable}

\index{REAL@\REAL!ui\_manager data item functions}
\index{IMAG@\IMAG!ui\_manager data item functions}
\index{ABS@\ABS!ui\_manager data item functions}
\index{ARG@\ARG!ui\_manager data item functions}
\index{COMPLEX@\COMPLEX!ui\_manager data item functions}
\index{FILENAME@\FILENAME!ui\_manager data item functions}
\index{  @Signs / Characters!. (dot)!structure item separator}
\index{STRUCT@\STRUCT!structure item separator}
\begin{tabularx}{\textwidth}{l|X}
function         & description  \\
\hline
\REAL            & Real value of \COMPLEX{} data item \\
\IMAG            & Imaginary quantity of \COMPLEX{} data item \\
\ABS             & Absolute value \\
\ARG             & Angle of \COMPLEX{} data item \\
\FILENAME        & Name of the file saved or opened through a filestream. \\
{\bfseries ID\_DATAVARIABLE} & references a data item declared in datapool (section
                   \nameref{sec:dpitem} page \pageref{sec:dpitem}). \\
{\bfseries ID\_DATASET} & references a data set declared in datapool (section 
                   \nameref{sec:dpset} page \pageref{sec:dpset}). \\
.                & structure item separator \\
\end{tabularx}

\input{diagrams/ui_field_indizes}
\input{diagrams/ui_field_index}
\label{uifieldindizes}
\vspace{0.5cm}

By using indexes you can refer to each value of a data item. \\

\index{  @Signs / Characters!\texttt{"+} (plus)!index offset}
\index{  @Signs / Characters!: (colon)!index range}
\begin{tabularx}{\textwidth}{l|X}
index                   & description \\ 
\hline
\verb+ID_INDEX+         & refers a index object defined in UI\_MANAGER (section \nameref{sec:uiindex} page \pageref{sec:uiindex}) \\
+                       & add constant offset to an array-index \\
wildcard                & see section \nameref{sec:wildcards} page \pageref{sec:wildcards} \\
:                       & specifies a range (eg. first 5 [0:4]) \\
\end{tabularx}
\vspace{0.5cm}

Examples: \\

\begin{boxedminipage}[t]{\linewidth}
\begin{alltt}
data_item_id      // a scalar item with index 0,0 (or matrix item)
data_item_id[ ]   // an array item
data_item_id[n]   // a scalar item with index 0,n
data_item_id[*]   // an array item with running index on row
data_item_id[m,*] // an array item with running index on column
data_item_id[m,n] // a scalar item with index m,n

data_item_id[index-object_identifier]
                       // it is displaied an index-object,
                       // which enables you to scroll through
                       // the data item using the maouse pointer. \\

data_item_id[ 5 + index-object_identifier]
                       // adds an offset of 5 to the index-object. \\
\end{alltt}
\end{boxedminipage}

\vspace{0.5cm}

\index{LABEL@\LABEL!index identifier}
\index{UNIT@\UNIT!index identifier}
For options like \LABEL{} and \UNIT{} use data items without indexes.
\vspace{0.5cm}
