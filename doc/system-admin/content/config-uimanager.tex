\newpage
\subsection{UI\_MANAGER}
%           -----------
\label{sec:uimanager}
%%%%%%%%%%%%%%%%%%%%%%%%%%%%%%%%%%%%%%%%%%%%%%%%%%%%%%%%%%%%%%%%%%%%%%%%%%%%%
%%%                             Description                               %%%
%%%%%%%%%%%%%%%%%%%%%%%%%%%%%%%%%%%%%%%%%%%%%%%%%%%%%%%%%%%%%%%%%%%%%%%%%%%%%
\subsubsection{Description}
\label{sec:uidescription}
\index{UI\_MANAGER@\UIMANAGER}
The \UIMANAGER{} block defines the appearance of the user interface
without having the system administrator
to place the interface objects geometrically. \INTENS{} places
the text fields, plots, tables, buttons etc.
automatically after calculating their sizes.

WARNING:  Due to this feature, the layout of the windows may look
slightly different after changing the language.

\input{diagrams/ui_manager_description}

An \INTENS{}-application usually has a main window with a menubar and
several user defined forms.
Each form may contain any number of horizontally
and vertically arranged fieldgroups, text
and plot windows and folder groups. Its layout is defined within a form declaration.
