\subsubsection{Wildcards}
%              ---------
\label{sec:wildcards}
%%%%%%%%%%%%%%%%%%%%%%%%%%%%%%%%%%%%%%%%%%%%%%%%%%%%%%%%%%%%%%%%%%%%%%%%%%%%%
%%%                            Wildcards                                  %%%
%%%%%%%%%%%%%%%%%%%%%%%%%%%%%%%%%%%%%%%%%%%%%%%%%%%%%%%%%%%%%%%%%%%%%%%%%%%%%
\index{wildcards}

Wildcards are used as a place holder and are represented by the following two characters. \\

\index{  @Signs / Characters!\# (hash)!wildcard}
\index{  @Signs / Characters!* (asterisk)!wildcard}
\input{diagrams/wildcard}
Both of them have the same meaning. There is no difference between the interpretation of them.
However, it is suggested to use only the asterisk.
\vspace{1cm}

Example:


\begin{boxedminipage}[t]{\linewidth}
\begin{verbatim}
Data_item_identifier[*]
Data_item_identifier[#]

Index_identifier[*]
Index_identifier[#]

\end{verbatim}
\end{boxedminipage}
