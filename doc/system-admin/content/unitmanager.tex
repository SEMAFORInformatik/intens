\section{Unit Manager} \label{sec:unitmanager}

The Unit Manager is an INTENS component that handles the
scaling of values in the user interface automatically.

We recommend having values in their base units in the datapool, and then
scale them for the user interface. This way, the values passed
to calculation programs, stored in the database or in files are
in base units. They are only scaled for the user interface.

\subsection{Example}
Throughout this section, let's use an example variable motor.Lls
that was defined this way:

\begin{intens}
DATAPOOL
  STRUCT Motor: Component {
    ...
    REAL {EDITABLE, SCALAR}
      Lls {
        LABEL=_("Stator Leakage"),
        HELPTEXT=_("Stator leakage inductance"),
        UNIT="mH"
      },
    ...
  };
  Motor
    motor {
      LABEL=_("Motor"),
      PERSISTENT, SCALAR
    }
  ;
\end{intens}

It has the value 0.001 H, shown as 1 mH in the user interface.

\begin{figure}[h]
  \begin{center}
    \includegraphics[width=0.35\linewidth]{images/unitmanager-test-application-default}
  \end{center}
  \caption{Test application using Unit Manager}\label{fig:unitmanager-test-application-default}
\end{figure}

\subsection{Without Unit Manager}
Without the Unit Manager, the variable motor.Lls needs a scale
factor *1e3 when shown in the user interface:

\begin{intens}[escapeinside={(*}{*)}]
FIELDGROUP motor_properties_fg(
  ...
  LABEL(motor.Lls)      motor.Lls(*\textbf{*1e3}*)   UNIT(motor.Lls),
  ...
);
\end{intens}

INTENS reads the value motor.Lls (0.001), multiplies it by 1000 (1.0)
and shows this in the user interface. When the user enters a value,
INTENS takes that value (1.0), divides it by 1000 (0.001) and stores it.
See figure \ref{fig:unitmanager-scaling} on page \pageref{fig:unitmanager-scaling}.

\begin{figure}[h]
  \begin{center}
    \includegraphics[width=0.7\linewidth]{xfig/unitmanager-scaling}
  \end{center}
  \caption{Unit Scaling}\label{fig:unitmanager-scaling}
\end{figure}

That brings the risk of forgetting to provide the (correct)
scale factors in the UI\_MANAGER, while the UNIT is defined in the
DATAPOOL. This is where the Unit Manager comes in.

\subsection{With Unit Manager}
You activate the Unit Manager using the INTENS
command line option \texttt{--}unitManager.

As mentioned, the Unit Manager handles the scaling of values
in the user interface automatically.

You no longer provide the scale factor:
\begin{intens}
FIELDGROUP motor_properties_fg(
  ...
  LABEL(motor.Lls)      motor.Lls   UNIT(motor.Lls),
  ...
);
\end{intens}

In fact, you could even keep the scale factors, as they are all ignored in Unit Manager mode.

As you defined the UNIT ``mH'' in the variable definition, INTENS
knows that you want to see motor.Lls in ``mH''.

\subsection{Configuring the Units}
INTENS knows a set of standard units. For example
\begin{lstlisting}[language=json, escapeinside={(*}{*)}]
[...,
 {"name": "H", "factor": 1.0,
 "derived": [
     {"name": "mH", "factor": 1e3},
     {"name": "(*$\mu$*)H", "factor": 1e6}
 ]},
 ...
]
\end{lstlisting}

``H'' is the base unit. It has the scale factor 1.0 - as most
base units do.
And there are two derived units, ``mH'' and ``$\mu$H'', with their
scaling factors.
That information is enough for INTENS to automatically show
motor.Lls (0.001 H) as 1 mH.

Now, your application likely needs more units than the small set of
units known by INTENS.
You can extend or change the units for the application using the built in
Unit Manager form (menu Options > Unit Manager, see figure
\ref{fig:unitmanager-editor} on page \pageref{fig:unitmanager-editor}.

\begin{figure}[h]
  \begin{center}
    \includegraphics[width=\linewidth]{images/unitmanager-editor}
  \end{center}
  \caption{Unit Manager: Editor}\label{fig:unitmanager-editor}
\end{figure}

You can add new base units and / or derived units as needed.
When you close the form, the information is stored in the file
<APPHOME>/etc/application\_unit.json.
You can also extend or change the units for the application by modifying
this file directly. Just keep in mind that the file is read when the application
starts, so you need to restart the application to see the effect of your
modifications.
All of this is normally done during the application development.
Remember to distribute the file etc/application\_unit.json with your application.

If a single value ``mH'' is preferred (i.e. no combobox for inductance units), the following configuration
can be written to the file

\verb+etc/application_unit.json+:
\begin{lstlisting}[language=json]
{"name": "H", "factor": 1.0, "use_set": 0,
"derived": [
    {"name": "mH", "factor": 1e3}
]}
\end{lstlisting}
The value ``\verb+use_set+'' controls the content of the set.
Only units with ``\verb+use_set+'' value 1 (which is default)
are included. use\_set is labelled ``Use'' in the Unit Manager form.

\subsection{Dynamically changing a Unit}
Maybe a user wants to see different units in the user interface?
%
You can tell INTENS to show the units as comboboxes instead of
as labels.
%
To do so, provide one of two values to the command line option unitManager:
\begin{itemize}
\item \texttt{--}unitManager=comboBox\_hide\_single
\item \texttt{--}unitManager=comboBox\_always
\end{itemize}

As mentioned above, there are two derivates, ``mH'' and ``$\mu$H'',
for the base unit ``H''. So, INTENS will show a combobox with the
three entries: H, mh, $\mu$H,
and H as the base unit (ie. the unit of the values in the DATAPOOL).

``mH'' will be selected by default, as that is what you defined in the
variable definition.

\begin{figure}[h]
  \begin{center}
    \includegraphics[width=0.75\linewidth]{images/unitmanager-test-application-hide-single}
  \end{center}
  \caption{--unitManager=comboBox\_hide\_single}\label{fig:unitmanager-test-application-hide-single}
\end{figure}

When you select a different unit, INTENS knows the new unit name and
its scale factor and now shows motor.Lls as 0.001 H or 1000 $\mu$H.

This choice is remembered for you and when you start the INTENS
application the next time, motor.Lls will be shown in the chosen unit.
%
(INTENS stores these choices in the user resource file.)

If you prefer to see a combobox for all (known) units, use the command line
option \newline
\texttt{--}unitManager=comboBox\_always (see figure
\ref{fig:unitmanager-test-application-always} on page \pageref{fig:unitmanager-test-application-always}.

\begin{figure}[h]
  \begin{center}
    \includegraphics[width=0.35\linewidth]{images/unitmanager-test-application-always}
  \end{center}
  \caption{--unitManager=comboBox\_always}\label{fig:unitmanager-test-application-always}
\end{figure}

Now, the unit ``Vs'' is also shown as a combobox, even though there is no other choice.
The unit ``Ω'' is still shown as a label, because this unit is not configured yet.

You can use this mode if you prefer the look or to see what units you have not yet defined.
