\section{Release Notes 3.1}
%        =================
\label{sec:releasenotes-3-1}
%%%%%%%%%%%%%%%%%%%%%%%%%%%%%%%%%%%%%%%%%%%%%%%%%%%%%%%%%%%%%%%%%%%%%%%%%%%%%
%%%                        Release Notes 3.1                              %%%
%%%%%%%%%%%%%%%%%%%%%%%%%%%%%%%%%%%%%%%%%%%%%%%%%%%%%%%%%%%%%%%%%%%%%%%%%%%%%
\index{release notes}
The following features have been changed or added since Version 2.1:\\

%%---------------------------------------------------------------
\subsection{May 2003}
\subsubsection{Functions}
\begin{enumerate}
\item The INIT-function is executed at startup.
\label{releasenotesfuncinit}
\index{FUNC@\FUNC!release notes}
\index{INIT@\INIT!function identifier!release note}
\index{FUNCTIONS@\FUNCTIONS!\INIT!release note}

\begin{boxedminipage}[t]{\linewidth}
\begin{alltt}
  FUNC \INIT \{ default_value = 19; \};
\end{alltt}
\end{boxedminipage}

See section \nameref{sec:funcidentifiers} page \pageref{sec:funcidentifiers}. \\

\item The AFTER\_DB\_LOGON-function is executed when logging in to the Database.
\index{AFTER\_DB\_LOGON@\AFTERDBLOGON!release note}
\index{FUNCTIONS@\FUNCTIONS!\AFTERDBLOGON!release note}

\begin{boxedminipage}[t]{\linewidth}
\begin{alltt}
  FUNC \AFTERDBLOGON \{ RUN filter_identifier; \};
\end{alltt}
\end{boxedminipage}

See section \nameref{sec:funcidentifiers} page \pageref{sec:funcidentifiers}. \\

\index{ON\_CYCLE\_SWITCH@\ONCYCLESWITCH!release note}
\item The ON\_CYCLE\_SWITCH-function is executed when switching between datapool cycles.
\index{FUNCTIONS@\FUNCTIONS!\ONCYCLESWITCH!release note}

\begin{boxedminipage}[t]{\linewidth}
\begin{alltt}
  FUNC \ONCYCLESWITCH \{ RUN func_init_dataitems; \};
\end{alltt}
\end{boxedminipage}

See section \nameref{sec:funcidentifiers} page \pageref{sec:funcidentifiers}. \\

\item The QUIT-function is executed at INTENS exit.
\index{QUIT@\QUIT!release note}
\index{FUNCTIONS@\FUNCTIONS!\QUIT!release note}

\begin{boxedminipage}[t]{\linewidth}
\begin{alltt}
  FUNC \QUIT \{ RUN func_save_all; \};
\end{alltt}
\end{boxedminipage}

See section \nameref{sec:funcidentifiers} page \pageref{sec:funcidentifiers}. \\

\item CYCLENAME can be used to set or get the name of a cycle. \\
\index{CYCLENAME@\CYCLENAME!release note}
Statement:

\begin{boxedminipage}[t]{\linewidth}
\begin{alltt}
  \CYCLENAME = "current cycle";
  \CYCLENAME[2] = "this is cycle 2";
\end{alltt}
\end{boxedminipage}

See section \nameref{fucyclestatements} page \pageref{fucyclestatements}. \\
Expression:

\begin{boxedminipage}[t]{\linewidth}
\begin{alltt}
  PRINT("This is cycle:",\CYCLENAME);
  PRINT("cycle 2 is:",\CYCLENAME[2]);
\end{alltt}
\end{boxedminipage}

See section \nameref{fuexpressionscycle} page \pageref{fuexpressionscycle}. \\

\label{relnotescolorbit}
\item By unseting a higherlevel COLORBIT, the dataitem is displayed with previously set color of lower level.
\index{SET@\SET!release note}
\index{COLORBIT@\COLORBIT!release note}

\begin{boxedminipage}[t]{\linewidth}
\begin{alltt}
  SET( \COLORBIT, data_item, 4 );
  ...
  SET( \COLORBIT, data_item, 2 );
  ...
  UNSET( \COLORBIT, data_item, 2 );  // color is reset to 4
\end{alltt}
\end{boxedminipage}

See section \nameref{fudatastatementset} page \pageref{fudatastatementset}. \\

\index{TIMESTAMP@\TIMESTAMP!release note}
\item Set/unset TIMESTAMP is used to to change the modify status of a data item (in conjunction whith DB\_MANAGER modify).

\begin{boxedminipage}[t]{\linewidth}
\begin{alltt}
  SET( \TIMESTAMP, data_item );
  ...
  UNSET( \TIMESTAMP, data_item );
  ...
\end{alltt}
\end{boxedminipage}

See section \nameref{fudatastatementset} page \pageref{fudatastatementset}. \\

\index{EXIT@\EXIT!release note}
\item The EXIT-statement is used to quit INTENS. It will not prompt for confirmation and quit immediatly. \\

\begin{boxedminipage}[t]{\linewidth}
\begin{alltt}
  \EXIT;
\end{alltt}
\end{boxedminipage}

See section \nameref{fuprintstatements} page \pageref{fuprintstatements}. \\

\item File access expressions return file and directory names:
\index{FILENAME@\FILENAME!function expression!release note}
\index{BASENAME@\BASENAME!release note}
\index{DIRNAME@\DIRNAME!release note}
\index{OPEN@\OPEN!function expression!release note}

\begin{boxedminipage}[t]{\linewidth}
\begin{alltt}
  FUNC get_data {
    STRING filename, dirname, basename;
    OPEN(filestream_id);
    filename = \FILENAME(filestream_id);
    dirname = \DIRNAME(filestream_id);
    basename = \BASENAME(filestream_id);
  };
\end{alltt}
\end{boxedminipage}

See section \nameref{fuexpressionsfunctions} page \pageref{fuexpressionsfunctions}. \\

\item Processgroup output-stream option FORMAT is not supported anymore:
\index{ASCII}
\index{BINARY}
The output format is set to ASCII by default. It is not possible to change it to BINARY by the format-option.
See section \nameref{opprocessgroupoutputstreamoption} page \pageref{opprocessgroupoutputstreamoption}. \\


\end{enumerate}


%%---------------------------------------------------------------
\subsection{July 2002}
\subsubsection{General}
\begin{enumerate}
\item 
Datapool variables may be used as scale factors. \\
See section \nameref{sec:scale} page \pageref{sec:scale}. \\
\index{LABEL@\LABEL!function expression!release note}
\index{UNIT@\UNIT!function expression!release note}
\item The functions LABEL() and UNIT() can be used where constant string expressions
   are expected.
The syntax has changed also:
\begin{itemize}
\item Version 2 \\
\LABEL.data\_item \\
\UNIT.data\_item \\
\item Version 3 \\
\LABEL (data\_item) \\
\UNIT (data\_item) \\
\end{itemize}
See section \nameref{sec:string} page \pageref{sec:string}. \\
\item 
String concatenation by the '\&'-character: \\
The syntax has changed:
\index{PROCESS@\PROCESS!release note!string concatenation}
\begin{itemize}
\item Version 2 \\
PROCESS plot\_ct\_s1\_proc : BATCH \\
\{ "/bin/cat"   " /projects/adtranz/tracpack/tracomo/1.7.0/tcmo/.tcmo1.plt"\}; \\
\item Version 3 \\
PROCESS plot\_ct\_s1\_proc : BATCH \\
\{ "/bin/cat" {\bfseries \&} " /projects/adtranz/tracpack/tracomo/1.7.0/tcmo/.tcmo1.plt"\};
\end{itemize}
See section \nameref{sec:string} page \pageref{sec:string}. \\
\item commandlineoptions
\begin{itemize}
\item -leftIcon
\item -rightIcon
\item -startupImage
\item -includePath
\end{itemize}
See section \nameref{sec:cl-options} page \pageref{sec:cl-options}. \\
\end{enumerate}
%%
%%---------------------------------------------------------------
\subsubsection{Datapool}
\begin{enumerate}
\item A data variable can be composed by a fixed number of
  fields described by a structured data type:
\index{DATAPOOL@\DATAPOOL!data structure!release note}
\index{DATAPOOL@\DATAPOOL!struct!release note}
\index{STRUCT@\STRUCT!release note}

\begin{boxedminipage}[t]{\linewidth}
\begin{alltt}
  \STRUCT Motor \{
    \STRING    name;
    \INTEGER   poles;
    \REAL      nom_power;
  \};
  \STRUCT DriveSystem \{
    \STRING  id;
    \INTEGER  num_motors;
    Motor motor;
  \};
       
  DriveSystem dsys;
\end{alltt}
\end{boxedminipage}

See section \nameref{sec:dpstruct} page \pageref{sec:dpstruct}. \\
\item New data item option TAG is used by navigator to reference datapool fields:
\index{TAG@\TAG!release note}

\begin{boxedminipage}[t]{\linewidth}
\begin{alltt}
  STRUCT Motor \{
    INTEGER   id;
    STRING    name         \{TAG={\bfseries Tname}\};
    INTEGER   poles        \{TAG={\bfseries Tpoles}\};
    REAL      nom_power;
  \};

  ....

  NAVIGATOR
    motornav \{ COL( "Name" {\bfseries Tname}, "poles" {\bfseries Tpoles} ) \}

  ....

\end{alltt}
\end{boxedminipage}

See DATAPOOL section \nameref{dia:dataitemoptions} page \pageref{dia:dataitemoptions} and 
NAVIGATOR section \nameref{sec:uinavigator} page \pageref{sec:uinavigator}. \\
\item Predefined data item HOST. \\
See section \nameref{dataitempredefined} page \pageref{dataitempredefined}. \\
\index{COMPLEX@\COMPLEX!release note}
\item new data type COMPLEX. \\
See section \nameref{sec:dpitem} page \pageref{sec:dpitem}. \\
\item Data item attributes CLASSNAME, SCALAR, CELL and GLOBAL. \\
See section \nameref{dataitemattributes} page \pageref{dataitemattributes}. \\
\item new data item options COMBOBOX, CLASSNAME, SCALAR, CELL, HIDDEN and GLOBAL. \\
See section \nameref{dia:dataitemoptions} page \pageref{dia:dataitemoptions}. \\
\end{enumerate}
%%
%%---------------------------------------------------------------
\subsubsection{User Interface}
\begin{enumerate}
\item The fieldgroup option \MENU=\HIDDEN{} suppresses the index-menu of a table 
(when pressing the right mouse button over column index). \\
See section \nameref{sec:uifieldgroupotions} page \pageref{sec:uifieldgroupotions}. \\
\item Usergroups may be described and referred by usergroup identifier. \\
\index{USERGROUPS@\USERGROUPS!release note}

\begin{boxedminipage}[t]{\linewidth}
\begin{alltt}
\USERGROUPS group_id ( "staff","others" );
\end{alltt}
\end{boxedminipage}

See section \nameref{sec:usergroups} page \pageref{sec:usergroups}. \\
\index{LIST@\LIST!release note}
\item LIST is a new GUI element for displaying values in
a scrollable list.

\begin{boxedminipage}[t]{\linewidth}
\begin{alltt}
  \LIST list_id ( "Column label" data_item[*] );
\end{alltt}
\end{boxedminipage}

See section \nameref{sec:uilist} page \pageref{sec:uilist}. \\
\item Structured values can be displayed in a navigator:
\index{NAVIGATOR@\NAVIGATOR!release note}

\begin{boxedminipage}[t]{\linewidth}
\begin{alltt}
  \NAVIGATOR navigator\_id \{\COL (data-item\_TAG)\} (
                         structure\_id[*] 
  );
\end{alltt}
\end{boxedminipage}

See section \nameref{sec:uinavigator} page \pageref{sec:uinavigator}. \\
\item A pop up \MENU{} is assigned to Navigator by naming the menu identifier 
the same as the datapool structure identifier. \\
See example on page \pageref{navigatordragndrop}. \\
\item INDEX is a GUI-Object which represents the index of data items. \\

\begin{boxedminipage}[t]{\linewidth}
\begin{alltt}
  \INDEX inx_id;
  FIELDGROUP fg_id ( inx_id, data_item[inx_id] );
\end{alltt}
\end{boxedminipage}

See section \nameref{sec:uiindex} page \pageref{sec:uiindex}. \\
\item To display bitmap graphics use the PIXMAP fieldgroup command. \\
See section \nameref{sec:uifieldgroup} page \pageref{sec:uifieldgroup}. \\
\item additional attributes to display pictures on buttons or using scrollbars for string items. \\
See section \nameref{sec:uifieldgroup} page \pageref{sec:uifieldgroup}. \\
\item Predefined variable holding the current opened filename: \\
The syntax has changed:
\index{FILENAME@\FILENAME!ui\_manager data item functions!relese note}
\begin{itemize}
\item Version 2 \\
FIELDGROUP inp\_control\_fg (VOID "Input Files", \\
"Mech Tractive Unit" {\bfseries p\_unt\_stream}:-80 ); \\
\item Version 3 \\
FIELDGROUP inp\_control\_fg (VOID "Input Files", \\
"Mech Tractive Unit" {\bfseries \FILENAME (fs\_p\_unt\_stream)}:-80 ); \\
\end{itemize}
See section \nameref{dia:uifielddatareference} page \pageref{dia:uifielddatareference}. \\
\item Void is now sizable. \\
See section \nameref{sec:uifieldgroup} page \pageref{sec:uifieldgroup}. \\
\index{  @Signs / Characters!- (hyphen)!left alignment release note}
\item Negative width ('-') alignes the data item to the left. \\
See section \nameref{dia:uifieldattributes} page \pageref{dia:uifieldattributes}. \\
\item TSEP displays numeric values with thousand separators. \\
See section \nameref{dia:uifieldattributes} page \pageref{dia:uifieldattributes}. \\
\item Title string in TABLE may be justified. \\
See section \nameref{sec:uitable} page \pageref{sec:uitable}. \\
\item form and container options:
\begin{itemize}
\item JUSTIFY \\
    Alignment of fields or fieldgroups.
\item PW / NP (paned window, no paned window) and SB, NS (scrollbar, no scrollbar) \\
   Scrollbars within and separators between fieldgroups.
\index{APP\_MODAL@\APPMODAL!release note}
\item APP\_MODAL \\
   Opens the form dialog modal.
\item USERGROUPS \\
  This form is available only to previously defined usergroups.
\end{itemize}
  See section \nameref{sec:uiform} page \pageref{sec:uiform}. \\
\item menu options:
\index{FORM@\FORM!ui\_manager menu option!release note}
\index{TOGGLE@\TOGGLE!ui\_manager forms menu!release note}
 \begin{itemize}
  \item form menu \\
    \begin{itemize}
    \item HELPTEXT \\
    Displays text when crossing the menu label with mouse pointer.
    \item FORM \\
    Places menu into specified form.\\
    \item FORM option TOGGLE \\
    The menu entry toggles specified form on or off.\\
    \end{itemize}
  See section \nameref{sec:uimenuid} page \pageref{sec:uimenuid}. \\
  \end{itemize}
 \begin{itemize}
  \item file menu \\
    \begin{itemize}
    \item STD\_WINDOW, LOG\_WINDOW \\
    Text window forms. \\
    \item FUNC \\
    Calls a predefined function. \\
  \end{itemize}
  See section \nameref{uifilemenu} page \pageref{uifilemenu}. \\
 \end{itemize}
\end{enumerate}
%%
%%---------------------------------------------------------------
\subsubsection{Plots}
\begin{enumerate}
\item New and extremely flexible plot for 2 dimensional curves:
\index{LABEL@\LABEL!plot2d option!release note}

\begin{boxedminipage}[t]{\linewidth}
\begin{alltt}
  \PLOTTWOD  curve_plot (
        plot\_id (
           xValues \{LABEL="xaxis"\} (
              y1Values \{YAXIS1, LABEL="y1-axis"\} 
              y2Values \{YAXIS2, LABEL="y2-axis"\} 
           )
       );
\end{alltt}
\end{boxedminipage}

See section \nameref{sec:uiplot2d} page \pageref{sec:uiplot2d}. \\
\index{XRTGRAPH!release note}
\item It is suggested to use PLOT2D instead of XRTGRAPH. \\
\index{SCROLLPLOT (no more supported)!release note}
\item 
  SCROLLPLOT will not be supported any longer. \\
  It is suggested to use PLOT2D instead of SCROLLPLOT. \\
\item 
PLOTGROUP must be replaced by LISTPLOT. \\
  Exept XRT3DPLOT, UNIPLOT and PLOT2D, which are explained below.\\
\item xrt3Dplot \\
The syntax has changed:
\index{PLOTGROUP (old plot syntax)!release note}
\index{TYPE (old PLOTGROUP syntax)!release note}
\index{XRT3DPLOT@\XRTTHREEDPLOT!release note (old PLOTGROUP syntax)}
\begin{itemize}
\item Version 2 \\
  \PLOTGROUP{} conv\_losses\_3d\_pg \{ "Converter Losses 3D" \\
  , \TYPE=\XRTTHREEDPLOT, CAPTION = losses\_3d\_capt\_stream \} \\
  {\bfseries ( (} \\
  plo\_3d \{ XAXIS = ZK\_LosSR \{ LABEL = "Effort [\%]" \} ...... \\
  {\bfseries ) );} \\
\item Version 3 \\
  \XRTTHREEDPLOT conv\_losses\_3d\_pg \{ "Converter Losses 3D" \\
  ,CAPTION = losses\_3d\_capt\_stream \} \\
  {\bfseries (} \\
  plo\_3d \{ XAXIS = ZK\_LosSR \{ LABEL = "Effort [\%]" \} ...... \\
  {\bfseries );} \\
\end{itemize}
\item uniplot \\
The syntax has changed:
\index{TYPE (old UNIPLOT syntax)!release note}
\index{UNIPLOT@\UNIPLOT!release note}
\begin{itemize}
\item Version 2 \\
\PLOTGROUP{} simpel\_mech\_pg \{ "Plot Mech", \TYPE=\UNIPLOT{} \} \\
{\bfseries ()} ; \\
\item Version 3 \\
\UNIPLOT{} simpel\_mech\_pg \{ "Plot Mech" \} \\
{\bfseries ;} \\
\end{itemize}
\item Listplot option BUTTONS will not be supported any longer. 
   Use GUI Index instead (see \nameref{sec:uiindex} on page \pageref{sec:uiindex})\\
\end{enumerate}
%%
%%---------------------------------------------------------------
\subsubsection{Functions}
\begin{enumerate}

\item Read and write operations on files can be invoked
  by a function which is included in a menu
\index{FUNC@\FUNC!ui\_manager file menu!release note}
\index{OPEN@\OPEN!ui\_manager menu!release note}

\begin{boxedminipage}[t]{\linewidth}
\begin{alltt}
  \MENU OPEN ( fstream_id = "menu label",
              \FUNC{} func_id = "menu label" );
\end{alltt}
\end{boxedminipage}

See section \nameref{uifilemenu} page \pageref{uifilemenu}. \\

\item Input used as statement modifies data item calling the function

\begin{boxedminipage}[t]{\linewidth}
\begin{alltt}
  FUNC reset_data \{ \INPUT{} = default_value; \};
\end{alltt}
\end{boxedminipage}

See section \nameref{fudatastatements} page \pageref{fudatastatements}. \\
\index{THIS@\THIS!release note}
\item THIS references the structure holding the data item calling the function

\begin{boxedminipage}[t]{\linewidth}
\begin{alltt}
  FUNC reset_data \{ \THIS.field[2] = default_value; \};

  FUNC store_data \{ storage_item = \THIS.field[2]; \};

\end{alltt}
\end{boxedminipage}

See function statement on page \pageref{fudatastatements} 
and function expression on page \pageref{fuexpressionsdata} \\
\index{CONFIRM@\CONFIRM!release note}
\index{REASON_DROP@\REASONDROP!release note}
\index{SOURCE@\SOURCE!release note}
\item SOURCE references the structure holding the data item
which was dragged while {\bfseries drag'n'drop} actions on navigator trees

\begin{boxedminipage}[t]{\linewidth}
\begin{alltt}
  IF( REASON_DROP )\{
    CONFIRM( "Replace ",THIS.field," by ",{\bfseries SOURCE}.field,"?" );
  \}
\end{alltt}
\end{boxedminipage}

See section \nameref{navigatordragndrop} page \pageref{navigatordragndrop}. \\
\item \VAR{} returns the data-item-identifier out from a string data-item. 
It is used for variable data referencing. The data reference may be changed 
at runtime.
\index{VAR@\VAR!function statement!release note}

\begin{boxedminipage}[t]{\linewidth}
\begin{alltt}
  FUNC reset_data \{
    data_ref = "struct_identifier.field[2]";
    \VAR("data_ref") = default_value;
  \};
\end{alltt}
\end{boxedminipage}

See section \nameref{fudatastatements} page \pageref{fudatastatements}. \\
\index{SIZE@\SIZE!function statement!release note}
\item \SIZE{} returns the number of elements of data-items into 
another data-item

\begin{boxedminipage}[t]{\linewidth}
\begin{alltt}
  FUNC get_number_elements \{ \SIZE(output_item, test_item); \};
\end{alltt}
\end{boxedminipage}

See section \nameref{fudatastatements} page \pageref{fudatastatements}. \\
\item \ASSIGNCORR{} copies equal structure fields

\begin{boxedminipage}[t]{\linewidth}
\begin{alltt}
  FUNC copy \{ \ASSIGNCORR(struct_from, struct_out); \};
\end{alltt}
\end{boxedminipage}

See section \nameref{fudatastatements} page \pageref{fudatastatements}. \\
\item File open dialog window can be opened by using the \OPEN{} statement
\index{OPEN@\OPEN!ui\_manager menu!release note}

\begin{boxedminipage}[t]{\linewidth}
\begin{alltt}
  FUNC get_data \{ \OPEN(filestream_identifier); \};
\end{alltt}
\end{boxedminipage}

See section \nameref{fudatastatements} page \pageref{fudatastatements}. \\
\item File save dialog window can be opened by using the save statement
\index{SAVE@\SAVE!ui\_manager menu!release note}

\begin{boxedminipage}[t]{\linewidth}
\begin{alltt}
  FUNC write_file \{ \SAVE(filestream_identifier); \};
\end{alltt}
\end{boxedminipage}

See section \nameref{fudatastatements} page \pageref{fudatastatements}. \\
\item The background color of any data item can be changed dynamically
\index{COLOR@\COLOR!release note}

\begin{boxedminipage}[t]{\linewidth}
\begin{alltt}
  SET( \COLOR, data_item, 4 );
\end{alltt}
\end{boxedminipage}

See section \nameref{fudatastatements} page 
  \pageref{fudatastatements}. \\
\index{EDITABLE@\EDITABLE!set/unset!release note}
\index{FALSE@\FALSE!function statement (set/unset)!release note}
\item 
The function data statement SET( \EDITABLE, \ldots)
allows to change the editable attribute at run time:

\begin{boxedminipage}[t]{\linewidth}
\begin{alltt}
  SET( \EDITABLE, data_item, FALSE );
\end{alltt}
\end{boxedminipage}

See section \nameref{fudatastatements} page 
  \pageref{fudatastatements}.
\item 
The function data statement SET( \LOCK, \ldots)
allows to change the lock attribute at run time:

\begin{boxedminipage}[t]{\linewidth}
\begin{alltt}
  SET( \LOCK, data_item, FALSE );
\end{alltt}
\end{boxedminipage}

See section \nameref{fudatastatements} page \pageref{fudatastatements}. \\
\item The abort statement stops the execution of the task and 
prints out a specified message
\index{ABORT@\ABORT!release note}

\begin{boxedminipage}[t]{\linewidth}
\begin{alltt}
  \ABORT ("An error has occured");
\end{alltt}
\end{boxedminipage}

See section \nameref{fuprintstatements} page \pageref{fuprintstatements}. \\
\item Messagebox displays a window with ok-button.
\index{MESSAGEBOX@\MESSAGEBOX!release note}

\begin{boxedminipage}[t]{\linewidth}
\begin{alltt}
  \MESSAGEBOX ("Text to be displayed");
\end{alltt}
\end{boxedminipage}

See section \nameref{fumessagebox} page \pageref{fumessagebox}. \\
\item To exit a function use the return statement

\begin{boxedminipage}[t]{\linewidth}
\begin{alltt}
  \RETURN;
\end{alltt}
\end{boxedminipage}

See section \nameref{fumessagebox} page \pageref{fumessagebox}. \\
\item The opening or execution of previously declared 
forms, tasks, functions or processgroups
may be enabled or disabled:
\index{ALLOW@\ALLOW!release note}
\index{DISALLOW@\DISALLOW!release note}

\begin{boxedminipage}[t]{\linewidth}
\begin{alltt}
  \ALLOW{} form_identifier;
  \DISALLOW{} form_identifier;
\end{alltt}
\end{boxedminipage}

See section \nameref{key:map} page \pageref{key:map}. \\
\index{DISABLE@\DISABLE!release note}
\index{ENABLE@\ENABLE!release note}
\item data fields or menu items may be enabled or disabled

\begin{boxedminipage}[t]{\linewidth}
\begin{alltt}
  \ENABLE{} fieldgroup_identifier;
  \DISABLE{} menu_identifier;
\end{alltt}
\end{boxedminipage}

See section \nameref{key:map} page \pageref{key:map}. \\
\item Previously defined index objects are referrable by their identifiers

\begin{boxedminipage}[t]{\linewidth}
\begin{alltt}
  {\bfseries indexIdentifier} = 3;
\end{alltt}
\end{boxedminipage}

See section \nameref{key:map} page \pageref{key:map}. \\
\item To create a new cycle use newcycle instead of appendcycle\\
The syntax has changed:
\begin{itemize}
\item Version 2 \\
  {\bfseries APPENDCYCLE};
\item Version 3 \\
  \NEWCYCLE;
\end{itemize}
See section \nameref{fucyclestatements} page \pageref{fucyclestatements}. \\
\index{DELETECYCLE@\DELETECYCLE!release note}
\item To remive recent cycle use deletecycle

\begin{boxedminipage}[t]{\linewidth}
\begin{alltt}
  \DELETECYCLE;
\end{alltt}
\end{boxedminipage}

See section \nameref{fucyclestatements} page \pageref{fucyclestatements}. \\
\item Using gocycle you can switch to any cycle you want
\index{GOCYCLE@\GOCYCLE!release note}

\begin{boxedminipage}[t]{\linewidth}
\begin{alltt}
  \GOCYCLE( 21 );
\end{alltt}
\end{boxedminipage}

See section \nameref{fucyclestatements} page \pageref{fucyclestatements}. \\
\item Clearcycle clears values of entire datapool of the current cycle
\index{CLEARCYCLE@\CLEARCYCLE!release note}

\begin{boxedminipage}[t]{\linewidth}
\begin{alltt}
  \CLEARCYCLE;
\end{alltt}
\end{boxedminipage}

See section \nameref{fucyclestatements} page \pageref{fucyclestatements}. \\

%% expressions
\item The current cycle is returned by the getcycle expression
\index{GETCYCLE@\GETCYCLE!release note}

\begin{boxedminipage}[t]{\linewidth}
\begin{alltt}
  data_item = \GETCYCLE;
\end{alltt}
\end{boxedminipage}

See section \nameref{fuexpressionsdata} page \pageref{fuexpressionsdata}. \\
\item The number of cycles is returned by maxcycle
\index{MAXCYCLE@\MAXCYCLE!release note}

\begin{boxedminipage}[t]{\linewidth}
\begin{alltt}
  data_item = \MAXCYCLE;
\end{alltt}
\end{boxedminipage}

See section \nameref{fuexpressionsdata} page \pageref{fuexpressionsdata}. \\
\item To test values use function expression valid. \\
The syntax has changed:
\index{OLDVALUE@\OLDVALUE!release note}
\index{INPUT@\INPUT!release note}
\index{VALID@\VALID!release note}
\begin{itemize}
\item Version 2 \\
data\_item.VALID\\
INPUT.VALID\\
OLDVALUE.VALID\\
\item Version 3 \\
VALID( data\_item )\\
VALID( INPUT )\\
VALID( OLDVALUE )\\
\end{itemize}
See section \nameref{fuexpressionsfunctions} page \pageref{fuexpressionsfunctions}. \\
\index{COMPLEX@\COMPLEX!release note}
\index{ARG@\ARG!release note}
\index{REAL@\REAL!release note}
\item Several new function expressions have been introduced to handle complex data

\begin{boxedminipage}[t]{\linewidth}
\begin{alltt}
  data_item = \REAL( complex_item );
  data_item = \IMAG( complex_item );
  data_item = \ARG( complex_item );
  complex_item = \COMPLEX( real_val, imag_val );
\end{alltt}
\end{boxedminipage}

See section \nameref{fuexpressionsfunctions} page \pageref{fuexpressionsfunctions}. \\
\item Function expression abs returns the absolute value
\index{ABS@\ABS!release note}

\begin{boxedminipage}[t]{\linewidth}
\begin{alltt}
  data_abs = \ABS( data_item );
\end{alltt}
\end{boxedminipage}

See section \nameref{fuexpressionsfunctions} page \pageref{fuexpressionsfunctions}. \\
\item Data is rounded by using one of three new round expressions
\index{ROUND@\ROUND!release note}
\index{ROUND05@\ROUNDFIVE!release note}
\index{ROUND10@\ROUNDTEN!release note}

\begin{boxedminipage}[t]{\linewidth}
\begin{alltt}
  result = {\bfseries ROUND}( data_item, 2 );
  result = {\bfseries ROUND5}( data_item );
  result = {\bfseries ROUND10}( data_item );
\end{alltt}
\end{boxedminipage}

See section \nameref{fuexpressionsfunctions} page \pageref{fuexpressionsfunctions}. \\
\index{CONFIRM@\CONFIRM!release note}
\item Data may be confirmed before storing into datapool variables

\begin{boxedminipage}[t]{\linewidth}
\begin{alltt}
  result = \CONFIRM( "Do it?" );
\end{alltt}
\end{boxedminipage}

See section \nameref{fuexpressionsfunctions} page \pageref{fuexpressionsfunctions}. \\
\item The function statements LABEL() and UNIT()
 return the string value of the label or unit
 attributes:
\index{LABEL@\LABEL!function expression!release note}
\index{UNIT@\UNIT!function expression!release note}

\begin{boxedminipage}[t]{\linewidth}
\begin{alltt}
  FUNC display \{
    PRINT ( "Label: ", \LABEL( dataitem ), EOLN );
    PRINT ( "Units: ", \UNIT( dataitem ), EOLN );
  \}
\end{alltt}
\end{boxedminipage}

See section \nameref{fuexpressionsfunctions} page 
  \pageref{fuexpressionsfunctions}.

\item Reason expressions are to test what was the reason of calling the function

\begin{boxedminipage}[t]{\linewidth}
\begin{alltt}
  IF (\REASONINPUT) \{ PRINT ("you entered ", INPUT); \}
  IF (\REASONINSERT) \{ \ldots \}
  IF (\REASONDUPLICATE) \{ \ldots \}
  IF (\REASONCLEAR) \{ \ldots \}
  IF (\REASONREMOVE) \{ \ldots \}
  IF (\REASONPACK) \{ \ldots \}
  IF (\REASONSELECT) \{ \ldots \}
  IF (\REASONUNSELECT) \{ \ldots \}
  IF (\REASONACTIVATE) \{ \ldots \}
  IF (\REASONDROP) \{ \ldots \}
\end{alltt}
\end{boxedminipage}

See section \nameref{fuexpressionsreason} page \pageref{fuexpressionsreason}. \\
\end{enumerate}
%%
%%---------------------------------------------------------------
\subsubsection{Stream}
\begin{enumerate}
\item Using \VAR{} to reference datapool items the reference may be changed at runntime \\
\index{VAR@\VAR!streamer!release note}

\begin{boxedminipage}[t]{\linewidth}
\begin{alltt}
  DATAPOOL
    STRING \{EDITABLE\} data\_ref;
    REAL data\_item;
  END DATAPOOL;
  FUNCTIONS
    FUNC INIT \{ data\_ref = "data\_item"; \};
  END FUNCTIONS;
  STREAMER
    stream\_identifier (\VAR(data_ref), EOLN);
  END STREAMER;
\end{alltt}
\end{boxedminipage}

See section \nameref{sec:stvariables} page \pageref{sec:stvariables}. \\
\index{XML@\XML!streamer format command!release note}
\item The XML format command is used to read or write xml-files into or from
datapool data-items using the filestream \\

\begin{boxedminipage}[t]{\linewidth}
\begin{alltt}
  STREAMER
    stream\_identifier \{\XML\} (data\_item, EOLN);
  END STREAMER;
\end{alltt}
\end{boxedminipage}

Using the keyword \DATAPOOL{} instead of data\_item\_identifiers,
the entire datapool is writen or read. \\
See section \nameref{sec:stxmlcommand} page \pageref{sec:stxmlcommand}. \\
\index{RANGE@\RANGE!release note}
\item Plotgroup option RANGE \\

\begin{boxedminipage}[t]{\linewidth}
\begin{alltt}
  PLOTGROUP = "plot_id" \{\RANGE ( 2, 6 )\};
\end{alltt}
\end{boxedminipage}

See section \nameref{fig:st_format_command} page \pageref{fig:st_format_command}. \\
\index{DATASET\_TEXT@\DATASETTEXT!release note}
\item dataset\_text \\
The syntax has changed:
\begin{itemize}
\item Version 2 \\
\DATASETTEXT.data\_item \\
\item Version 3 \\
\DATASETTEXT (data\_item) \\
\end{itemize}
See section \nameref{fig:st_format_command} page \pageref{fig:st_format_command}. \\

\item filestream \\
\index{FILESTREAM@\FILESTREAM!release note}
\index{STREAM@STREAM(old syntax)!release note}
The syntax has changed:
\begin{itemize}
\item Version 2 \\
STREAM p\_unt\_stream : \FILESTREAM \{ "Mech Tractive Unit (*.unt)" ...... \} \\
\item Version 3 \\
\FILESTREAM{} fs\_p\_unt\_stream = p\_unt\_stream \{ "Mech Tractive Unit (*.unt)" ...... \} \\
\end{itemize}
See section \nameref{sec:opfilestreams} page \pageref{sec:opfilestreams}. \\
\item filestream option \NOLOG{} suppresses log messages.
See section \nameref{sec:opfilestreams} page \pageref{sec:opfilestreams}. \\
\item filestream option \EXTENSION{} defines file extension added to filename.\\
See section \nameref{sec:opfilestreams} page \pageref{sec:opfilestreams}. \\
\item reportstream \\
The syntax has changed:
\begin{itemize}
\index{REPORTSTREAM@\REPORTSTREAM!release note}
\item Version 2 \\
STREAM stream\_id : \REPORTSTREAM \{ "Menu label" ..... \} \\
\item Version 3 \\
\REPORTSTREAM{} rep\_stream = stream\_id \{ "Menu label" .... \} \\
\end{itemize}
See section \nameref{sec:opreportstreams} page \pageref{sec:opreportstreams}. \\
\index{FILTER@\FILTER!reportstream option!release note}
\item Rerportstream option filter defines a unix process 
where the stream is piped to. \\
See section \nameref{sec:opreportstreams} page \pageref{sec:opreportstreams}. \\
\end{enumerate}
%%
%%---------------------------------------------------------------
\subsubsection{Process}
\begin{enumerate}
\index{DAEMON@\DAEMON!release note}
\index{PROCESS@\PROCESS!release note!daemon}
\item Daemon processes are just started but not stopped any more.\\
No data may be transferred from or to it: \\

\begin{boxedminipage}[t]{\linewidth}
\begin{alltt}
  \PROCESS process_id: \DAEMON {"xterm"};
\end{alltt}
\end{boxedminipage}

See section \nameref{sec:opprocess} page \pageref{sec:opprocess}. \\
\item Usergroups-option allows to predefine which users 
are allowed to use the process. \\
See section \nameref{sec:opprocessgroup} page \pageref{sec:opprocessgroup}. \\
\item Option \NOLOG{} does suppress log messages.\\
See section \nameref{sec:opprocessgroup} page \pageref{sec:opprocessgroup}. \\
\item Option \HIDDEN{} hides the processgroup from menu. \\
See section \nameref{sec:opprocessgroup} page \pageref{sec:opprocessgroup}. \\
\index{FORM@\FORM!processgroup option!relese note}
\item Option \FORM=\NONE{} hides the push-button also. \\
See section \nameref{sec:opprocessgroup} page \pageref{sec:opprocessgroup}. \\
\index{FORMAT}
\item the output stream option \FORMAT{} has been removed. \\
\end{enumerate}
%%
%%---------------------------------------------------------------
\subsubsection{ProcessGroup}
\begin{enumerate}
\index{PROCESS@\PROCESS!release note!ui\_update}
\item The UI\_UPDATE option updates the user interface after specified lines
are received from an external process. \\

\begin{boxedminipage}[t]{\linewidth}
\begin{alltt}
  PROCESS process_id: BATCH \{"cat megafile.dat"\};
  PROCESSGROUP pg_id \{\UIUPDATE(10)\} (
    streamer_id = process_id();
  );
\end{alltt}
\end{boxedminipage}

See section \nameref{sntxdiagram:processgroupoptions} page \pageref{sntxdiagram:processgroupoptions}. \\
\end{enumerate}
%%
%%---------------------------------------------------------------
\subsubsection{Task}
\begin{enumerate}
\item Usergroups-option allows to predefine which users 
are allowed to use the process. \\
See section \nameref{sec:optasks} page \pageref{sec:optasks}. \\
\item Option \NOLOG{} suppresses log messages.\\
See section \nameref{sec:optasks} page \pageref{sec:optasks}. \\
\item Option \HIDDEN{}  hides the processgroup from menu. \\
See section \nameref{sec:optasks} page \pageref{sec:optasks}. \\
\index{FORM@\FORM!task option!relese note}
\item Option \FORM=\NONE{} hides the push-button also. \\
See section \nameref{sec:optasks} page \pageref{sec:optasks}. \\
\item the output stream option FORMAT has been removed. \\
\end{enumerate}
%%
%%---------------------------------------------------------------
\subsubsection{Database}
\begin{enumerate}
\index{FILTER@\FILTER!db\_manager!release note}
\item filter\\
The syntax has changed:
\begin{itemize}
\item Version 2 \\
Select\_Motor  {\bfseries : \FILTER} \{"Select Motor", RESET\} (....) \\
\item Version 3 \\
\FILTER{} Select\_Motor   \{"Select Motor", RESET\} (....) \\
\end{itemize}
See section \nameref{sec:dbfilter} page \pageref{sec:dbfilter}. \\
\index{FORM@\FORM!db\_manager option!relese note}
\item The \FORM{} option creates the push button on a predefined form.

\begin{boxedminipage}[t]{\linewidth}
\begin{alltt}
  TRANSACTION trans_id \{ \FORM = form_id \} (\ldots
  TRANSACTION trans_id \{ \FORM = \NONE \} (\ldots
\end{alltt}
\end{boxedminipage}

See section \nameref{sec:dbtransactionoption} page \pageref{sec:dbtransactionoption}. \\
\item The helptext appears by the mouse pointer moving over the push button.

\begin{boxedminipage}[t]{\linewidth}
\begin{alltt}
  TRANSACTION trans_id \{ \HELPTEXT = "load data" \} (\ldots
\end{alltt}
\end{boxedminipage}

See section \nameref{sec:dbtransactionoption} page \pageref{sec:dbtransactionoption}. \\
\item Logon and logoff from Database by menu
\index{LOGON@\LOGON!relese note}
\index{LOGOFF@\LOGOFF!relese note}

\begin{boxedminipage}[t]{\linewidth}
\begin{alltt}
  MENU menu_id (
    \LOGON "login to database"
    \LOGOFF "logout from database"
  );
\end{alltt}
\end{boxedminipage}

See section \nameref{sec:dbmenu} page \pageref{sec:dbmenu}. \\

\end{enumerate}
%%
%%---------------------------------------------------------------
%%---------------------------------------------------------------
