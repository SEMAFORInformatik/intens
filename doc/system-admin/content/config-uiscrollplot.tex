%%%%%%%%%%%%%%%%%%%%%%%%%%%%%%%%%%%%%%%%%%%%%%%%%%%%%%%%%%%%%%%%%%%%%%%%%%%%%
%%%                              SCROLLPLOT                               %%%
%%%%%%%%%%%%%%%%%%%%%%%%%%%%%%%%%%%%%%%%%%%%%%%%%%%%%%%%%%%%%%%%%%%%%%%%%%%%%
\subsubsection{Scrollplot}
\label{sec:uiscrollplot}
The plot type
\SCROLLPLOT{} provides a very flexible means to display
a user selectable number of datapool list values
in a 2-dimensional diagram. The list of these items
has to be defined in the description file. At run time one or more values from
this list can be selected by pressing the {\bfseries Load}-button.
\index{Plot!Scrollplot}

There are two stretch bars, a horizontal one at the lower edge of the diagram and
a vertical one at the right edge, which allow to stretch and move the curves 
horizontally and vertically.

A region can be zoomed by selecting it with the left mouse button and then
pressing the {\bfseries Zoom} button.
\begin{figure}\label{fig:scrollplot}
   \begin{center}
      \includegraphics[width=\linewidth]{grab_scrollplot}
   \end{center}
\caption{example of \SCROLLPLOT}
\end{figure}
\vspace{1cm}

\includegraphics[width=\linewidth]{xfig/sntx_ui_scrollplot}
\index{SCROLLPLOT@\SCROLLPLOT}
\index{XAXIS@\XAXIS}
\index{YAXIS@\YAXIS}
\vspace{1cm}


\begin{boxedminipage}[t]{\linewidth}
\begin{alltt}
\DESCRIPTION "Example SCROLLPLOT";
  ...
\UIMANAGER
  \SCROLLPLOT
    scroll_plot\{ "Scroll Plot" \} (
      \XAXIS (
        xA = "Data A",
        xB = "Data B",
        xC = "Data C"
      );
      \YAXIS (
        yA = "Data A",
        yB = "Data B",
        yC = "Data C"
      );
    );
\END \UIMANAGER;
  ...
\END.
\end{alltt}
\end{boxedminipage}

