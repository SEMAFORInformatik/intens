\newpage
\subsection{FUNCTIONS}
\label{sec:functions}
%%%%%%%%%%%%%%%%%%%%%%%%%%%%%%%%%%%%%%%%%%%%%%%%%%%%%%%%%%%%%%%%%%%%%%%%%%%%%
%%%                             Description                               %%%
%%%%%%%%%%%%%%%%%%%%%%%%%%%%%%%%%%%%%%%%%%%%%%%%%%%%%%%%%%%%%%%%%%%%%%%%%%%%%
\subsubsection{Description}
\label{sec:fudescription}
Functions are mostly used to check user input (see section \nameref{sec:datapool} on page \pageref{sec:datapool}). \\

\input{diagrams/functions_description}
\input{diagrams/job_function_declaration}
\index{FUNCTIONS@\FUNCTIONS}
\index{FUNC@\FUNC}

\input{diagrams/job_option}

\label{sec:funcoptions}
\index{SILENT@\SILENT!function attributes}
\index{UPDATE\_FORMS@\UPDATEFORMS!function attributes}
\index{NONE@\NONE!function attributes}
\index{PRIORITY@\PRIORITY!function attributes}
\index{HIGH@\HIGH!function attributes}
\index{PROTO@\PROTO!function attributes}
\begin{tabularx}{\textwidth}{l|X}
option & description \\
\hline
\SILENT            & The functions write nothing to the log window (except \PRINT{}, \SETERROR{} and \ABORT{} messages),
                     the status line and no busy cursor appears. \\
\UPDATEFORMS=\NONE & The forms are not updated when the function is done. \\
\PRIORITY=\HIGH    & When a user action, a message queue message, a timer starts a function
                     and another function is already running, the new function is normally added
                     at the end of the function queue: it will run after all other waiting
                     functions finished. \newline
                     This option adds the function at the beginning of the function queue: it will
                     run before the other waiting functions. \\
\PROTO             & TODO. \\
\end{tabularx}

\input{diagrams/job_function}

\label{sec:funcidentifiers}
\index{FUNCTIONS@\FUNCTIONS!\INIT}
\index{FUNCTIONS@\FUNCTIONS!\QUIT}
\index{FUNCTIONS@\FUNCTIONS!\ONCYCLEEVENT}
\index{FUNCTIONS@\FUNCTIONS!\ONCYCLESWITCH}
\index{FUNCTIONS@\FUNCTIONS!\AFTERUPDATEFORMS}
\index{INIT@\INIT!function identifier}
\index{QUIT@\QUIT!function identifier}
\index{ON\_CYCLE\_EVENT@\ONCYCLEEVENT!function identifier}
\index{ON\_CYCLE\_SWITCH@\ONCYCLESWITCH!function identifier}
\index{AFTER\_UPDATE\_FORMS@\AFTERUPDATEFORMS!function identifier}

The following function identifiers have a special meaning.
If they are defined, INTENS calls these functions automatically in certain situations.

As of version 5.2.1, INTENS no longer calls these functions after cycle statements
(see section \nameref{fu:cycle:statement} on page \pageref{fu:cycle:statement}).

\begin{tabularx}{\textwidth}{l|X}
function identifier & description \\
\hline
\INIT             & \begin{itemize}[noitemsep,topsep=0pt,partopsep=0pt]
                    \item at startup
                    \item after clear cycle from Cycle Dialog
                    \end{itemize} \\
\QUIT             & \begin{itemize}
                    \item before quitting INTENS
                    \end{itemize}\\
\ONCYCLEEVENT     & \begin{itemize}
                    \item after cycle events (clear, delete, new, rename, switch)
                    from Cycle Dialog\footnotemark[\value{footnote}].
                    \end{itemize}
                    Within the function, the event can be distinguished using
                    \begin{itemize}
                    \item \REASONCYCLECLEAR
                    \item \REASONCYCLEDELETE
                    \item \REASONCYCLENEW
                    \item \REASONCYCLERENAME
                    \item \REASONCYCLESWITCH
                    \end{itemize} \\
\ONCYCLESWITCH    & \begin{itemize}[noitemsep,topsep=0pt,parsep=0pt,partopsep=0pt]
                    \item after cycle events (delete, new, switch)
                    from Cycle Dialog\footnotemark[\value{footnote}]
                    \end{itemize} \\
\AFTERUPDATEFORMS & \begin{itemize}
                    \item after a gui update (after forms are updated)
                    \end{itemize} \\
\end{tabularx}

\footnotetext{Menu Options > Cycle or ctrl-l}

% \vspace{0.5cm}
% See examples on page \pageref{releasenotesfuncinit}.

\input{diagrams/job_local_variables}

\Slanted{Job local variables} are defined just like \Slanted{data items}
(see section \nameref{sec:dpitem} on page \pageref{sec:dpitem}).
The only difference are the \Slanted{data variable attributes} which are not available
for \Slanted{job local variables}.


%%%%%%%%%%%%%%%%%%%%%%%%%%%%%%%%%%%%%%%%%%%%%%%%%%%%%%%%%%%%%%%%%%%%%%%%%%%%%
%%%                             Statements                                %%%
%%%%%%%%%%%%%%%%%%%%%%%%%%%%%%%%%%%%%%%%%%%%%%%%%%%%%%%%%%%%%%%%%%%%%%%%%%%%%
\subsubsection{Statements}
\label{fustatements}
\input{diagrams/job_statement}

Function statements have the same syntax as task statements (OPERATOR section \nameref{sec:optasks}). \\

\input{diagrams/job_single_statement}

\input{diagrams/job_data_reference}
\label{fu:data:reference}
\index{VAR@\VAR!function command}
\index{PARENT@\PARENT!function command}
\index{DATASET\_TEXT@\DATASETTEXT!function command}
\index{STRING\_DATE@\STRINGDATE!function command}
\index{STRING\_TIME@\STRINGTIME!function command}
\index{STRING\_DATETIME@\STRINGDATETIME!function command}
\index{STRING\_VALUE@\STRINGVALUE!function command}
\index{CURRENT\_TIME@\CURRENTTIME!function expression}
\index{CURRENT\_DATE@\CURRENTDATE!function expression}
\index{CURRENT\_DATETIME@\CURRENTDATETIME!function expression}
\index{INPUT@\INPUT!function expression}
\index{SOURCE@\SOURCE!function expression}
\index{SOURCE2@\SOURCETWO!function expression}
\index{THIS@\THIS!function expression}
\index{BASE@\BASE!function expression}

\begin{tabularx}{\textwidth}{l|X}
statements        & description \\
\hline
\VAR(string-item) & references a datapool item element. \newline
                    String dataitem contains the identifier of a data item.
                    By changing the contents of string-item at runtime, specified datapool item is referenced. \\
\PARENT           & references parent structure of \verb+job_expression+ \\
\DATASETTEXT      & references the string of the \SET{} associated with the item (can only be
                    used if the item has a \SET) \\
{\bfseries STRING\_*}  & see section \nameref{fig:st_format_command} on page \pageref{fig:st_format_command} \\
{\bfseries CURRENT\_*}  & references the current time, date or datetime \\
\INPUT            & references the user input of the field that called the function \\
\SOURCE           & references structure of data item, which was dragged or connected. (see example on page \pageref{navigatordragndrop}) \\
\SOURCETWO        & references second structure of data item, which was connected (see section \nameref{sec:uinavigatorDiagram} page \pageref{sec:uinavigatorDiagram}). \\
\THIS             & references structure of data item, which called the function. (see \nameref{fuexample2} on page \pageref{fuexample2}, \pageref{navigatordragndrop}) \\
\BASE             & references structure of called function. (see \nameref{fuexample3} on page \pageref{fuexample3}) \\
\end{tabularx}


\input{diagrams/job_data_identifier_options}
\index{  @Signs / Characters!\texttt{"@} (at)!cycle number }

\begin{tabularx}{\textwidth}{l|X}
statements & description \\
\hline
{\verb+indexes+} & references a specific element of a (possibly multidimensional) array \\
{\verb+struct_reference_list+} & reference a specific element of a structure (i.E. motor.weight) \\
{\bfseries @}    & reference an element in a specific \CYCLE \newline
                   default is current \CYCLE \\
\end{tabularx}

\input{diagrams/job_data_indizes}
\label{fu:data:indizes}

\input{diagrams/job_data_struct_reference_list}
\label{fu:data:struct:reference:list}

\begin{tabularx}{\textwidth}{l|X}
statements        & description \\
\hline
\VAR(string-item) & references a datapool item element. \newline
                    String dataitem contains the identifier of a data item.
                    By changing the contents of string-item at runtime, specified datapool item is referenced. \newline
                    Note: only one \VAR{} is allowed in a \verb+job_data_reference+. \\
\end{tabularx}



\input{diagrams/data_statement}

\label{fu:data:statement}
\index{  @Signs / Characters!=!function statement}
\index{  @Signs / Characters!++!function statement}
\index{  @Signs / Characters!+=!function statement}
\index{  @Signs / Characters!- -!function statement}
\index{CLEAR@\CLEAR!function statement}
\index{ERASE@\ERASE!function statement}
\index{PACK@\PACK!function statement}
\index{ROW@\ROW!function statement}
\index{COL@\COL!function statement}
\index{SIZE@\SIZE!function statement}
\index{DATASIZE@\DATASIZE!function statement}
\index{INDEX@\INDEX!function statement}
\index{COMPARE@\COMPARE!function statement}
\index{ASSIGN\_CORR@\ASSIGNCORR!function statement}
\index{SET\_RESOURCE@\SETRESOURCE!function statement}
\index{COMPOSE@\COMPOSE!function statement}
\index{ASSIGN\_CONSISTENCY@\ASSIGNCONSISTENCY!function statement}
\begin{tabularx}{\textwidth}{l|X}
data statement    & description \\
\hline
{\verb+job_data_reference+}& references a (possibly indexed) datapool element. \\
\CLEAR            & clears all values of the specified datapool item elements. \\
{\bfseries ID\_TEXTWINDOW} & clears the content of the text window.
                    (see section \nameref{sec:uitextwindow} on page \pageref{sec:uitextwindow}) \\
\STDWINDOW        & clears the content of the text window \STDWINDOW.
                    (see section \nameref{sec:uitextwindow} on page \pageref{sec:uitextwindow}) \\
\LOGWINDOW        & clears the content of the text window \LOGWINDOW.
                    (see section \nameref{sec:uitextwindow} on page \pageref{sec:uitextwindow}) \\
\ERASE            & clears all values and attributes (color, etc.) of the specified datapool item elements. \\
\PACK             & removes empty entries \newline
                    {\verb+job_data_reference+} should have wildcard(s) (nothing is done otherwise). \newline
                    The wildcard(s) specify the dimension on which \PACK{} operates. \\
\ROW              & (=default) with two wildcards, empty rows (first wildcard) are removed. \\
\COL              & with two wildcards, empty columns (second wildcard) are removed. \\
\SIZE             & stores the number of elements of the first argument into the second. \newline
                    When the first argument has multiple dimentions (on the last level), the
                    second stores a list of sizes: \newline
                    data.value[2,3] = 0; \SIZE(data.value, size); -> size[0] is 3, size[1] is 4. \newline
                    The first argument may have wildcards. The second argument gets the size of every
                    wildcard: \SIZE(data[*].value[*,*], size); -> size[0] is size of data, size[1] is size of
                    first dimension of data.value, size[2] is size of second dimension of data.value \\
\DATASIZE         & stores the size of the first argument (number of characters when written as JSON) into the second. \newline
                    \TRANSIENT{} items are ignored (not counted) \newline
                    Can be used to know the data size of an object before if is sent to the database. \\
\INDEX            & \INDEX{} is used to know the index of the field that called the function. \newline
                    stores up the index of the first argument into the second. \newline
                    The first argument may be \THIS, \PARENT(\THIS), \PARENT(\PARENT(\THIS)), etc., \BASE. \newline
                    Example: data[2].value[3] was changed on the GUI and a function was called.
                    After {\verb+INDEX(THIS, idx);+}, idx is 2. \newline
                    See \nameref{dia:dataexpression} on page \pageref{dia:dataexpression}
                    and \nameref{dia:functionexpression} on page \pageref{dia:functionexpression}
                    for other \INDEX{} expressions. \\
\COMPARE          & compare structure data. All arguments must be of the same data type (usually a Structure).
                    With a third argument, the second and third arguments are compared.
                    Without a third argument, the values of the second argument in all cycles are compared. \newline
                    The result is written to the first argument. It can be used in
                    \begin{itemize}
                      \item a NAVIGATOR with the option \COMPARE{}
                        (see section \nameref{sec:uinavigator} on page \pageref{dia:uinavigatoroptionlist}).
                      \item the compare dialog: fill COMPAREDIALOG struct and call MAP ( COMPARE\_DIALOG );
                    \end{itemize} \\
\ASSIGNCORR       & assign corresponding elements from second argument structure to first argument structure. \\
\SETRESOURCE      & Set value of a resource property (with name {\bfseries string}). \newline
                    The value is written to the user resource file when the \INTENS{} application
                    is quit or when \WRITESETTINGS{} is called.
                    It can then be used when the \INTENS{} application is started next time(s). \\
\COMPOSE          & Compose string: write values into string. Needed for translatable strings. \newline
                    The (format-)string (first argument) contains \%1, \%2, ... and the value(s) of the
                    corresponding data reference is put in these places. \newline
                    Up to 15 values are implemented. \\
\ASSIGNCONSISTENCY & Assign the (value of the) second argument to the first. \newline
                    \INVALID: clear the first arguments value. \newline
                    Then, clear all dependent results
                    (see paragraph \nameref{par:stdependency} on page \pageref{par:stdependency}). \\
\end{tabularx}

\input{diagrams/file_statement}

\index{OPEN@\OPEN!function statement}
\index{LOAD@\LOAD!function statement}
\index{SAVE@\SAVE!function statement}
\index{SERIALIZE@\SERIALIZE!function statement}
\index{SERIALIZE\_FORM@\SERIALIZEFORM!function statement}
\index{WRITE\_SETTINGS@\WRITESETTINGS!function statement}
\index{XML@\XML!function statement option}
\index{JSON@\JSON!function statement option}
\index{PROTO@\PROTO!function statement option}
\begin{tabularx}{\textwidth}{l|X}
data statement          & description \\
\hline
\OPEN                   & Read a file and write its content to the \DATAPOOL{}. \newline
                          The \STREAM{} of the \FILESTREAM{} (first argument) defines
                          the files format and the variables to write to. \newline
                          Opens the file OPEN dialog unless the filename
                          is given as the second argument.\\
\verb+job data reference+ & string item containing the filename \\
                        & No file selection dialog window is opened and
                          the \OPEN{} or \SAVE{} operation is done using this filename. \\
                        & References a data item declared in the datapool
                          (section \nameref{sec:tempdatareference} page \pageref{sec:tempdatareference}). \\
\LOAD                   & Parse the content of the \STRING{} data reference. \\
\SAVE                   & Saves the contens of the first argument to a file. \newline
                          Opens the file SAVE dialog window unless the filename
                          is given as the second argument. \\
{\bfseries ID\_STREAM}  & Saves the content of the stream.
                          (see section \nameref{sec:streamer} on page \pageref{sec:streamer}) \\
{\bfseries ID\_REPORTSTREAM}  & Saves the content of the reportstream.
                          (see section \nameref{sec:reportstreams} on page \pageref{sec:reportstreams}) \\
{\bfseries ID\_TEXTWINDOW}    & Saves the content of the text window.
                          (see section \nameref{sec:uitextwindow} on page \pageref{sec:uitextwindow}) \\
\STDWINDOW              & Saves the content of the text window \STDWINDOW.
                          (see section \nameref{sec:uitextwindow} on page \pageref{sec:uitextwindow}) \\
\LOGWINDOW              & Saves the content of the text window \LOGWINDOW.
                          (see section \nameref{sec:uitextwindow} on page \pageref{sec:uitextwindow}) \\
\SERIALIZE              & Writes the definition and content of a ui element to a file. \\
\SERIALIZEFORM          & TODO \\
\WRITESETTINGS          & Write settings to user resource file now - as done when the \INTENS{} application is quit. \\
\XML                    & Use \XML{} format. \\
\JSON                   & Use \JSON{} format. \\
\PROTO                  & TODO. \\
\end{tabularx}

\input{diagrams/set_statement}
\input{diagrams/job_data_reference_boolean}

These statements are used to dynamically change the attributes of DATAPOOL elements at runtime. \\
\label{fu:set:statement}

\index{SET@\SET!function statement}
\index{SET!LOCK}
\index{SET!EDITABLE}
\index{SET!COLOR}
\index{SET!COLORBIT}
\index{SET!TIMESTAMP}
\index{UNSET@\UNSET!function statement}
\index{UNSET!LOCK}
\index{UNSET!EDITABLE}
\index{UNSET!COLORBIT}
\index{UNSET!TIMESTAMP}
\index{LOCK@\LOCK!function statement (set/unset)}
\index{EDITABLE@\EDITABLE!function statement (set/unset)}
\index{TIMESTAMP@\TIMESTAMP!function statement (set/unset)}
\index{COLOR@\COLOR!function statement (set)}
\index{COLORBIT@\COLORBIT!function statement (set/unset)}
\index{FALSE@\FALSE!function statement (set/unset)}
\index{TRUE@\TRUE!function statement (set/unset)}
\index{TOUCH@\TOUCH!function statement}
\begin{tabularx}{\textwidth}{l|X}
set statement    & description \\
\hline
{\verb+bool value+} & \TRUE{} or \FALSE. If omitted, TRUE is the default value. \SET{} FALSE is equivalent to \UNSET. \\
\LOCK               & See \LOCKABLE{} attribute (section \nameref{dataitemattributes} page \pageref{dataitemattributes}). \\
\EDITABLE           & See \EDITABLE{} attribute (section \nameref{dataitemattributes} page \pageref{dataitemattributes}). \\
\TIMESTAMP          & Efects the modify status. (See also DB\_MANAGER MODIFY page \pageref{sec:dbmodify}) \\
\COLOR              & SET the color to corresponding value 1-8 (or 1 - 255 with the \NOCOLORBIT{} options). \\
                    & Value 0 is used to reset color settings. \\
                    & Previously set colors will be lost. \\
\COLORBIT           & \SET{} the color to corresponding value 1-8. \\
                    & \UNSET ting higher level color (smaller value), the item is displayed whith next lower lever color. \\
\TOUCH              & Sets the attribute timestamp of \verb+job_data_reference+. This causes the next gui update
                      to update the gui elements. \newline
                      The data timestamp is not touched, so \verb+MODIFIED ( job_data_reference )+ stays at it was. \\
%%                    & See example at release notes page \pageref{relnotescolorbit}. \\
\end{tabularx}

\input{diagrams/gui_statement}
\input{diagrams/job_map_element}
\input{diagrams/job_unmap_element}
\label{fu:gui:statement}
\label{key:map_form}
\index{UPDATE\_FORMS@\UPDATEFORMS!function statement}
\index{UI\_UPDATE@\UIUPDATE!function statement}
\index{MAP@\MAP!function statement}
\index{OMIT\_TTRAIL@\OMITTTRAIL!function statement option}
\index{OMIT\_ACTIVATE@\OMITACTIVATE!function statement option}
\index{UNMAP@\UNMAP!function statement}
\index{ALLOW@\ALLOW!function statement}
\index{DISALLOW@\DISALLOW!function statement}
\index{ENABLE@\ENABLE!function statement}
\index{DISABLE@\DISABLE!function statement}
\index{CYCLE@\CYCLE!function statement option}

\input{diagrams/gui_more_statement}
\input{diagrams/job_get_selection_element}
\input{diagrams/job_set_stylesheet_element}
\index{RANGE@\RANGE!function statement}
\index{SELECT\_LIST@\SELECTLIST!function statement}
\index{GET\_SELECTION@\GETSELECTION!function statement}
\index{CLEAR\_SELECTION@\CLEARSELECTION!function statement}
\index{GET\_SORT\_CRITERIA@\GETSORTCRITERIA!function statement}
\index{STYLESHEET@\STYLESHEET!function statement}
\index{REPLACE@\REPLACE!function statement}

\begin{tabularx}{\textwidth}{l|X}
gui statement         & description \\
\hline
{\verb+identifier+}   & previously declared identifiers \\
\UPDATEFORMS          & updates all forms \\
\UIUPDATE             & Update a gui element. \\
\MAP                  & shows the form, progressbar, folder tab, table line or gui element referenced by the identifier. \newline
                        \OMITTTRAIL : The folder tab mapping is not handled by transactions (ABORT, UNDO, ...). \newline
                        \OMITACTIVATE : The folder tab button is visible again (after an \UNMAP), but not activated. \newline
                        \verb+temp_data_reference+ : The value of a string variable is used as the identifier.
                        This can be used to map something different in different situations. \newline
                        To map a folder tab, use its {\bfseries ID\_FOLDERGROUP} or the identifier of the \FOLDER{},
                        followed by a \verb+:+ and the tab index (as a integer or a variable) \newline
                        To map a table line, the second argument is the data reference shown in that table line,
                        including the wildcards, as given in the \TABLE{} definition. \\
\UNMAP                & closes (form, progressbar) or hides (folder tab, table line, gui element) specified element. \newline
                        \verb+temp_data_reference+ : The value of a string variable is used as the identifier.
                        This can be used to map / unmap something different in different situations. \newline
                        It is possible to unmap (hide) menu entries. Here are a few examples:
                        \begin{itemize}
                        \item "File->Open"
                        \item "File->Save"
                        \item "File->Print"
                        \item "Option"
                        \item "Help->Copyright"
                        \end{itemize} \\
\ALLOW                & mapping, opening or execution of specified identifier is allowed. \\
\DISALLOW             & mapping, opening or execution of specified identifier is not allowed (Menu buttons are gray). \\
\ENABLE               & input fields (data items) or menu items are enabled (editable, choosable). \\
\DISABLE              & input fields (data items) or menu items are disabled (no entries are possible. Menu buttons are gray). \\
\CYCLE                & Disable or enable all input fields of a cycle. \\
{\bfseries ID\_INDEX} & previously declared index-object (page \pageref{sec:uiindex}) is set to expression. \\
\RANGE                & Set range of previously declared \FIELDGROUP{} (with option \TABLESIZE) (see section \nameref{sec:uifieldgroupotions}
                        on page \pageref{sec:uifieldgroupotions}) to the interval [first job\_expression, second job\_expression].
                        This may also reduce the number of rows/lines shown to less than \TABLESIZE{} if the \FIELDGROUP{} has the option \STRETCH{}. \\
\SELECTLIST           & Select the line(s) defined in the second argument (array with integer index(es)) in the \LIST{}, the \TABLE{} or the \NAVIGATOR. \\
\GETSELECTION         & \LIST{} and \NAVIGATOR: write the selected row(s) index(es) to the second argument (array with integer index(es)). \newline
                        \TABLE: write the selected cell(s) row and column index(es) to the second (rows) and third (cols)
                        arguments (array with integer index(es)). \newline
                        When (at least) one complete row and/or column is selected, the index(es) of all completely selected row(s) and/or column(s) are writen to the
                        second (rows) and third (cols). \newline
                        \PLOTTWOD: write the coordinates (x and y), the y-axis type (1 or 2) and the y-axis title of the
                        selected points to the arguments (x, y, axisType, axisLabel). \\
\CLEARSELECTION       & \LIST, \TABLE{} and \NAVIGATOR: unselect all elements \\
\GETSORTCRITERIA      & \LIST: write the current sort criteria to the variable (job\_data\_reference) \\
\STYLESHEET           & To customize their look, you can set the Qt Style Sheet (qss) string of a gui element or a data reference.
                        The second argument (a data reference), contains the string.
                        With a data reference, the qss string is passed to the gui field showing that data reference. \newline
                        Example (gui element): use a picture as the background of a \PLOTTWOD: \newline
                        \verb+"QwtPlotCanvas{background-repeat: no-repeat; border-image: url(./bitmaps/fans/xy.png) 0 0 0 0 stretch stretch;}"+ \newline
                        Example (data reference): show a data item with bold font: \newline
                        \verb+"QLineEdit{font-weight:bold;}"+ \\
\REPLACE              & Replace the first gui element by the second. \\
\end{tabularx}

\input{diagrams/job_enable_disable_cycle}

\input{diagrams/cycle_statement}
\label{fu:cycle:statement}

\index{NEWCYCLE@\NEWCYCLE!function statement}
\index{FIRSTCYCLE@\FIRSTCYCLE!function statement}
\index{LASTCYCLE@\LASTCYCLE!function statement}
\index{NEXTCYCLE@\NEXTCYCLE!function statement}
\index{CLEARCYCLE@\CLEARCYCLE!function statement}
\index{DELETECYCLE@\DELETECYCLE!function statement}
\index{GOCYCLE@\GOCYCLE!function statement}
\index{CYCLENAME@\CYCLENAME!function statement}

The cycle statement is used to make a copy of the entire datapool, switch between cycles, delete a cycle, etc.\\
\vspace{0.5cm}

\begin{tabularx}{\textwidth}{l|X}
cycle statement   & description \\
\hline
\FIRSTCYCLE       & switches to the first cycle\\
\LASTCYCLE        & switches to the last cycle\\
\NEXTCYCLE        & switches to the next cycle. If already in last cycle, a new cycle is created.\\
\CLEARCYCLE       & clears all values of recent cycle whithout removing it. And executes INIT-function. \\
\DELETECYCLE      & deletes current cycle (as long as another one remains)\\
\GOCYCLE          & switches to cycle no. '(int)' \\
\NEWCYCLE         & creates a new cycle (with optional name)\\
\CYCLENAME        & sets name of current or specified cycle\\
\end{tabularx}


\input{diagrams/print_statement}

\label{fu:print:statement}
\index{PRINT@\PRINT!function statement}
\index{SET\_MSG@\SETMSG!function statement}
\index{LOG@\LOG!function statement}
\index{DEBUG@\DEBUG!LOG statement!log level}
\index{INFO@\INFO!LOG statement!log level}
\index{WARN@\WARN!LOG statement!log level}
\index{ERROR@\ERROR!LOG statement!log level}
\index{FATAL@\FATAL!LOG statement!log level}
\index{REPORT@\REPORT!function statement}
\index{PREVIEW@\PREVIEW!function statement}
\index{STD\_WINDOW@\STDWINDOW!PRINT statement}
\index{LOG\_WINDOW@\LOGWINDOW!PRINT statement}
\index{PRINT\_LOG@\PRINTLOG!function statement}
\begin{tabularx}{\textwidth}{l|X}
print statement  & description \\
\hline
\PRINT           & Prints its arguments to \LOGWINDOW. \\
\SETMSG          & Text to be printed after execution. \\
\LOG             & Log a message with the given level. \\
\REPORT          & Opens the printer dialog to print, save or preview the specified reportstream. \\
\PREVIEW         & Creates and opens the report (i.E. using a pdf reader). \\
\end{tabularx}


\input{diagrams/error_statement}

\label{fu:error:statement}
\index{SET\_ERROR@\SETERROR!function statement}
\index{ABORT@\ABORT!function statement}
\index{RESET\_ERROR@\RESETERROR!function statement}
\begin{tabularx}{\textwidth}{l|X}
error statement & description \\
\hline
\SETERROR       & prints its arguments to a dialog window. \\
\ABORT          & works like \SETERROR{} and aborts the running task immediately. \\
\RESETERROR     & resets error setting. \\
\end{tabularx}


\input{diagrams/exit_statement}

\label{fu:exit:statement}
\index{EXIT@\EXIT!function statement}
\begin{tabularx}{\textwidth}{l|X}
exit statement  & description \\
\hline
\EXIT           & quit INTENS execution. \\
\end{tabularx}


\input{diagrams/system_statement}

\label{fu:system:statement}
\index{BEEP@\BEEP!function statement}
\begin{tabularx}{\textwidth}{l|X}
system statement  & description \\
\hline
\BEEP           & play a system beep. \\
\end{tabularx}


\input{diagrams/messagebox_statement}

\index{MESSAGEBOX@\MESSAGEBOX!function statement}
\begin{tabularx}{\textwidth}{l|X}
messagebox statement  & description \\
\hline
\MESSAGEBOX           & Displays a messagebox with an OK-Button. \newline
                        The first job\_expression provides the message to be displayed. \newline
                        You can optionally provide the title as the second job\_expression.
                        The default title is ``Inform''. \\
\end{tabularx}


\input{diagrams/run_statement}
\input{diagrams/job_run_action}
\label{fu:run:statement}

\index{RUN@\RUN!function statement}
\begin{tabularx}{\textwidth}{l|X}
run statement  & description\\
\hline
\RUN           & Starts the process group, function, task referenced by the identifier\\
\verb+temp_data_reference+ & The value of a string variable is used as the identifier.
                 This can be used to run something different in different situations. \\
\end{tabularx}

\input{diagrams/return_statement}

\label{fu:return:statement}
\index{RETURN@\RETURN!function statement}
\begin{tabularx}{\textwidth}{l|X}
return statement  & description \\
\hline
\RETURN           & exits the function back to calling instance. \\
\end{tabularx}


\input{diagrams/send_statement}
\input{diagrams/send_action}

\label{fu:send:statement}
\index{SEND@\SEND!send statement}
\index{HOST@\HOST!send statement}
\index{PORT@\PORT!send statement}
\index{HEADER@\HEADER!send statement}
\index{STREAM@\STREAM!send statement}
\begin{tabularx}{\textwidth}{l|X}
send statement & description \\
\hline
\HOST          & host name or ip address. \\
\PORT          & host port number. \\
\HEADER        & header string. \\
\STREAM        & data. \\
\end{tabularx}
\vspace{0.5cm}

Send hhHEADERddddddddDATA using tcp to host:port. \\

\begin{tabularx}{\textwidth}{l|X}
hh       & number of characters of HEADER \\
dddddddd & number of characters of DATA \\
\end{tabularx}

%%%%%%%%%%%%%%%%%%%%%%%%%%%%%%%%%%%%%%%%%%%%%%%%%%%%%%%%%%%%%%%%%%%%%%%%%%%%%
%%%                            TIMER                                      %%%
%%%%%%%%%%%%%%%%%%%%%%%%%%%%%%%%%%%%%%%%%%%%%%%%%%%%%%%%%%%%%%%%%%%%%%%%%%%%%
\newpage
Timer statements are used to start and stop a timer.
See \nameref{sec:optimer} on page \pageref{sec:optimer}
and \nameref{sec:opexamples:messagequeue:publishsubscribe}
on page \pageref{sec:opexamples:messagequeue:publishsubscribe}. \\[2ex]

\input{diagrams/timer_statement}
\input{diagrams/job_timer_options}
\index{START@\START!timer function statement}
\index{STOP@\STOP!timer function statement}
\index{TIMER@\TIMER!timer function option}
\index{PERIOD@\PERIOD!timer function option}
\index{DELAY@\DELAY!timer function option}

\vspace{0.5cm}
\begin{tabularx}{\textwidth}{l|X}
timer statement  & description \\
\hline
\START           & Start the timer. \\
\STOP            & Stop the timer. \\
\hline
{\bfseries ID\_TIMER}  & The timer to start or stop. \\
\PERIOD{} = n    & The timer calls its function every n seconds. \newline
                   Default is 60 seconds. \newline
                   \PERIOD{} = 0: call the function once, without \DELAY. \\
\DELAY{} = n     & The timer starts to call its function after n seconds. \newline
                   Default is 0: start to call the function after \PERIOD{} seconds. \\
\end{tabularx}

%%%%%%%%%%%%%%%%%%%%%%%%%%%%%%%%%%%%%%%%%%%%%%%%%%%%%%%%%%%%%%%%%%%%%%%%%%%%%
%%%                            MESSAGEQUEUE                               %%%
%%%%%%%%%%%%%%%%%%%%%%%%%%%%%%%%%%%%%%%%%%%%%%%%%%%%%%%%%%%%%%%%%%%%%%%%%%%%%
\newpage
\input{diagrams/message_queue_statement}
\input{diagrams/message_queue_action}
\input{diagrams/job_message_queue_option}
\input{diagrams/job_plugin_option}

\index{REQUEST@\REQUEST!message queue function statement}
\index{PUBLISH@\PUBLISH!message queue function statement}
\index{SUBSCRIBE@\SUBSCRIBE!message queue function statement}
\index{SET\_MQ\_HOST@\SETMQHOST!message queue function statement}
\index{MESSAGE\_QUEUE@\MESSAGEQUEUE!message queue function option}
\index{PLUGIN@\PLUGIN!message queue function option}
\index{REQUEST@\REQUEST!message queue function option}
\index{RESPONSE@\RESPONSE!message queue function option}
\index{DEPENDENCIES@\DEPENDENCIES!message queue function option}
\index{NO\_DEPENDENCIES@\NODEPENDENCIES!message queue function option}
\index{HEADER@\HEADER!message queue function option}
\index{TIMEOUT@\TIMEOUT!message queue function option}
\index{AUTOCLEAR\_DEPENDENCIES@\AUTOCLEARDEPENDENCIES!message queue function option}
\index{FUNC@\FUNC!message queue function option}

Message Queue statements are used to send data (a message) via ZeroMQ,
to subscribe to a ZeroMQ publisher of a plugin or
to set the \HOST of a \SUBSCRIBE{} \MESSAGEQUEUE{}
(see section \nameref{sec:opmessagequeue} on page \pageref{sec:opmessagequeue}).
A message can be sent to another program using a \MESSAGEQUEUE{} defined in the \OPERATOR.
And it can be sent to a \PLUGIN{} defined in the \UIMANAGER{}
(see section \nameref{sec:uiplugin} on page \pageref{sec:uiplugin}).

The \SUBSCRIBE{} statement is used to subscribe to a \PLUGIN s publisher.

\MESSAGEQUEUE{} or \PLUGIN{} as well as \HEADER{} are mandatory. The other options
can be given as needed.

See \nameref{sec:opexamples:messagequeue:publishsubscribe}
on page \pageref{sec:opexamples:messagequeue:publishsubscribe}. \\[2ex]

\begin{tabularx}{\textwidth}{l|X}
message queue statement & description \\
\hline
\REQUEST                & Send a request message. \\
\PUBLISH                & Publish a message. \\
\SUBSCRIBE              & Subscribe to a publisher of a plugin. \\
\SETMQHOST              & Set the \HOST{} (second argument) of a \SUBSCRIBE{} \MESSAGEQUEUE{} (first argument).
                          This can be done once for every \SUBSCRIBE{} \MESSAGEQUEUE{}, but only when its
                          \HOST{} is set to "" (empty string) in the definition (in the \OPERATOR). \newline
                          Setting the \HOST{} makes the \MESSAGEQUEUE{} subscribe to the publisher. \\
\hline
\MESSAGEQUEUE           & Send the message using this message queue. \\
\PLUGIN                 & Send the message or subscribe to this plugin. \\
\REQUEST                & \REQUEST: Stream(s) define the message(s) to send. \\
\RESPONSE               & \REQUEST: The received message(s) is/are written to this/these stream(s). \newline
                          \SUBSCRIBE: The received message(s) is/are written to this/these stream(s). \newline
                          \PUBLISH: Stream(s) define the message(s) to publish. \\
\DEPENDENCIES           & \REQUEST: This stream is used with dependencies. \newline
                          Default, unless commandline option defaultMessageQueueDependencies false is used. \newline
                          (see paragraph \nameref{par:stdependency} on page \pageref{par:stdependency}) \\
\NODEPENDENCIES         & \REQUEST: This stream is not used with dependencies. \newline
                          (see paragraph \nameref{par:stdependency} on page \pageref{par:stdependency}) \\
\HEADER                 & Message header: first part of the message.
                          The receiver uses it to identify the message type. \\
\TIMEOUT                & \REQUEST: Number of seconds to wait for a connection to the receiver and its \RESPONSE. \newline
                          Without this option, the \TIMEOUT{} of the \MESSAGEQUEUE{} is used (default = 10 seconds). \newline
                          Set \TIMEOUT{} to 0 to have no \TIMEOUT. \\
\AUTOCLEARDEPENDENCIES  & Dependencies are cleared without user confirmation.
                          (see paragraph \nameref{par:stdependency} on page \pageref{par:stdependency}) \\
\FUNC                   & \SUBSCRIBE: Function to call after receiving the message. \\
\end{tabularx}


%%%%%%%%%%%%%%%%%%%%%%%%%%%%%%%%%%%%%%%%%%%%%%%%%%%%%%%%%%%%%%%%%%%%%%%%%%%%%
%%%                            DB STATEMENT                               %%%
%%%%%%%%%%%%%%%%%%%%%%%%%%%%%%%%%%%%%%%%%%%%%%%%%%%%%%%%%%%%%%%%%%%%%%%%%%%%%
\newpage
\input{diagrams/database_statement}
\label{fu:database:statement}
\index{BEGINTRANSACTION@\BEGINTRANSACTION!database function statement}
\index{COMMITTRANSACTION@\COMMITTRANSACTION!database function statement}
\index{ABORTTRANSACTION@\ABORTTRANSACTION!database function statement}
\index{SET\_DB\_TIMESTAMP@\SETDBTIMESTAMP!database function statement}

\begin{tabularx}{\textwidth}{l|X}
database statement      & description \\
\hline
\BEGINTRANSACTION       & TODO \\
\COMMITTRANSACTION      & TODO \\
\ABORTTRANSACTION       & TODO \\
\SETDBTIMESTAMP         & Sets the database timestamp of \verb+job_data_reference+ to the data timestamp. \newline
                          Afterwards, \verb+MODIFIED ( job_data_reference )+ returns false. \\
\end{tabularx}


\input{diagrams/if_statement}
\label{fu:if:statement}
\index{IF@\IF!function statement}
\index{ELSE@\ELSE!function statement}
\begin{tabularx}{\textwidth}{l|X}
if statement        & description \\
\hline
\IF                 & executes the \verb+statement+, if the \verb+expression+ is not 0 (true) \\
\ELSE               & executes the \verb+statement+, if the if-expression was 0 (false) \\
\end{tabularx}

\input{diagrams/while_statement}
\label{fu:while:statement}
\index{WHILE@\WHILE!function statement}
\begin{tabularx}{\textwidth}{l|X}
while statement  & description \\
\hline
\WHILE           & executes the \verb+statement+ while the \verb+expression+ is not 0 (true) \\
\end{tabularx}
\vspace{0.5cm}

Example:

\begin{boxedminipage}[t]{\linewidth}
\begin{alltt}
\FUNCTIONS
  \FUNC function_identifier \{
    \IF( data_item_identifier_1 == 99 )\{
       data_item_identifier_2 = "Error message" ;
    \}
    \WHILE( data_item_identifier_1 <= 99 )\{
       data_item_identifier_1++ ;
    \}
  \}
;
\END \FUNCTIONS;
\end{alltt}
\end{boxedminipage}


\input{diagrams/copy_paste_statement}
\label{fu:copy:paste:statement}
\index{COPY@\COPY!function statement}
\index{PASTE@\PASTE!function statement}
\begin{tabularx}{\textwidth}{l|X}
copy paste statement & description \\
\hline
\COPY                & Copies a streamable object (
                       \hyperref[sec:uilist]{LIST},
                       \hyperref[sec:uitable]{TABLE},
                       \hyperref[sec:streamer]{STREAM}
                       ) into the clipboard. \newline
                       Or copies a screenshot of a
                       \hyperref[dia:uiformelementidentifier]{\nameref{dia:uiformelementidentifier}}
                       or a \hyperref[sec:uiform]{FORM}
                       into the clipboard. \\
\PASTE               & Pastes the clipboard to a streamable object (
                       \hyperref[sec:uilist]{LIST},
                       \hyperref[sec:uitable]{TABLE},
                       \hyperref[sec:streamer]{STREAM}
                       ). \\
\end{tabularx}

\newpage
\input{diagrams/set_func_statement}

These statements are used to change function attributes.

When a function is called, \INTENS{} sets some attributes that can be used
by the function to know why it was called. Sometimes, such a function should
be called from another function and these attributes have to be changed so
that the function does what is needed.

Use these statements carefully as they may change the behaviour of the rest
of the function, including possible parent functions.
\\

\index{SET\_THIS@\SETTHIS!function statement}
\index{THIS@\THIS!function statement}
\index{SET\_INDEX@\SETINDEX!function statement}
\index{INDEX@\INDEX!function statement}
\index{SET\_REASON@\SETREASON!function statement}
\index{REASON@\REASON!function statement}
\begin{tabularx}{\textwidth}{l|X}
statement & description \\
\hline
\SETTHIS     & \THIS{} now references \verb+job_data_reference+. \newline
               See \nameref{dia:jobdatareference} on page \pageref{dia:jobdatareference}. \\
\SETINDEX    & Set \INDEX{} to the value of \verb+job_data_reference+. \newline
               See \nameref{dia:dataexpression} on page \pageref{dia:dataexpression}. \\
\SETREASON   & Set the \REASON{}. To set \REASONINPUT, do \newline
               \verb+  s = ``INPUT'';+ \newline
               \verb+  SET_REASON(s);+. \newline
               See \nameref{fuexpressionsreason} on page \pageref{fuexpressionsreason}. \\
\end{tabularx}

\newpage
\paragraph*{Rest Service}
\label{par:restService}
Intens knows the following actions to communicate with an INTENS db service (REST).

\input{diagrams/restService_statement}
\input{diagrams/restService_action}
\input{diagrams/restService_responseStream}
\input{diagrams/restService_get_option}
\input{diagrams/restService_put_option}
\input{diagrams/restService_post_option}
\input{diagrams/restService_path}
\label{fu:restService:statement}
\index{GET@\GET!restService statement}
\index{DELETE@\DELETE!restService statement}
\index{PUT@\PUT!restService statement}
\index{POST@\POST!restService statement}
\index{PATH@\PATH!restService statement}
\index{DATA@\DATA!restService statement}
\index{SET\_DB\_TIMESTAMP@\SETDBTIMESTAMP!restService statement}
\index{REST\_LOGON@\RESTLOGON!restService statement}
\index{REST\_JWT\_LOGON@\RESTJWTLOGON!restService statement}
\index{REST\_LOGOFF@\RESTLOGOFF!restService statement}
\begin{tabularx}{\textwidth}{l|X}
restService statement & description \\
\hline
\GET           & Send a GET request. \\
\DELETE        & Send a DELETE request. \\
\PUT           & Send a PUT request. \\
\POST          & Send a POST request. \\
\PATH          & Second part of the \Slanted{URL} the request is sent to.
                 The first part is defined in the login dialog. It defaults to
                 environment variable \Slanted{REST\_SERVICE\_BASE} or to the first
                 value of the environment variable \Slanted{REST\_SERVICE\_BASE\_LIST}
                 (list of urls separated by ';'). \newline
                 {\bfseries ID\_STREAM}: use an \URL{} stream for percent-encoding \\
\DATA          & Defines the stream that contains the data for \PUT{} and \POST{} requests. \\
\FILTER        & Defines the stream that contains filter parameters for \GET{}, \DELETE{} and \POST{} requests.
                 Valid entries are appended to the URL in the form
                 name=value or name=(value[0],value[1], ...). \\
\verb+ui xfer+ & Data item (see section \nameref{dia:uixfer} on page \pageref{dia:uixfer}). \\
\SETDBTIMESTAMP & Update the database timestamp of \DATA{}. \\
\RESTLOGON     & Set rest credentials. The three arguments are base url, username and password. \\
\RESTJWTLOGON  & Set rest credentials. The two arguments are base url and jwt (json web token). \\
\RESTLOGOFF    & Clear rest credentials. Login will appear with next \GET{}, \PUT{} or \DELETE{}. \\
\end{tabularx}

If an input stream is defined ({\bfseries ID\_STREAM =}), the results are written to it and the database
timestamp of its data items is set. \\
Username and password are automatically asked for when a \GET{}, \PUT{} or \DELETE{}
request is called and the credentials are missing. They are kept in memory for later usage
until the application is closed or \RESTLOGOFF{} is called in a function. They are also written
to the environment variable REST\_SERVICE\_AUTHHEADER, i.E. to be used in \BATCH{} programms called
from the application: base64 encoded ``<username>:<password>''.
\vspace{0.5cm}

Example:

\begin{boxedminipage}[t]{\linewidth}
\begin{alltt}
\FUNCTIONS
  \FUNC
    getMotor \{
      [motor] = \GET( \PATH="/components/123" );
    \}
  , saveMotor \{
      [motor] = \PUT( \PATH="/components"
                    , \DATA="motor_stream"
                    , \SETDBTIMESTAMP );
    \}
  ;
\END \FUNCTIONS;
\end{alltt}
\end{boxedminipage}

\subparagraph*{Rest Service: Version control}
\label{par:restServiceVersionControl}
\index{REST\_SERVICE.APP\_VERSION\_MAJOR@\RESTSERVICEAPPVERSIONMAJOR!Rest Service: Version control}
\index{REST\_SERVICE.APP\_VERSION\_MINOR@\RESTSERVICEAPPVERSIONMINOR!Rest Service: Version control}
\index{REST\_SERVICE.APP\_VERSION\_PATCH@\RESTSERVICEAPPVERSIONPATCH!Rest Service: Version control}
\index{REST\_SERVICE.DB\_VERSION\_MAJOR@\RESTSERVICEDBVERSIONMAJOR!Rest Service: Version control}
\index{REST\_SERVICE.DB\_VERSION\_MINOR@\RESTSERVICEDBVERSIONMINOR!Rest Service: Version control}
\index{REST\_SERVICE.DB\_VERSION\_PATCH@\RESTSERVICEDBVERSIONPATCH!Rest Service: Version control}
\index{REST\_SERVICE.DB\_VERSION\_IGNORE@\RESTSERVICEDBVERSIONIGNORE!Rest Service: Version control}

\def\DbVer{\keyword{database version}}
\def\MinDbVer{\keyword{minimal database version}}
\def\AppVer{\keyword{application version}}

Sometimes, an INTENS application requires a \MinDbVer{} or the database requires a minimal
INTENS \AppVer{}. Login to a database is not allowed when these requirements are not met.

Here is how that version control works and how you can use it.

The \DbVer{} is defined in a (single) component of type AppVersionCtrl (stored in the database)
with the attribues
\begin{itemize}
\item app\_major (mandatory)
\item app\_minor (optional, defaults to 0)
\item app\_patch (optional, defaults to 0)
\end{itemize}

The \AppVer{} is defined using the \DATAPOOL{} variables
\begin{itemize}
\item REST\_SERVICE.APP\_VERSION\_MAJOR (mandatory)
\item REST\_SERVICE.APP\_VERSION\_MINOR (optional, defaults to 0)
\item REST\_SERVICE.APP\_VERSION\_PATCH (optional, defaults to 0)
\end{itemize}

The \MinDbVer{} is defined using the \DATAPOOL{} variables
\begin{itemize}
\item REST\_SERVICE.DB\_VERSION\_MAJOR (mandatory)
\item REST\_SERVICE.DB\_VERSION\_MINOR (optional, defaults to 0)
\item REST\_SERVICE.DB\_VERSION\_PATCH (optional, defaults to 0)
\end{itemize}

When none of these are set, there is no version control and login is allowed.

Another possibility to disable version control is to set the \DATAPOOL{} variable

\RESTSERVICEDBVERSIONIGNORE{} to a value > 0.

Otherwise, version control is active and
login is not allowed in any of the following situations:
\begin{itemize}
\item \DbVer{} or \AppVer{} are not set
\item \AppVer{} < \DbVer{} \\
      \DbVer{} defines the minimal \AppVer{} required
\item \MinDbVer{} is set and \DbVer{} < \MinDbVer{} \\
      \MinDbVer{} defines the minimal \DbVer{} required
\end{itemize}

So, to use the version control, you have to
\begin{itemize}
\item require a minimal INTENS \AppVer{} \\
      Define \DbVer{} in the database and  \AppVer{} in the \DATAPOOL{}.
      \AppVer{} must not be lower.
\end{itemize}
and you can additionally
\begin{itemize}
\item require a \MinDbVer{} \\
      Define \MinDbVer{} in the \DATAPOOL{}. \DbVer{} must not be lower.
\end{itemize}

%%%%%%%%%%%%%%%%%%%%%%%%%%%%%%%%%%%%%%%%%%%%%%%%%%%%%%%%%%%%%%%%%%%%%%%%%%%%%
%%%                             Expressions                               %%%
%%%%%%%%%%%%%%%%%%%%%%%%%%%%%%%%%%%%%%%%%%%%%%%%%%%%%%%%%%%%%%%%%%%%%%%%%%%%%
\newpage
\subsubsection{Expressions}
\label{fuexpressions}
\input{diagrams/job_expression}

\begin{tabularx}{\textwidth}{l|X}
expression                   & description\\
\hline
{\verb+Data expression+}    & is called whithout any parameters. (Page \pageref{fuexpressionsdata})\\
{\verb+Cycle expression+}   & is called whithout any parameters. (Page \pageref{fuexpressionscycle})\\
{\verb+Function expression+}& must be acompanied by parameters put in parethesis. (Page \pageref{fuexpressionsfunctions})\\
{\verb+Math expression+}    & calculates a value using parameter(s). (Page \pageref{fuexpressionsmath})\\
{\verb+Filename expression+}& gives filename, basename or dirname of a parameter. (Page \pageref{fuexpressionsfilename})\\
{\verb+Reason expression+}  & is used to test where the function itself is called from. (Page \pageref{fuexpressionsreason})\\
\end{tabularx}

%---------------------------------------------------------------------------%
\input{diagrams/data_expression}
\label{fuexpressionsdata}

\index{EOLN@\EOLN!function expression}
\index{INVALID@\INVALID!function expression}
\index{OLDVALUE@\OLDVALUE!function expression}
\index{INDEX@\INDEX!function expression}
\index{COL@\COL!function expression}
\index{ROW@\ROW!function expression}
\index{DIAGRAMXPOS@\DIAGRAMXPOS!function expression}
\index{DIAGRAMYPOS@\DIAGRAMYPOS!function expression}
\index{ERROR@\ERROR!function expression}
\index{  @Signs / Characters!\texttt{"!} (exclamation mark)!not}
\index{  @Signs / Characters!- (hyphen)!negative value}

\begin{tabularx}{\textwidth}{l|X}
data expression & description\\
\hline
{\verb+real+}   & floating point number (constant) \\
{\verb+integer+}& integer number (constant) \\
{\verb+string+} & (see section \nameref{sec:string} on page \pageref{sec:string}) \\
\EOLN            & represents an end of line character \\
\INVALID         & returns an invalid value \\
\OLDVALUE        & returns the old (previous) value of the changed data item \\
\INDEX           & represents the index of the field that called the function.
                   Changing data[2].value[3], \INDEX{} is set to 3. \newline
                   See \nameref{dia:datastatement} on page \pageref{dia:datastatement}
                   and \nameref{dia:functionexpression} on page \pageref{dia:functionexpression}
                   for other \INDEX{} expressions. \\
\COL             & returns the column index of the field that called the function \\
\ROW             & returns the row index of the field that called the function \\
\DIAGRAMXPOS     & returns the x coordinate of the diagram element that called the function \\
\DIAGRAMYPOS     & returns the y coordinate of the diagram element that called the function \\
\ERROR           & returns True if a error has occured \\
{\bfseries ID\_INDEX}& references index-object previously declared in ui\_manager (page \pageref{sec:uiindex}) \\
{\bfseries -}    & inverts the sign of job\_expression (\verb|j = -i;|) \\
{\bfseries !}    & represents a logical {\bfseries NOT} \\
\end{tabularx}

\label{fuexpressionsdataoperator}
\index{  @Signs / Characters!\texttt{"+} (plus)!arithmetic operator}
\index{  @Signs / Characters!- (hyphen)!arithmetic operator}
\index{  @Signs / Characters!* (asterisk)!arithmetic operator}
\index{  @Signs / Characters!/ (slash)!arithmetic operator}
\index{  @Signs / Characters!\% (percent)!modulo operator}
\index{  @Signs / Characters!\texttt{"\^}!exponentiation}
\index{  @Signs / Characters!\texttt{">} (greater than)!logic operator}
\index{  @Signs / Characters!\texttt{">}=!logic operator}
\index{  @Signs / Characters!\texttt{"<} (less than)!logic operator}
\index{  @Signs / Characters!\texttt{"<}=!logic operator}
\index{  @Signs / Characters!==!logic operator}
\index{  @Signs / Characters!\texttt{"!}=!logic operator}
\index{  @Signs / Characters!\&\&!logic operator}
\index{  @Signs / Characters!\texttt{"|}\texttt{"|}!logic operator}

\input{diagrams/job_string_expression}
\index{LABEL@\LABEL!function expression}
\index{UNIT@\UNIT!function expression}
\begin{tabularx}{\textwidth}{l|X}
string expression & description\\
\hline
\LABEL         & returns previously defined label of data-item. (section \nameref{dia:dataitemoptions} page \pageref{dia:dataitemoptions})\\
\UNIT          & returns previously defined units of data-item. (section \nameref{dia:dataitemoptions} page \pageref{dia:dataitemoptions})\\
\end{tabularx}

\input{diagrams/cycle_expression}
\label{fuexpressionscycle}
\index{GETCYCLE@\GETCYCLE!function expression}
\index{MAXCYCLE@\MAXCYCLE!function expression}
\index{CYCLENAME@\CYCLENAME!function expression}

\begin{tabularx}{\textwidth}{l|X}
cycle expression     & description\\
\hline
\GETCYCLE            & returns current cycle no. \\
\MAXCYCLE            & returns ammount of cycles \\
\CYCLENAME           & returns name of current cycle \\
\CYCLENAME {\bfseries [i]}  & returns name of cycle {\verb+i+}. \\
\end{tabularx}


%---------------------------------------------------------------------------%
\input{diagrams/function_expression}
\label{fuexpressionsfunctions}
\index{VALID@\VALID!function expression}
\index{INDEX@\INDEX!function expression}
\index{CONFIRM@\CONFIRM!function expression}
\index{CONFIRM\_CANCEL@\CONFIRMCANCEL!function expression}
\index{GETTEXT@\GETTEXT!function expression}
\index{TIMESTAMP@\TIMESTAMP!function expression}
\index{MODIFIED@\MODIFIED!function expression}
\index{CLASSNAME@\CLASSNAME!function expression}
\index{NODE@\NODE!function expression}
\index{RUN@\RUN!function expression}
\index{LOAD@\LOAD!function expression}
\index{SORTORDER@\SORTORDER!function expression}
\index{CHANGED@\CHANGED!function expression}
\index{VISIBLE@\VISIBLE!function expression}
\index{EDITABLE@\EDITABLE!function expression}
\index{COMPARE@\COMPARE!function expression}
\index{SET\_RESOURCE@\SETRESOURCE!function expression}
\index{COMPOSE@\COMPOSE!function expression}
\index{ASSIGN\_CONSISTENCY@\ASSIGNCONSISTENCY!function expression}

\begin{tabularx}{\textwidth}{l|X}
function expression & description\\
\hline
\VALID         & returns true if the expression is valid \\
\INDEX         & returns the index of the field that called the function.
                 The first argument defines the dimension (counted from right to left -
                 the last dimension being dimension 0), the optional second argument
                 defines the structLevel (defaults to the last (highest)). \newline
                 Example: changing data[1,2].value[3,4]: \newline
                   \INDEX(1) is 3 (same as \INDEX(1,1) \newline
                   \INDEX(1,0) is 1 \newline
                   \INDEX(0,0) is 2 \newline
                   \INDEX(0,1) is 4 (same as \INDEX(0) or \INDEX) \newline
                   See \nameref{dia:datastatement} on page \pageref{dia:datastatement}
                   and \nameref{dia:dataexpression} on page \pageref{dia:dataexpression}
                   for other \INDEX{} expressions. \\
\CONFIRM       & displays a messagebox with a yes and no button. It evaluates to 1(yes) or 0(no). \\
\CONFIRMCANCEL & displays a messagebox with a yes, no and cancel button. It evaluates to 1(yes), 0(no) or \INVALID(cancel). \\
\GETTEXT       & displays a dialog with the given text and an input field where the user enters a text. \\
\TIMESTAMP     & Retruns the data timestamp of {\bfseries job\_data\_reference}. \\
\MODIFIED      & Is {\bfseries job\_data\_reference} modified (compares db and data timestamps)? \\
\CLASSNAME     & returns previously defined classname of data-item. (section \nameref{dataitemattributes} page \pageref{dataitemattributes}) \\
\NODE          & returns node name of argument. Useful with \INPUT, \THIS{} or \BASE.
                 (see \nameref{fuexample3} on page \pageref{fuexample3}) \\
\RUN           & run the argument. \\
\LOAD          & parse the content of the \STRING{} data reference \\
\SORTORDER     & returns the current \SORTORDER{} of the \LIST. \\
               & The \SORTORDER{} is the number of the column used to sort the list,
                 starting with 1. Hidden columns are counted.
                 Descending sort returns a negative number. \\
               & Defaults to 0. \\
\CHANGED       & Has any data item that shows in the \FORM{}
                 changed after timestamp (stored in first
                {\bfseries job\_data\_reference})? \newline
                With a second {\bfseries job\_data\_reference}, only data items
                that belong to that struct object are checked. \newline
                \TRANSIENT{} data items are ignored. \newline
                When \FORM{} is the \MAIN{} \FORM{}, data items in all \FORM{}s are checked. \\
\VISIBLE       & returns true if the \FORM{} or the \FOLDER{} group is visible (MAPPED). \\
\EDITABLE      & returns true if the variable or \FIELDGROUP{} is editable. \\
\COMPARE       & see \COMPARE{} function statement on page \pageref{dia:datastatement} \\
\SETRESOURCE   & Set value of a resource property (with name {\bfseries string}). \newline
                 The value is written to the user resource file when the \INTENS{} application
                 is quit or when \WRITESETTINGS{} is called.
                 It can then be used when the \INTENS{} application is started next time(s). \\
\COMPOSE       & Compose string: write values into string. Needed for translatable strings. \newline
                 The (format-)string (first argument) contains \%1, \%2, ... and the values of the
                 corresponding data references are put in these places. \newline
                 Up to 15 values are implemented. \\
\ASSIGNCONSISTENCY & Assign the (value of the) second argument to the first. \newline
                    \INVALID: clear the first arguments value. \newline
                    Then, clear all dependent results
                    (see paragraph \nameref{par:stdependency} on page \pageref{par:stdependency}).
\end{tabularx}


\index{BUTTON\_YES@\BUTTONYES!confirm option}
\index{BUTTON\_NO@\BUTTONNO!confirm option}
\index{BUTTON\_CANCEL@\BUTTONCANCEL!confirm option}
\input{diagrams/job_confirm_option}
\begin{tabularx}{\textwidth}{l|X}
confirm option & description\\
\hline
\BUTTONYES    & label shown on the yes button \\
\BUTTONNO     & label shown on the no button \\
\BUTTONCANCEL & label shown on the cancel button \\
\end{tabularx}


\input{diagrams/job_visible_action}

\begin{tabularx}{\textwidth}{l|X}
                    & description \\
\hline
\verb+temp_data_reference+ & The value of a string variable is used as the identifier.
                 This can be used to check the visibility of something different in different situations. \\
\end{tabularx}


\input{diagrams/job_editable_action}

\begin{tabularx}{\textwidth}{l|X}
                    & description \\
\hline
\verb+job_data_reference+ & Check if a variable is editable. \\
{\bfseries ID\_FIELDGROUP} & Check if a \FIELDGROUP{} is editable.
\end{tabularx}

%---------------------------------------------------------------------------%
\input{diagrams/math_expression}
\label{fuexpressionsmath}
\index{ABS@\ABS!function expression}
\index{LENGTH@\LENGTH!function expression}
\index{SIN@\SIN!function expression}
\index{COS@\COS!function expression}
\index{TAN@\TAN!function expression}
\index{ASIN@\ASIN!function expression}
\index{ACOS@\ACOS!function expression}
\index{ATAN@\ATAN!function expression}
\index{ATAN2@\ATANTWO!function expression}
\index{LENGTH@\LENGTH!function expression}
\index{LOG@\LOG!function expression}
\index{LOG10@\LOGTEN!function expression}
\index{SQRT@\SQRT!function expression}
\index{ROUND@\ROUND!function expression}
\index{ROUND5@\ROUNDFIVE!function expression}
\index{ROUND10@\ROUNDTEN!function expression}
\index{REAL@\REAL!function expression}
\index{IMAG@\IMAG!function expression}
\index{ARG@\ARG!function expression}
\index{COMPLEX@\COMPLEX!function expression}

\begin{tabularx}{\textwidth}{l|X}
math expression & description\\
\hline
\ABS           & returns the absolute-value of the expression \\
\LENGTH        & returns the string length of the expression \\
\SIN           & returns the sine-value of the expression (radians) \\
\COS           & returns the cosine-value of the expression (radians) \\
\TAN           & returns the tangent-value of the expression (radians) \\
\ASIN          & returns the arcsine-value (radians) of the expression \\
\ACOS          & returns the arccosine-value (radians) of the expression \\
\ATAN          & returns the arctangent-value (radians) of the expression \\
\ATANTWO       & returns the arctangent-value (radians) of (y, x) \\
\LENGTH        & returns the string length of the expression \\
\LOG           & returns the natural logarithm of the expression \\
\LOGTEN        & returns the base 10 logarithm of the expression \\
\SQRT          & returns the nonnegative square root of the expression \\
\ROUND         & rounds the value (first expression) to n (second expression) digits after the decimal point \\
\ROUNDFIVE     & rounds the value to the next 20-th, i.E. to 5 cents \\
\ROUNDTEN      & rounds the value to the next 10-th, i.E. to 10 cents \\
\REAL          & returns the real-value of the expression, or invalid if it is not real \\
\IMAG          & returns the imaginary quantity of a complex expression \\
\ARG           & returns the angle-value of the expression \\
\COMPLEX       & returns the complex-notation of 'expression, expression' \\
\end{tabularx}

%---------------------------------------------------------------------------%
\input{diagrams/file_expression}
\label{fuexpressionsfilename}
\index{OPEN@\OPEN!function expression}
\index{SAVE@\SAVE!function expression}
\index{FILENAME@\FILENAME!function expression}
\index{BASENAME@\BASENAME!function expression}
\index{DIRNAME@\DIRNAME!function expression}
\index{FILTER@\FILTER!filename option}
\index{DIRNAME@\DIRNAME!filename option}
\index{OPEN@\OPEN!filename option}
\index{SAVE@\SAVE!filename option}

\begin{tabularx}{\textwidth}{l|X}
math expression & description\\
\hline
\OPEN{} ( ... )   & read data from a file (see \nameref{dia:filestatement} on page \pageref{dia:filestatement}) \\
\SAVE{} ( ... )   & save to a file (see \nameref{dia:filestatement} on page \pageref{dia:filestatement}) \\
\FILENAME{} ( ID )  & returns the entire file name (including directory) of previously used file stream. \\
\BASENAME{} ( ID )  & returns the file name of previously used file stream. \\
\DIRNAME{} ( ID )   & returns the directory path of previously used file stream. \\
\FILENAME{} \{ ... \} & open a file selection dialog and return the name of the selected file. \\
\DIRNAME{} \{ ... \}  & open a directory selection dialog and return the name of the selected directory. \\
\FILTER           & specify the kind of files that should be shown. \newline
                    Only xml files: ``XML (*.xml)'' \newline
                    Image files (one filter): ``Image files (*.png *.xpm *.jpg)'' \newline
                    xml or json (two filters): ``XML (*.xml);;JSON (*.json)'' \\
\DIRNAME{} = ...  & default directory used in the select dialog \\
\OPEN             & use an open select dialog to get a filename \\
\SAVE             & use a save select dialog to get a filename \\
\verb+ui xfer+   & Data item (see section \nameref{dia:uixfer} on page \pageref{dia:uixfer}). \\
\end{tabularx}
\vspace{0.5cm}

%---------------------------------------------------------------------------%

\paragraph{Reason} \hfill \\
\label{fuexpressionsreason}
\index{REASON@\REASON}

Functions may be used to handle input events on data fields. These functions will be invoked each time after
  a data item has been edited (see \FUNC{} option in \DATAPOOL{} data item declaration
  section \nameref{sec:dpitem} on page \pageref{sec:dpitem}).

Functions may also be used to handle \LIST{} events. These functions will be invoked each time a
  line is selected, activated (mouse: double click or keyboard: return/enter) or unselected (see \FUNC{}
  option in \LIST{} declaration section \nameref{sec:uilist} on page \pageref{dia:uilistoptionlist}).

Functions may also be used to handle \FORM{} events. These functions will be invoked each time a \FORM{}
  is opened, closed or activated (see \FUNC{} option in \FORM{} declaration section
  \nameref{sec:uiform} on page \pageref{dia:uiformoptionlist}).

Functions may also be used to handle \PLOTTWOD{} events. These functions will be invoked each time a
  point or rectangle is selected in the \PLOTTWOD{} (see graph option \FUNC{} in \PLOTTWOD{} declaration
  section \nameref{sec:uiplot2d} on page \pageref{dia:uiplot2dgraphoption}).

Data items may be displayed as a table by adding the \TABLESIZE{} option to fieldgroups
  (\UIMANAGER{} fieldgroup options section \nameref{sec:uifieldgroup}) or by defining a \TABLE{}
  (\UIMANAGER{} table declaration \nameref{sec:uitable}). \\
Tables are shown with column- and row-titles. By pressing the right mouse button on it, a menu is popped up.
Selecting an item will execute the data item assigned function first.

Inside a function, you can test the reason for calling it using \REASON{} expressions:

\begin{boxedminipage}[t]{\linewidth}
\begin{alltt}
  \FUNC a\_func \{
    ...
    \IF(\REASONINPUT) \{
      ...
    \}
    ...
  \};
\end{alltt}
\end{boxedminipage} \\[2ex]

Sometimes, you may want to call a function with a different REASON.
See \nameref{dia:setfuncstatement} on page \pageref{dia:setfuncstatement} to know
how to do so.

You may use one or more of the following REASON expressions:

%---------------------------------------------------------------------------%
\input{diagrams/reason_expression}
\index{REASON_INPUT@\REASONINPUT!function exression}
\index{REASON_INSERT@\REASONINSERT!function exression}
\index{REASON_DUPLICATE@\REASONDUPLICATE!function exression}
\index{REASON_CLEAR@\REASONCLEAR!function exression}
\index{REASON_REMOVE@\REASONREMOVE!function exression}
\index{REASON_PACK@\REASONPACK!function exression}
\index{REASON_SELECT@\REASONSELECT!function exression}
\index{REASON_UNSELECT@\REASONUNSELECT!function exression}
\index{REASON_SELECT_POINT@\REASONSELECTPOINT!function exression}
\index{REASON_SELECT_RECTANGLE@\REASONSELECTRECTANGLE!function exression}
\index{REASON_SORT@\REASONSORT!function exression}
\index{REASON_ACTIVATE@\REASONACTIVATE!function exression}
\index{REASON_OPEN@\REASONOPEN!function exression}
\index{REASON_CLOSE@\REASONCLOSE!function exression}
\index{REASON_DROP@\REASONDROP!function exression}
\index{REASON_MOVE@\REASONMOVE!function exression}
\index{REASON_CONNECTION@\REASONCONNECTION!function exression}
\index{REASON_REMOVE_CONNECTION@\REASONREMOVECONNECTION!function exression}
\index{REASON_REMOVE_ELEMENT@\REASONREMOVEELEMENT!function exression}
\index{REASON_CYCLE_CLEAR@\REASONCYCLECLEAR!function exression}
\index{REASON_CYCLE_DELETE@\REASONCYCLEDELETE!function exression}
\index{REASON_CYCLE_NEW@\REASONCYCLENEW!function exression}
\index{REASON_CYCLE_RENAME@\REASONCYCLERENAME!function exression}
\index{REASON_CYCLE_SWITCH@\REASONCYCLESWITCH!function exression}
\index{REASON_FUNCTION@\REASONFUNCTION!function exression}
\index{REASON_TASK@\REASONTASK!function exression}
\index{REASON_GUI_UPDATE@\REASONGUIUPDATE!function exression}
\begin{tabularx}{\textwidth}{l|X}
reason expression       & description\\
\hline
% data item
\REASONINPUT            & the function was called after the assigned data item has been edited. \\
% table popup menu
\REASONINSERT           & insert has been choosen from table popup menu \\
\REASONDUPLICATE        & duplicate has been choosen from table popup menu \\
\REASONCLEAR            & clear has been choosen from table popup menu \\
\REASONREMOVE           & remove has been choosen from table popup menu \\
\REASONPACK             & pack has been choosen from table popup menu \\
% select
\REASONSELECT           & a list or navigator line was selected by a mouseclick \newline
                          (see section List \nameref{sec:uilist} page \pageref{sec:uilist}) \newline
                          (see section Navigator \nameref{sec:uinavigator} page \pageref{sec:uinavigator}) \\
\REASONUNSELECT         & a list or navigator line was unselected by a mouseclick \newline
                          (see section List \nameref{sec:uilist} page \pageref{sec:uilist})\newline
                          (see section Navigator \nameref{sec:uinavigator} page \pageref{sec:uinavigator})\\
\REASONSELECTPOINT      & a plot2d point is selected by a mouseclick \newline
                          (see paragraph \nameref{par:uiplot2duimode} in section \nameref{sec:uiplot2d} on page \pageref{par:uiplot2duimode})\\
\REASONSELECTRECTANGLE  & a plot2d rectangle is selected by a mouseclick \newline
                          (see paragraph \nameref{par:uiplot2duimode} in section \nameref{sec:uiplot2d} on page \pageref{par:uiplot2duimode})\\
% activate
\REASONACTIVATE         & a list line was activated by pressing the enter key
                          (see section \nameref{sec:uilist} page \pageref{sec:uilist}). \newline
                          a navigator entry was doubleclicked
                          (see section \nameref{sec:uinavigator} page \pageref{sec:uinavigator}) \newline
                          a form is activated (becomes the active window)
                         (see section \nameref{sec:uiform} page \pageref{sec:uiform}). \\
% list
\REASONSORT             & a list was sorted differently
                          (see section \nameref{sec:uilist} page \pageref{sec:uilist}). \\
% form
\REASONOPEN             & a form is opened
                         (see section \nameref{sec:uiform} page \pageref{sec:uiform}). \\
\REASONCLOSE            & a form is closed
                         (see section \nameref{sec:uiform} page \pageref{sec:uiform}). \\
% navigator
\REASONDROP             & a navigator entry was drag'n'dropped \newline
                          (see section \nameref{sec:uinavigator} page \pageref{sec:uinavigator}) \newline
                          (see section \nameref{sec:uinavigatorDiagram} page \pageref{sec:uinavigatorDiagram})\\
\REASONMOVE             & a navigator (diagram) element was moved
                          (see section \nameref{sec:uinavigatorDiagram} page \pageref{sec:uinavigatorDiagram})\\
\REASONCONNECTION       & two navigator (diagram) elements were connected
                          (see section \nameref{sec:uinavigatorDiagram} page \pageref{sec:uinavigatorDiagram})\\
\REASONREMOVECONNECTION & a navigator (diagram) connection was removed
                          (see section \nameref{sec:uinavigatorDiagram} page \pageref{sec:uinavigatorDiagram})\\
\REASONREMOVEELEMENT    & a navigator (diagram) element was removed
                          (see section \nameref{sec:uinavigatorDiagram} page \pageref{sec:uinavigatorDiagram})\\
% cycle
\REASONCYCLECLEAR       & cycle was cleared. \newline
                          Used in \FUNCTION{} \ONCYCLEEVENT{}
                          (see section \nameref{sec:funcidentifiers} on page \pageref{sec:funcidentifiers}) \\
\REASONCYCLEDELETE      & cycle was deleted. \newline
                          Used in \FUNCTION{} \ONCYCLEEVENT{}
                          (see section \nameref{sec:funcidentifiers} on page \pageref{sec:funcidentifiers}) \\
\REASONCYCLENEW         & cycle was new. \newline
                          Used in \FUNCTION{} \ONCYCLEEVENT{}
                          (see section \nameref{sec:funcidentifiers} on page \pageref{sec:funcidentifiers}) \\
\REASONCYCLERENAME      & cycle was renamed. \newline
                          Used in \FUNCTION{} \ONCYCLEEVENT{}
                          (see section \nameref{sec:funcidentifiers} on page \pageref{sec:funcidentifiers}) \\
\REASONCYCLESWITCH      & cycle was switched. \newline
                          Used in \FUNCTION{} \ONCYCLEEVENT{}
                          (see section \nameref{sec:funcidentifiers} on page \pageref{sec:funcidentifiers}) \\
% other
\REASONFUNCTION         & the function was called directly (i.E. from a menu) \\
\REASONTASK             & the function was called from a \TASK{} that was called directly (i.E. using its button) \\
\REASONGUIUPDATE        & the function was called by a gui update \\
\end{tabularx}

%%%%%%%%%%%%%%%%%%%%%%%%%%%%%%%%%%%%%%%%%%%%%%%%%%%%%%%%%%%%%%%%%%%%%%%%%%%%%
%%%                             Examples                                  %%%
%%%%%%%%%%%%%%%%%%%%%%%%%%%%%%%%%%%%%%%%%%%%%%%%%%%%%%%%%%%%%%%%%%%%%%%%%%%%%
\subsubsection{Examples}
\label{fuexamples}
\paragraph*{Example 1}
\label{fuexample1}

\begin{boxedminipage}[t]{\linewidth}
\begin{alltt}
\FUNCTIONS
  \FUNC testminmax \{
    \IF( \VALID(SnMin) && \VALID(SnMax) )\{
      \IF( SnMin > SnMax )\{
        \SETERROR( "Minimum value must be less than maximum!"
                 , \EOLN );
      \}
    \}
  \}
;
\END \FUNCTIONS;
\end{alltt}
\end{boxedminipage}


\newpage
\paragraph*{Example 2}
\label{fuexample2}

\begin{boxedminipage}[t]{\linewidth}
\begin{alltt}

DESCRIPTION "Example of THIS";

DATAPOOL
  SET Set_identifier ("Pass" = 1, "Fail" = 2 );
  STRUCT Structure \{
    REAL \{EDITABLE\}
      Data_item_choice \{ SET = Set_identifier,
                         FUNC = save_test_results \};
    STRING
      Data_item_test_result;
  \};
  Structure {\bfseries TestResults} ;
END DATAPOOL;

UI_MANAGER
  FIELDGROUP fieldgroup_identifier_1{\bfseries  \{TABLESIZE=5\}} (
    "Choose restult of each test-run:"
    {\bfseries TestResults [*] }.Data_item_choice
  );
  FORM form_identifier_1 \{MAIN\} (
    ( fieldgroup_identifier_1 )
  );
END UI_MANAGER;

FUNCTIONS
  FUNC save_test_results \{
    IF ( \THIS.Data_item_choice == 1 ) \{
      \THIS.Data_item_test_result = "Test result was OK";
    \}
    ELSE \{
      \THIS.Data_item_test_result = "Test failed";
    \}
  \}
;
END FUNCTIONS;

END.
\end{alltt}
\end{boxedminipage}

\newpage
\paragraph*{Example 3}
\label{fuexample3}

When \INTENS{} is started with the option \texttt{--}withInputStructFunc
and an input variable does not have a function, the function
of its struct (or the first parent of it that has a function)
is called when the value is changed on the GUI. Inside that
function, the following expressions can be used:

\begin{description}
\item[\BASE] references the structure with the called function.
It is not set if the input variable has a function.

\item[\NODE(\INPUT)] can be used to know the name of the variable that was changed.

\item[\NODE(\THIS)] can be used to know the name of the structure of the variable that was changed.

\item[\NODE(\BASE)] can be used to know the name of the structure with the called function.
\end{description}

The following table shows the called functions, the references of \THIS{} and \BASE{} as well
as the values of different \NODE() expressions of the example, when variable is changed on the GUI:

\begin{tabularx}{\textwidth}{l|l|l|l|l|l|l}
 & & & & \multicolumn{3}{c}{\NODE()} \\
variable & function & \THIS & \BASE & \INPUT & \THIS & \BASE \\
\hline
data1.p1.x & x\_func & data1.p1 & - & x & p1 & - \\
data1.p1.y & p1\_func & data1.p1 & data1.p1 & y & p1 & p1 \\
data1.p2.x & x\_func & data1.p2 & - & x & p2 & - \\
data1.p2.y & data1\_func & data1.p2 & data1 & y & p2 & data1 \\
data2.p1.x & x\_func & data2.p1 & - & x & p1 & - \\
data2.p1.y & p1\_func & data2.p1 & data2.p1 & y & p1 & p1 \\
data2.p2.x & x\_func & data2.p2 & - & x & p2 & - \\
data2.p2.y & - & - & - & - & - & - \\
\end{tabularx}


\begin{boxedminipage}[t]{\linewidth}
\begin{alltt}

DESCRIPTION "Example of Input Struct Func and NODE, BASE";

DATAPOOL
  STRUCT Point \{
    REAL \{EDITABLE, SCALAR\}
      x \{FUNC=x_func\}
    , y
    ;
  \};

  STRUCT Data \{
    Point
      p1 \{FUNC=p1_func\}
    , p2
    ;
  \};

  Data
    data1 \{FUNC=data1_func\}
  , data2
  ;
END DATAPOOL;

UI_MANAGER
  FIELDGROUP main_fg (
    VOID        "x"         "y"
  , "data1.p1"  data1.p1.x  data1.p1.y
  , "data1.p2"  data1.p2.x  data1.p2.y
  , "data2.p1"  data2.p1.x  data2.p1.y
  , "data2.p2"  data2.p2.x  data2.p2.y
  );

  FORM main_form \{MAIN, HIDE_CYCLE\} (main_fg);
END UI_MANAGER;

...
\end{alltt}
\end{boxedminipage}

\begin{boxedminipage}[t]{\linewidth}
\begin{alltt}
...
FUNCTIONS
  FUNC print_func \{
    PRINT("NODE(INPUT): ", NODE(INPUT),
          ", NODE(THIS): ", NODE(THIS),
          ", NODE(BASE): ", NODE(BASE), EOLN, EOLN);
  \};

  FUNC x_func \{
    PRINT("x_func:", EOLN);
    RUN(print_func);
  \};

  FUNC p1_func \{
    PRINT("p1_func:", EOLN);
    RUN(print_func);
  \};

  FUNC data1_func \{
    PRINT("data1_func:", EOLN);
    PRINT("BASE.p1:", BASE.p1, ", BASE.p1.x:", BASE.p1.x, EOLN);
    RUN(print_func);
  \};
END FUNCTIONS;

END.
\end{alltt}
\end{boxedminipage}
