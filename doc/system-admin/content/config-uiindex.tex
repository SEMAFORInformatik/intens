%%%%%%%%%%%%%%%%%%%%%%%%%%%%%%%%%%%%%%%%%%%%%%%%%%%%%%%%%%%%%%%%%%%%%%%%%%%%%
%%%                                Index                                  %%%
%%%%%%%%%%%%%%%%%%%%%%%%%%%%%%%%%%%%%%%%%%%%%%%%%%%%%%%%%%%%%%%%%%%%%%%%%%%%%
\subsubsection{Index-Object}
\label{sec:uiindex}

\begin{figure}[h]\label{fig:indexObject}
  \begin{center}
    \includegraphics{grab_index-object}
  \end{center}
  \caption{Index-Object}
\end{figure}

An index-object is a GUI-Object which may be invisible or placed in a fieldgroup.
It can be used as a variable index.
(see section \nameref{sec:uivariables} on page \pageref{sec:uivariables}). \\
The index-object displays like a scrollbar. Its arrows let you navigate through
the associated data items by mouse click. \\

\input{diagrams/ui_index_list}
\input{diagrams/ui_index_option_list}
\index{INDEX@\INDEX!ui\_manager index}

\index{ORIENTATION@\ORIENTATION!ui\_manager index}
\index{STEP@\STEP!ui\_manager index}
\index{RANGE@\RANGE!ui\_manager index}
\index{VERTICAL@\VERTICAL!ui\_manager index}
\index{HORIZONTAL@\HORIZONTAL!ui\_manager index}
\index{FUNC@\FUNC!ui\_manager index}
\begin{tabularx}{\textwidth}{l|X}
index options  & description \\
\hline
\ORIENTATION   & The index-object-arrows are shown: \\
               & \VERTICAL ($\bigtriangleup$ $\bigtriangledown$) \\
               & \HORIZONTAL ($\triangleleft$ $\triangleright$) \\
\STEP          & defines the step for each arrow-pressing (back and forward) \\
\FUNC          & calls a function defined in section \nameref{sec:functions} page \pageref{sec:functions}.\\
\RANGE         & determines the index range that can be used for the associated
                 data items. The default starting index is 0.\\

\end{tabularx}

%%%%%%%%%%%%%%%%%%%%%%%%%%%%%%%%%%%%%%%%%%%%%%%%%%%%%%%%%%%%%%%%%%%%%%%%%%%%%
%%%                              Examples                                 %%%
%%%%%%%%%%%%%%%%%%%%%%%%%%%%%%%%%%%%%%%%%%%%%%%%%%%%%%%%%%%%%%%%%%%%%%%%%%%%%
\vspace{0.5cm}

Examples:

\begin{boxedminipage}[t]{\linewidth}
\begin{intens}
DESCRIPTION "Example INDEX";
DATAPOOL
  REAL {EDITABLE}
    a
   ;
END DATAPOOL;
  ...
UI_MANAGER
  INDEX
    index_a { RANGE(1,10) }
  ;
  FIELDGROUP
    fieldgroup_a (
      "Variable a"  a[index_a]:10
     ,"Index of a"  index_a
    );
END UI_MANAGER;
END.
\end{intens}
\end{boxedminipage}
