\newpage
\subsection{DATAPOOL}
%           --------
\label{sec:datapool}
%%%%%%%%%%%%%%%%%%%%%%%%%%%%%%%%%%%%%%%%%%%%%%%%%%%%%%%%%%%%%%%%%%%%%%%%%%%%%
%%%                             Description                               %%%
%%%%%%%%%%%%%%%%%%%%%%%%%%%%%%%%%%%%%%%%%%%%%%%%%%%%%%%%%%%%%%%%%%%%%%%%%%%%%
\subsubsection{Description}
\label{sec:dpdescription}
\index{DATAPOOL@\DATAPOOL}
All {\bfseries Data Items} used in an \INTENS{}-application are defined in the \DATAPOOL{} section.\\
One of the most important functions of the datapool is the dynamic allocation of memory for these data items.\\
\vspace{0.5cm}
To define {\bfseries STRUCTURES} you have to give an identifier, which will
be used as a data type to define data items later.
(see section \nameref{sec:dpstruct} on page \pageref{sec:dpstruct}).\\
\vspace{0.5cm}
You can define \SET s, which are used by the \UIMANAGER{}
to show values of data items in a Combobox or Option menu.
(see section \nameref{sec:dpset} on page \pageref{sec:dpset}).\\
\vspace{0.5cm}
You can also define \COLOR s, which are used by the \UIMANAGER{}
to show items with different colors corresponding to their values.
(see section \nameref{sec:dpcolorset} on page \pageref{sec:dpcolorset}).\\

\input{diagrams/datapool_description}
\index{DATAPOOL@\DATAPOOL!description}

%%%%%%%%%%%%%%%%%%%%%%%%%%%%%%%%%%%%%%%%%%%%%%%%%%%%%%%%%%%%%%%%%%%%%%%%%%%%%
%%%                              Data Item                                %%%
%%%%%%%%%%%%%%%%%%%%%%%%%%%%%%%%%%%%%%%%%%%%%%%%%%%%%%%%%%%%%%%%%%%%%%%%%%%%%
\newpage
\subsubsection{Data Item}
\label{sec:dpitem}
\index{DATAPOOL@\DATAPOOL!data types}
Since the \DATAPOOL{} allocates memory dynamically, it is
not necessary to declare the dimension of the data items (unsigned integers
between braces).
Nevertheless the declaration is recommended, if
the dimension is known and if it will not change during the calculation
process. Their index range always begins at 0.
See \hyperref[dia:datadimension]{data\_dimension} on page \pageref{dia:datadimension} for the syntax. \\

\input{diagrams/data_item_declaration}
\input{diagrams/data_variable_declaration}
\index{data item!declaration}
\index{INTEGER@\INTEGER!datapool data type}
\index{INT@\INT!datapool data type (see INTEGER)}
\index{REAL@\REAL!datapool data type}
\index{COMPLEX@\COMPLEX!datapool data type}
\index{CDATA@\CDATA!datapool data type}
\index{STRING@\STRING!datapool data type}

\begin{tabularx}{\textwidth}{l|X}
data item           & Description \\
\hline
\verb+identifier+  & data item identifier. It is needed for referencing this data item. \\
\REAL               & defines data items for storing real number values. \\
\INTEGER            & defines data items for storing integer number values. \\
\STRING             & defines data items for storing characters. \\
\COMPLEX            & defines data items for storing {\bfseries complex} quantities. \\
\CDATA              & defines data items for storing character or binary data.
                      (database: mapped to BLOB, no length restrictions while
                       strings are limited to 255 characters) \\
{\bfseries ID\_DATASTRUCTURE} & defines data items for storing structured data. \\
                    & must be a previously defined structure
                      (section \nameref{sec:dpstruct} on page \pageref{sec:dpstruct}) \\
\end{tabularx}
\vspace{0.5cm}

\begin{boxedminipage}[t]{\linewidth}
\begin{intens}
DATAPOOL
  REAL
    data_item_identifier_A[10],        // 1 dimension
    data_item_identifier_B[3,20],      // 2 dimensions
    data_item_identifier_C[2,4,10];    // 3 dimensions

END DATAPOOL;
\end{intens}
\end{boxedminipage}

\vspace{0.5cm}
\input{diagrams/data_dimension}
\index{data dimension!declaration}
\vspace{0.5cm}

Declare the dimension of the data item. See first paragraph in
section \nameref{sec:dpitem} on page \pageref{sec:dpitem} for more information.

\vspace{0.5cm}

\index{DATAPOOL@\DATAPOOL!predefined data items}
The following data items are {\bfseries predefined} by the system at startup
       and must not be redefined: \\
\vspace{0.5cm}
\index{data item!predefined}
\index{DATE@\DATE!predefined datapool item}
\index{USER@\USER!predefined datapool item}
\index{HOST@\HOST!predefined datapool item}
\index{IPADDR@\IPADDR!predefined datapool item}
\index{INTENS\_VERSION@\INTENSVERSION!predefined datapool item}
\index{INTENS\_VERSION\_MAJOR@\INTENSVERSIONMAJOR!predefined datapool item}
\index{INTENS\_VERSION\_MINOR@\INTENSVERSIONMINOR!predefined datapool item}
\index{INTENS\_VERSION\_PATCH@\INTENSVERSIONPATCH!predefined datapool item}
\index{INTENS\_REVISION@\INTENSREVISION!predefined datapool item}
\index{RESTUSERNAME@\RESTUSERNAME!predefined datapool item}
\index{RESTUSERNAMELIST@\RESTUSERNAMELIST!predefined datapool item}
\index{RESTBASE@\RESTBASE!predefined datapool item}
\index{REST\_SERVICE.APP\_VERSION\_MAJOR@\RESTSERVICEAPPVERSIONMAJOR!predefined datapool item}
\index{REST\_SERVICE.APP\_VERSION\_MINOR@\RESTSERVICEAPPVERSIONMINOR!predefined datapool item}
\index{REST\_SERVICE.APP\_VERSION\_PATCH@\RESTSERVICEAPPVERSIONPATCH!predefined datapool item}
\index{REST\_SERVICE.DB\_VERSION\_MAJOR@\RESTSERVICEDBVERSIONMAJOR!predefined datapool item}
\index{REST\_SERVICE.DB\_VERSION\_MINOR@\RESTSERVICEDBVERSIONMINOR!predefined datapool item}
\index{REST\_SERVICE.DB\_VERSION\_PATCH@\RESTSERVICEDBVERSIONPATCH!predefined datapool item}
\index{REST\_SERVICE.DB\_VERSION\_IGNORE@\RESTSERVICEDBVERSIONIGNORE!predefined datapool item}
\index{PLOT2D\_UIMODE@\PLOTTWODUIMODE!predefined datapool item}
\index{PLOT2D\_SYMBOLSIZE@\PLOTTWODSYMBOLSIZE!predefined datapool item}
\index{Global\_Point@\GlobalPoint!predefined datapool item}
\index{Global\_Rect@\GlobalRect!predefined datapool item}
\label{dataitempredefined}
\begin{tabularx}{\textwidth}{l|X}
Itemname    & Description \\
\hline
\DATE       & contains the current date (yyyy-mm-dd, \STRING) \\
\USER       & contains the user name (\STRING) \\
\HOST       & hostname of the working machine (use VAR(``HOST'') to access it, \STRING) \\
\IPADDR     & ip address of the working machine (\STRING) \\
\INTENSVERSION & \INTENS{} version (i.E. 5.3.1/5.3.2dev, \STRING) \\
\INTENSVERSIONMAJOR & \INTENS{} MAJOR version (i.E. 5, \INTEGER) \\
\INTENSVERSIONMINOR & \INTENS{} MINOR version (i.E. 3, \INTEGER) \\
\INTENSVERSIONPATCH & \INTENS{} PATCH version (i.E. 1/2dev, \STRING) \\
\INTENSREVISION & \INTENS{} revision (i.E. -/67-g0df41633, \STRING) \\
\RESTUSERNAME  & username used to login to the RESTful web service (\STRING) \\
\RESTUSERNAMELIST  & list of usernames to show in login dialog for RESTful web service (\STRING) \\
\RESTBASE   & base url of the RESTful web service (\STRING) \\
\RESTSERVICEAPPVERSIONMAJOR & see paragraph \nameref{par:restServiceVersionControl}
              on page \pageref{par:restServiceVersionControl} (\INTEGER) \\
\RESTSERVICEAPPVERSIONMINOR & see paragraph \nameref{par:restServiceVersionControl}
              on page \pageref{par:restServiceVersionControl} (\INTEGER) \\
\RESTSERVICEAPPVERSIONPATCH & see paragraph \nameref{par:restServiceVersionControl}
              on page \pageref{par:restServiceVersionControl} (\INTEGER) \\
\RESTSERVICEDBVERSIONMAJOR & see paragraph \nameref{par:restServiceVersionControl}
              on page \pageref{par:restServiceVersionControl} (\INTEGER) \\
\RESTSERVICEDBVERSIONMINOR & see paragraph \nameref{par:restServiceVersionControl}
              on page \pageref{par:restServiceVersionControl} (\INTEGER) \\
\RESTSERVICEDBVERSIONPATCH & see paragraph \nameref{par:restServiceVersionControl}
              on page \pageref{par:restServiceVersionControl} (\INTEGER) \\
\RESTSERVICEDBVERSIONIGNORE & see paragraph \nameref{par:restServiceVersionControl}
              on page \pageref{par:restServiceVersionControl} (\INTEGER) \\
\PLOTTWODUIMODE  & see paragraph \nameref{par:uiplot2duimode} in section \nameref{sec:uiplot2d}
              on page \pageref{par:uiplot2duimode} (\STRING) \\
\PLOTTWODSYMBOLSIZE  & see paragraph \nameref{par:uiplot2symbolsize} in section \nameref{sec:uiplot2d}
              on page \pageref{par:uiplot2symbolsize} (\INTEGER) \\
\GlobalPoint & structure object with data items X, Y and Y2 \newline
              Used in UI Mode ``Select Point''.
              See paragraph \nameref{par:uiplot2duimode} in section \nameref{sec:uiplot2d}
              on page \pageref{par:uiplot2duimode} \\
\GlobalRect & structure object with data items X1, Y1, X2 and Y2 \newline
              Used in UI Mode ``Select Rectangle''.
              See paragraph \nameref{par:uiplot2duimode} in section \nameref{sec:uiplot2d}
              on page \pageref{par:uiplot2duimode} \\
%% \ProgressDialog & structure object with data items MainTitle, MainPercent, SubTitle, SubPercent, ErrorString \newline
%%               TODO \\
%% \ProgressDialogAbortCommand & TODO (\STRING) \\
%% \ProgressDialogLoopTitle & TODO (\STRING) \\
\end{tabularx}

\input{diagrams/data_variable_attributes}
\index{DATAPOOL@\DATAPOOL!data item attributes}

\index{EDITABLE@\EDITABLE!datapool attributes}
\index{OPTIONAL@\OPTIONAL!datapool attributes}
\index{LOCKABLE@\LOCKABLE!datapool attributes}
\index{SCALAR@\SCALAR!datapool attributes}
\index{CELL@\CELL!datapool attribute}
\index{GLOBAL@\GLOBAL!datapool attribute}
\index{OMIT\_TTRAIL@\OMITTTRAIL!datapool attribute}
\index{NO\_DEPENDENCIES@\NODEPENDENCIES!datapool attribute}
\index{CLASSNAME@\CLASSNAME!datapool attributes}
\label{dataitemattributes}
\begin{tabularx}{\textwidth}{l|X}
Attributes       & Description \\ \hline
\EDITABLE        & The value can be changed (edited) by the user
                   (see also section
                   \nameref{sec:uimanager} on page \pageref{sec:uimanager})\\
\OPTIONAL        & Same as \EDITABLE. The corresponding text fields
                   may be given different %%X resources
                   foreground/background color, font etc. \\
                   %%(see section \nameref{sec:x-resources} on page
                   %%\pageref{sec:x-resources})\\
\LOCKABLE        & The data item can be protected against being overwritten
                   by stream operations.
                   (see also section
                   \nameref{sec:uimanager} on page \pageref{sec:uimanager})\\
\SCALAR          & Data items with this attribute are transferred as scalars
                   instead of matrices to any Matlab function
                   (or instread of vectors to and from the database).
                   They are not written as a list in a \JSON{} \STREAM.\\
\CELL            & The data item is transferred to the Matlab workspace as cell instead of array. \\
\GLOBAL          & The data item is not changed by cycle operations. \\
\OMITTTRAIL      & The data item is not handled by transactions (ABORT, UNDO, ...). \\
\NODEPENDENCIES  & No dependencies between this input and its outputs (results) are added.
                   (see paragraph \nameref{par:stdependency} on page \pageref{par:stdependency}) \\
\CLASSNAME       & Data items having a classname are transferred as
                   objects to any Matlab function. \newline
                   The \CLASSNAME{} can also be used in \FUNCTIONS: \CLASSNAME(item). \\
\end{tabularx}

\index{data item!options}
\input{diagrams/data_item_options}
\index{DATAPOOL@\DATAPOOL!data item options}
\index{SET@\SET!datapool item option}
\index{INDEXED\_SET@\INDEXEDSET!datapool item option}
\index{COLOR@\COLOR!datapool item option}
\index{NO\_COLORBIT@\NOCOLORBIT!datapool item option}
\index{FUNC@\FUNC!datapool item option}
\index{LABEL@\LABEL!datapool item option}
\index{UNIT@\UNIT!datapool item option}
\index{UNITS (see UNIT)}
\index{PATTERN@\PATTERN!datapool item option}
\index{HELPTEXT@\HELPTEXT!datapool item option}
\index{BUTTON@\BUTTON!datapool item option}
\index{SLIDER@\SLIDER!datapool item option}
\index{PROGRESS@\PROGRESS!datapool item option}
\index{RANGE@\RANGE!datapool item option}
\index{STEP@\STEP!datapool item option}
\index{TOGGLE@\TOGGLE!datapool item option}
\index{RADIO@\RADIO!datapool item option}
\index{CLASSNAME@\CLASSNAME!datapool item option}
\index{SCALAR@\SCALAR!datapool item option}
\index{CELL@\CELL!datapool item option}
\index{HIDDEN@\HIDDEN!datapool item option}
\index{PLACEHOLDER@\PLACEHOLDER!datapool item option}

\input{diagrams/data_item_more_option}
\index{PERSISTENT@\PERSISTENT!datapool item option}
\index{TRANSIENT@\TRANSIENT!datapool item option}
\index{DBATTR@\DBATTR!datapool item option}
\index{DBUNIT@\DBUNIT!datapool item option}
\index{TAG@\TAG!datapool item option}
\index{FOLDER@\FOLDER!datapool item option}
\index{NONE@\NONE!datapool item option}
\index{STRING\_DATE@\STRINGDATE!datapool item option}
\index{STRING\_TIME@\STRINGTIME!datapool item option}
\index{STRING\_DATETIME@\STRINGDATETIME!datapool item option}
\index{PASSWORD@\PASSWORD!datapool item option}
\index{WHEEL\_EVENT@\WHEELEVENT!datapool item option}

\begin{tabularx}{\textwidth}{l|X}
Options          & Description \\ \hline
\SET             & The items values are part of a \SET. A set is a limited number of values of the same type.
                   The item is displayed as Combobox on the user interface.
                   \verb+ID_DATASET+ must be previously defined (see section \nameref{sec:dpset} page \pageref{sec:dpset})\\
\INDEXEDSET      & Like \SET. When the data item is used as a vector, each element can have a different
                   list of values to select from. \\
\COLOR{} = ...   & The field is shown in the colors that corresponds to its items value.
                   \verb+ID_COLORSET+ must be previously defined (see section \nameref{sec:dpcolorset} page \pageref{sec:dpcolorset})\\
\COLOR           & String data item to define a color.
                   The item is displayed as a color picker: a button that shows the color.
                   Pressing the button opens a color dialog to choose a color.
                   The chosen color is stored as the value of the variable. \\
\NOCOLORBIT      & The eight available color bits (see section \nameref{fu:set:statement} page \pageref{fu:set:statement}) are interpreted as a
                   8-bit number. This gives the possibility to use up to 255 different colors. \\
\FUNC            & Defines the function that will be called after an interactive modification
                   of the items value.
                   (see also section \nameref{sec:functions} on page \pageref{sec:functions})\\
\LABEL{} = ...   & The item has a label which can be used by the \UIMANAGER{} and \STREAMER.
                   (see also section \nameref{sec:uifieldgroup} on page \pageref{sec:uifieldgroup}) \\
\UNIT            & The item has a unit which can be used by the \UIMANAGER{} and \STREAMER.
                   (see also section \nameref{sec:uifieldgroup} on page \pageref{sec:uifieldgroup})\\
\PATTERN         & A regular expression that defines the possible input values. \\
\HELPTEXT        & For each item a helptext may be defined.
                    This text appears as soon as the mouse-pointer crosses the field,
                   which contains the data-item.\\
\BUTTON          & The item is displayed as button on the user interface.
                    Activating the button executes the associated function
                   (see above \FUNC).\\
\SLIDER          & The item is displayed as a slider. \\
\PROGRESS        & The item is displayed as a progress bar. \newline
                   This makes it possible to show a progress bar inside a \FIELDGROUP{}. Assign the desired
                   value (from 0 to 100) to the (integer) variable and the item shows the progress. \\
\RANGE           & Minimal and maximal values for a \SLIDER. \\
\STEP            & Difference between the labels of \SLIDER. \\
\TOGGLE          & The item is displayed as Toggle on the user interface.
                   Activating the toggle changes the value from 0 to 1,
                   deactivating changes the value from 1 to 0.
                   A variable that has no value (invalid) is shown as a deactivated toggle,
                   just as a variable with the value 0. \\
\RADIO           & The item is displayed as ragio button on the user interface.
                   Activating the radio button changes the value from 0 to 1,
                   deactivating changes the value from 1 to 0.
                   A variable that has no value (invalid) is shown as a deactivated radio button,
                   just as a variable with the value 0. \\
\LABEL           & The item is displayed as a label (not editable).\\
\CLASSNAME       & Data items having a classname are transferred as objects to the Matlab workspace. \newline
                   The \CLASSNAME{} can also be used in \FUNCTIONS: \CLASSNAME(item) \\
\SCALAR          & Data items with this attribute are transferred as scalars
                      instead of matrices to the Matlab workspace
                      (or instread of vectors to and from the database).
                      They are not written as a list in a \JSON{} \STREAM.\\
\CELL            & The data item is transferred to Matlab workspace as cell instead of array. \\
\HIDDEN          & The data item is not transferred to Matlab.
                   It is, by default, not written in \JSON{} \STREAM{}s (see \nameref{dia:stjsonoptions} on page \pageref{dia:stjsonoptions}).
                   It is however sent to the rest service (see \nameref{dia:restServicestatement} on page \pageref{dia:restServicestatement}).\\
\PLACEHOLDER     & For each item a placeholder text may be defined.
                   This text appears in the field which contains the data-item when it is \INVALID.\\
\PERSISTENT      & This data item can be stored to and retrieved from the database.\\
\TRANSIENT       & This data item is not stored to and retrieved from the database.
                   It is useful in a \PERSISTENT{} \STRUCT.\\
\DBATTR          & defines the corresponding db attribute name.\\
\DBUNIT          & defines the corresponding db attribute unit.\\
\TAG             & The TAG-identifier is referenced by the navigators COL-definition.
                   (see section \nameref{sec:uinavigatoroptions} on page \pageref{sec:uinavigatoroptions})\\
%%\FOLDER{} = \NONE  & verhindert, dass die Struktur im Navigator als Folder dargestellt wird.\\
\STRINGDATE      & String data item to show and enter a date (popup calendar may be used).\\
\STRINGTIME      & String data item to show and enter a time.\\
\STRINGDATETIME  & String data item to show and enter a date and time.\\
\PASSWORD        & String data item is displayed in password mode.\\
\WHEELEVENT      & The mouse wheel can be used to increment and decrement the value,
                   just as the up and down keys can be used for every numeric data item. \newline
                   This does not work when commandline option \texttt{--}withoutArrowKeys is given. \newline
                   This option enables the mouse wheel event for one numeric data item,
                   whereas the commandline option \texttt{--}withWheelEvent enables it for every numeric data item. \\
\end{tabularx}

\begin{multicols}{2}
\input{diagrams/data_tags}
\input{diagrams/data_tag}
\end{multicols}

\begin{tabularx}{\textwidth}{l|X}
tag          & Description \\ \hline
{\bfseries IDENTIFIER} & Defines a new tag. \\
{\bfseries ID\_TAG} & References an existing tag. \\
\end{tabularx}

%%%%%%%%%%%%%%%%%%%%%%%%%%%%%%%%%%%%%%%%%%%%%%%%%%%%%%%%%%%%%%%%%%%%%%%%%%%%%
%%%                              Data Set                                 %%%
%%%%%%%%%%%%%%%%%%%%%%%%%%%%%%%%%%%%%%%%%%%%%%%%%%%%%%%%%%%%%%%%%%%%%%%%%%%%%
\newpage
\subsubsection{Data Set}
\label{sec:dpset}
\index{DATAPOOL@\DATAPOOL!data set}
Data sets are used for assigning labels to data items. Data sets may be assigned
to data items using the option SET=data-set\_identifier of a previously declared
data set. (section \nameref{dia:dataitemoptions} on page \pageref{dia:dataitemoptions}) \\
By editing such a data item the associated label-strings are shown, while the corresponding
values are transferred to the external process.\\

\input{diagrams/set_declaration}
\input{diagrams/data_set_attr}
\input{diagrams/data_set_item}
\index{SET@\SET}
\index{SET@\SET!datapool data set}

\index{GLOBAL@\GLOBAL!data set option}
\index{INVALID@\INVALID!data set option INVALID=NONE}
\index{NONE@\NONE!data set option INVALID=NONE}
\begin{tabularx}{\textwidth}{l|X}
Option           & Description \\ \hline
\GLOBAL          & The data set is not changed by cycle operations. \\
\INVALID=\NONE   & No empty (invalid) entry is created. \\
\end{tabularx}
\vspace{0.5cm}

The label-strings of a set define the option menu labels. The value types
can be \REAL, \INTEGER{} or \STRING.
The values correspond to the text of the label.
If the values are missing, \INTENS{} assumes increasing integer numbers
starting at 0 for real and integer items. String items contain the string itself.


\begin{boxedminipage}[t]{\linewidth}
\begin{intens}
DATAPOOL
  SET
    set_identifier ( "circle" = 0, "triangle" = 1, "square"= 2 );

  INTEGER {EDITABLE}
    data_item_identifier {SET = set_identifier};

END DATAPOOL;
\end{intens}
\end{boxedminipage}

%%%%%%%%%%%%%%%%%%%%%%%%%%%%%%%%%%%%%%%%%%%%%%%%%%%%%%%%%%%%%%%%%%%%%%%%%%%%%
%%%                              Dynamic Combobox                         %%%
%%%%%%%%%%%%%%%%%%%%%%%%%%%%%%%%%%%%%%%%%%%%%%%%%%%%%%%%%%%%%%%%%%%%%%%%%%%%%
\newpage
\subsubsection{Dynamic Combobox}
\index{STRING@\STRING!dynamic combobox}
\label{stringdynamiccombobox}
Sometimes it is not possible to use a \SET{} to build a combobox. Then, a \STRING{}
item can be changed to a combobox by assigning a special value to it:

\verb+{"value": "a", "input":["A","B","C",null], "output":["a","b","c",null]}+

It must be a JSON object with the members:
\begin{itemize}
  \item \Slanted{value}: the selected value. One of the values of \Slanted{output}.
  \item \Slanted{input}: list of strings. The elements presented to the user in a combobox.
  \item \Slanted{output}: list of values (normally strings). The values corresponding to the \Slanted{input} values.
\end{itemize}
When the first element of \Slanted{input} is selected, \Slanted{value} is set to the first element of \Slanted{output}. \\
To extract the \Slanted{value} from the total string in a function, use \STRINGVALUE{}(item).


%%%%%%%%%%%%%%%%%%%%%%%%%%%%%%%%%%%%%%%%%%%%%%%%%%%%%%%%%%%%%%%%%%%%%%%%%%%%%
%%%                             Data ColorSet                             %%%
%%%%%%%%%%%%%%%%%%%%%%%%%%%%%%%%%%%%%%%%%%%%%%%%%%%%%%%%%%%%%%%%%%%%%%%%%%%%%
\newpage
\subsubsection{Data ColorSet}
\label{sec:dpcolorset}
\index{DATAPOOL@\DATAPOOL!data colorset}
Data colorsets are used for assigning colors to values. Data colorsets may be assigned
to data items using the option COLOR=data-colorset\_identifier of a previously declared
data colorset. (section \nameref{dia:dataitemoptions} on page \pageref{dia:dataitemoptions}) \\
The field of such a data item will change to the associated colors that correspond to its value.\\

\input{diagrams/color_declaration}
\input{diagrams/data_colorset_item}
\index{COLOR@\COLOR}
\index{COLOR@\COLOR!datapool data set}

\begin{tabularx}{\textwidth}{l|X}
value                       & Description \\
\hline
\INVALID                    & invalid value. \\
\verb+data colorset value+  & less than, more than or exaclty one specific value
                              (see paragraph \nameref{par:dpcolorsetvaluerange} on page \pageref{par:dpcolorsetvaluerange}).\\
\verb+data colorset range+  & data range (see paragraph \nameref{par:dpcolorsetvaluerange} on page \pageref{par:dpcolorsetvaluerange}).\\
\ELSE                       & all other values (not matched by any other colorset item values). \\
\end{tabularx}
\vspace{0.5cm}

The pair of bg\_color\_string and fg\_color\_string of a colorset define the
background and foreground color used for value defined on the left. \\
The color-string may be in one of these formats:
\begin{itemize}
  \item \#RGB (each of R, G and B is a single hex digit)
  \item \#RRGGBB
  \item A name from the list of colors defined in the list of
  \href{http://www.w3.org/TR/SVG/types.html#ColorKeywords}{SVG color keyword names}
  provided by the World Wide Web Consortium;
  for example, "steelblue" or "gainsboro". These color names work on
  all platforms.
\end{itemize}

Instead of providing static color-strings, the colors can also be a data reference. If the data item has
a valid color string value, that value is used. This can be used i.E. to change the color of plot curves
(see section \nameref{sec:uiplot2d} on page \pageref{sec:uiplot2d}).

\newpage
\paragraph{Data ColorSet Value and Range}
\label{par:dpcolorsetvaluerange}
~\\[0.5cm]

\input{diagrams/data_colorset_value}
\input{diagrams/data_colorset_range}

\begin{tabularx}{\textwidth}{l|X}
value                    & Description \\
\hline
\verb+real or int value+ & any value or mathematical 'combination' as shown in the example
                           in section \nameref{sec:scale} on page \pageref{sec:scale}. \\
\verb+string+            & any string constant or 'combination'.
                           (see section \nameref{sec:string} on page
                           \pageref{sec:string}) \\
\verb+range data reference+ & references a data item declared in the datapool
                             (section \nameref{sec:rangedatareference} on page
                              \pageref{sec:rangedatareference}). \\
\end{tabularx}
\vspace{0.5cm}

\begin{boxedminipage}[t]{\linewidth}
\begin{intens}
DATAPOOL
  COLOR
    red_green_blue_color (
      INVALID = ( "red", "black" ),
      < 0     = ( "#fff", "#f00" ),
      RANGE ( 0, <2 ) = ( "#ffffff", "#00ff00" ),
      2 = ( "#ffffff", "#0000ff" ),
      ELSE = ( "#ffffff", "#0000ff" )
    )
  ;

  INTEGER {EDITABLE}
    data_item_identifier {COLOR = red_green_blue_color};

END DATAPOOL;
\end{intens}
\end{boxedminipage}

%%%%%%%%%%%%%%%%%%%%%%%%%%%%%%%%%%%%%%%%%%%%%%%%%%%%%%%%%%%%%%%%%%%%%%%%%%%%%
%%%                     Structured Data Items                             %%%
%%%%%%%%%%%%%%%%%%%%%%%%%%%%%%%%%%%%%%%%%%%%%%%%%%%%%%%%%%%%%%%%%%%%%%%%%%%%%
\newpage
\subsubsection{Structure definition}
\label{sec:dpstruct}
\index{DATAPOOL@\DATAPOOL!data structure}
\index{DATAPOOL@\DATAPOOL!struct}
Data items may be grouped in structures. This enhances the possibilities
of storing and transmitting data. \\
A structure holds several data items of any valid data type described in
section \nameref{sec:dpitem} on page \pageref{sec:dpitem}.

\input{diagrams/structure_declaration}
\index{STRUCT@\STRUCT}
\index{STRUCT@\STRUCT!inheritance}
\index{  @Signs / Characters!: (colon)!structure inheritance}

\begin{tabularx}{\textwidth}{l|X}
Structure definition            & Description \\
\hline
{\bfseries ID\_DATASTRUCTURE}   & identifier of previously defined structure. \\
                                & inherit definition of this struct. \\
\verb+data item declaration+ & see section \nameref{sec:dpitem} on page \pageref{sec:dpitem}. \\
\end{tabularx}
\vspace{0.5cm}



\begin{boxedminipage}[t]{\linewidth}
\begin{intens}
DATAPOOL
  STRUCT structure_def
  {
     INTEGER {EDITABLE}
        data_item_identifier_1,
        data_item_identifier_2;
     REAL {OPTIONAL}
        data_item_identifier_3;
  };

  structure_def
    structure_identifier_1,
    structure_identifier_2;

END DATAPOOL;
\end{intens}
\end{boxedminipage}


%%%%%%%%%%%%%%%%%%%%%%%%%%%%%%%%%%%%%%%%%%%%%%%%%%%%%%%%%%%%%%%%%%%%%%%%%%%%%
%%%                              Examples                                 %%%
%%%%%%%%%%%%%%%%%%%%%%%%%%%%%%%%%%%%%%%%%%%%%%%%%%%%%%%%%%%%%%%%%%%%%%%%%%%%%
\newpage
\subsubsection{Examples}
\label{sec:dpexamples}


\begin{boxedminipage}[t]{\linewidth}
\begin{intens}
DESCRIPTION "Example of data defining and referring"

DATAPOOL
  SET shape_set ( "circle" = 0, "triangle" = 1, "square"= 2 );

  STRUCT Object {
     INTEGER {EDITABLE}
        shape {SET = shape_set,COMBOBOX
              ,LABEL = "The Shape"};
     REAL {OPTIONAL}
        size  {LABEL = "The Size", UNIT = "cm"};
  };

  Object obj[20];

  STRING {EDITABLE, SCALAR}
     text;

END DATAPOOL;

UIMANAGER
  FIELDGROUP
    fieldgroup_identifier
    (
      "Select first shape:" obj[0].shape,
      "Enter first size:"   obj[0].size,

      "Select last shape:"  obj[19].shape,
      "Enter last size:"    obj[19].size,

      "Enter a remark:"     text
    )
  ;

  FORM
    form_identifier {MAIN}
    (
      (fieldgroup_identifier)
    )
  ;

END UIMANAGER;

\end{intens}
\end{boxedminipage}
