\newpage
\subsection{STREAMER}
%           ---------
\label{sec:streamer}
%%%%%%%%%%%%%%%%%%%%%%%%%%%%%%%%%%%%%%%%%%%%%%%%%%%%%%%%%%%%%%%%%%%%%%%%%%%%%
%%%                             Description                               %%%
%%%%%%%%%%%%%%%%%%%%%%%%%%%%%%%%%%%%%%%%%%%%%%%%%%%%%%%%%%%%%%%%%%%%%%%%%%%%%
\subsubsection{Description}
\label{sec:stdescription}
The Streamer manages all input and output formats. A stream consists
of a sequence of data items and string constants.

\paragraph{Dependency}
\label{par:stdependency}
Together with the
datapool and the operator, the streamer controls the dependencies
between input and output variables (datapool items).
In order to keep the data consistent,
all data items that belong to an output stream are set invalid
when one of the items of a corresponding input stream is modified.

Dependencies between input and output \STREAM{}s are defined through
\PROCESSGROUP{}s and \MESSAGEQUEUE{} \REQUEST{}s. They are activated
when the \PROCESSGROUP{} or the \MESSAGEQUEUE{} \REQUEST{} is run.

The user is asked for confirmation before dependencies are cleared
- unless the option \AUTOCLEARDEPENDENCIES{} is used.

\STREAM{}s can be excluded from dependencies using the option \NODEPENDENCIES{}
(see \nameref{dia:jobmessagequeueoption} on page \pageref{dia:jobmessagequeueoption} or
\nameref{dia:jobpluginoption} on page \pageref{dia:jobpluginoption}) or by not giving
a dependency option and using the \INTENS command line argument
\texttt{--}defaultMessageQueueDependencies false.

\input{diagrams/streamer_description}
\index{STREAMER@\STREAMER}

%%%%%%%%%%%%%%%%%%%%%%%%%%%%%%%%%%%%%%%%%%%%%%%%%%%%%%%%%%%%%%%%%%%%%%%%%%%%%
%%%                              IO Stream                                %%%
%%%%%%%%%%%%%%%%%%%%%%%%%%%%%%%%%%%%%%%%%%%%%%%%%%%%%%%%%%%%%%%%%%%%%%%%%%%%%
\subsubsection{IO Stream}
\label{sec:ststream}
\input{diagrams/st_declaration}
\vspace{1cm}

\index{XML@\XML!streamer format command}
\index{JSON@\JSON!streamer format command}

A stream is used to read or write to or from data items, referencing them by their identifier. \\
\vspace{1cm}

A stream is a sequence of format commands that has a unique identifier.
An identifier within a format command references a data item.
If an identifier is preceeded by an exclamation mark it will be checked
to have a valid value before the corresponding calculation program
is called. Item values that are not valid will not be transferred.

The streamer adds one blank character (see \DELIMITER{} on page \pageref{dia:stoptionlist}) after data items to separate them.
This blank character is added only after data items and indices without
a fixed width
 (see {\bfseries field conversion} on page \pageref{par:fieldconversion}
  and {\bfseries field length} on page \pageref{par:stfieldlength}).
\index{  @Signs / Characters!: (colon)!streamer dataitem-format width}
\index{width}

The line containing the \XML{} token defines an xml stream
(see section \nameref{sec:stxmlcommand} on page \pageref{sec:stxmlcommand}).

The line containing the \JSON{} token defines an json stream
(see section \nameref{sec:stjsoncommand} on page \pageref{sec:stjsoncommand}).

Stream identifiers are registered datapool items and contain the filename
(if any) of their last read or write operation.

\input{diagrams/st_option_list}
\index{LATEX@\LATEX!stream option}
\index{URL@\URL!stream option}
\index{DELIMITER@\DELIMITER!stream delimiter}
\index{LOCALE@\LOCALE!stream option}
\index{PROCESS@\PROCESS!stream filter process}
\index{APPEND@\APPEND!stream option}
\index{NO\_GZ@\NOGZ!stream option}
\index{FILE@\FILE!stream option}

\begin{tabularx}{\textwidth}{l|X}
option                   & Description \\
\hline
\LATEX                   & \LaTeX{} special characters
                           (i.E. \textbackslash{} \textasciicircum{} \textbar{} or \{)
                           are replaced with their \LaTeX{} command. \\
\URL                     & reserved characters are percent-encoded and no \DELIMITER{}s are added. \\
\DELIMITER               & defines the character used to separete stream items.
                           default is the blank character. \\
\LOCALE                  & stream uses locale decimal separator. \\
\PROCESS                 & defines a filter process.
                           The streams data is processed by the filter process. \\
\APPEND                  & append received data to the \CDATA{} element of the \STREAM. \\
\NOGZ                    & Stream content is not gunzipped when reading a file stream,
                           although the filename ends in '.gz'. \\
\FILE                    & write to or read from a temporary file instead of the \DATAPOOL. \newline
                           This reduces memory needed by \INTENS{}, but values are not in the \DATAPOOL. \\
\end{tabularx}
\vspace{0.5cm}

Not all combinations are allowed:
\begin{itemize}

\item \LATEX{} or \URL{} \\
      you can only choose one filter

\item \URL{} : no \DELIMITER{} \\
      No \DELIMITER{}s are added with \URL{} option

\end{itemize}

\input{diagrams/st_format_command}
\label{fig:st_format_command}
\index{  @Signs / Characters!\texttt{"!} (exclamation mark)!streamer validation check}
\index{  @Signs / Characters!\# (hash)!streamer index repesent.}
\index{  @Signs / Characters!: (colon)!streamer skip width}
\index{MATRIX@\MATRIX!streamer format command}
\index{Scale factors!streamer}
\index{DATASET\_TEXT@\DATASETTEXT!streamer format command}
\index{SET@\SET!dataset\_text streamer format command}
\index{STRING\_DATE@\STRINGDATE!streamer format command}
\index{STRING\_TIME@\STRINGTIME!streamer format command}
\index{STRING\_DATETIME@\STRINGDATETIME!streamer format command}
\index{STRING\_VALUE@\STRINGVALUE!streamer format command}
\index{PLOTGROUP@\PLOTGROUP!streamer format command}
%%\index{XMLGROUP@\XMLGROUP!streamer format command}
\index{SKIP@\SKIP!streamer format command}
\index{EOLN@\EOLN!streamer format command}

\input{diagrams/st_plotgroup_option}
\label{fig:st_plotgroup_option}
\index{RANGE@\RANGE!streamer plotgroup option}
\index{TRANSPARENT@\TRANSPARENT!streamer plotgroup option}

\begin{tabularx}{\textwidth}{l|X}
format command        & description \\
\hline
{\bfseries !}         & perform a validation check \\
\verb+data reference+ & the possibly indexed data item
                        (see section \nameref{sec:stvariables} page \pageref{fig:st_data_reference}) \\
\MATRIX               & The data item will be transferred as a matrix (see section
                         \nameref{sec:stmatrix} page \pageref{sec:stmatrix}) \\
\verb+scale factor+   & see section \nameref{sec:scale} page
                      \pageref{sec:scale}. \\
                      & The value of the data item is multiplied by the scale factor on
an output operation and divided by the scale factor on an input operation. \\
\verb+field conversion+ & see page \pageref{par:fieldconversion}. \\
\DATASETTEXT          & includes the string of the \SET{} associated with the item (can only be
                        used if the item has a \SET) \\
\STRINGDATE           & prints the date in the local format (i.E. 24.01.08 instead of 2008-01-24)
                    ( see section \nameref{dia:dataitemoptions} on page \pageref{dia:dataitemoptions} ) \\
\STRINGTIME           & prints the time in the local format (i.E. 15:08:00:000 instead of 15:08:00)
                    ( see section \nameref{dia:dataitemoptions} on page \pageref{dia:dataitemoptions} ) \\
\STRINGDATETIME       & prints the date-time in the local format (i.E. 24.01.08 15:08 instead of 2008-01-24T15:08:00)
                    ( see section \nameref{dia:dataitemoptions} on page \pageref{dia:dataitemoptions} ) \\
\STRINGVALUE          & extracts the value of a dynamic \COMBOBOX{}
                    ( see section \nameref{stringdynamiccombobox} on page \pageref{stringdynamiccombobox} ) \\
\verb+#+              & represents the explicit index of {\bfseries pre-indexed} arrays. \\
                      & On input-operations data will not be filled into the array in
                        order of incoming from the stream, but exactly to the position where
                         this value points to. \\
                      & On output-operations it represents the index value  of
                        the data-item in the array. \\
\verb+# IDENTIFIER+   & \verb+#+ may be followed by any number or letter sign. \\
                      & This is used as an identifier when having more than one array-index. \\
                      &  (section \nameref{sec:stvariables}
                           page \pageref{fig:st_data_reference}), each array may have its own index. \\
\verb+string+         & a string constant (section \nameref{sec:string} page \pageref{sec:string}) \\
\verb+st field length+ & see page \pageref{par:stfieldlength}. \\
\PLOTGROUP            & the following identifier specifies a plot group which will be included as
                         PostSript file (Should only be used with \LaTeX) \\
%%\XMLGROUP             & the following identifier specifies a plugin which will be included as
%%                         PostScript file (Should only be used with \LaTeX) \\
\MAIN{} {\bfseries ID\_FORM} & \XML{} tree of {\bfseries MAINFORM} or {\bfseries ID\_FORM}. \\
\SKIP{}\verb+:n+      & ignore the following \verb+n+ characters \\
\EOLN                 & a new line \\
( )                   & Format commands within a second pair of parentheses
                        build a repetition group. Each repetition group must
                        have at least one array item. \\ \hline
\RANGE{} (X-val1, X-val2)& Shows plot beginning at x-axis position X-val1 through X-val2 \\
\TRANSPARENT          & Shows the plot with (\TRUE, default) or without (\FALSE) a transparent background. \\
\end{tabularx}
\vspace{0.5cm}

\begin{boxedminipage}[t]{\linewidth}
\begin{alltt}
STREAMER
  streamer_identifier_1 (data_item_identifier_1);
  streamer_identifier_2 (#, data_item_identifier_2[#]);
  streamer_identifier_3 (#1, (#2, data_id_1[#1], data_id_2[#2]));
END STREAMER;
\end{alltt}
\end{boxedminipage}

\vspace{0.5cm}

\input{diagrams/field_conversion}
\label{par:fieldconversion}
\vspace{0.5cm}

Field conversion is optional. \\
The first : is followed by {\bfseries width},
the second by {\bfseries precision} and
the third by {\bfseries \TSEP}.\\

\index{  @Signs / Characters!- (hyphen)!left alignment}
\index{  @Signs / Characters!: (colon)!streamer dataitem-format width}
\index{  @Signs / Characters!: (colon)!streamer dataitem-format precision}
\index{  @Signs / Characters!: (colon)!streamer dataitem-format tsep}
\index{width}
\index{precision}
\index{TSEP@\TSEP!streamer dataitem-format}
\begin{tabularx}{\textwidth}{l|X}
field conversion   & description \\
\hline
{\verb+-+}         & Alignment: alignes the data item to the left (default is right) \\
{\verb+width+}     & defines the length of the field \\
{\verb+precision+} & a) defines the number of digits after the decimal point for \REAL{} items, \newline
                     b) has no meaning for \INTEGER{} and \STRING{} items\\
\TSEP              & a) Thousand separator (12{\bfseries '}345.67) for \REAL{} items, \newline
                     b) has no meaning for \INTEGER{} and \STRING{} items\\ \\
\end{tabularx}
\vspace{0.5cm}

\begin{boxedminipage}[t]{\linewidth}
\begin{alltt}
STREAMER
  frequency_stream (frequency:10:2:TSEP);
END STREAMER;
\end{alltt}
\end{boxedminipage}

gives the following stream:\\[2ex]

123'456.78
\vspace{0.5cm}

\input{diagrams/st_field_length}
\label{par:stfieldlength}

\index{  @Signs / Characters!- (hyphen)!left alignment}
\index{  @Signs / Characters!: (colon)!streamer dataitem-format width}
\index{width}
\begin{tabularx}{\textwidth}{l|X}
field length       & description \\
\hline
{\verb+-+}         & Alignment: alignes the data item to the left (default is right) \\
{\verb+width+}     & defines the length of the field \\
\end{tabularx}

\subsubsection{XML format command}
\label{sec:stxmlcommand}

The \XML{} token is used to export datapool variables to xml-files or import values from such files into datapool variables. \\
\vspace{0.5cm}

\label{fig:sntx_st_xml_command}
\input{diagrams/st_xml_format_command}

\index{DATAPOOL@\DATAPOOL!xml format command (streamer)}
\index{CYCLE@\CYCLE!xml format command (streamer)}
\begin{tabularx}{\textwidth}{l|X}
xml format command & description \\
\hline
\verb+st data reference+ & see section \nameref{sec:stvariables} on page \pageref{fig:st_data_reference} \\
\DATAPOOL          & all data items defined in datapool are in- or exported by filestream actions.\\
\CYCLE             & all data items defined in cycle are in- or exported by filestream actions.\\
\end{tabularx}

\input{diagrams/st_xml_options}

\index{ATTRS@\ATTRS!xml format option (streamer)}
\index{UNIT@\UNIT!xml format option (streamer)}
\index{LABEL@\LABEL!xml format option (streamer)}
\index{HELPTEXT@\HELPTEXT!xml format option (streamer)}
\index{SCHEMA@\SCHEMA!xml format option (streamer)}
\index{NAMESPACE@\NAMESPACE!xml format option (streamer)}
\index{VERSION@\VERSION!xml format option (streamer)}
\index{STYLESHEET@\STYLESHEET!xml format option (streamer)}
\begin{tabularx}{\textwidth}{l|X}
xml format option & description \\
\hline
\UNIT             & include unit attribute (see chapter \nameref{dia:dataitemoptions} on page \pageref{dia:dataitemoptions}). \\
\LABEL            & include label attribute (see chapter \nameref{dia:dataitemoptions} on page \pageref{dia:dataitemoptions}). \\
\HELPTEXT         & include helptext attribute (see chapter \nameref{dia:dataitemoptions} on page \pageref{dia:dataitemoptions}). \\
\SCHEMA           & set the xml schema.\\
\NAMESPACE        & set the xml namespace.\\
\VERSION          & add a version attribute to the xml-file.\\
\STYLESHEET       & set the xml stylesheet.\\
\end{tabularx}
\vspace{0.5cm}

\begin{boxedminipage}[t]{\linewidth}
\begin{alltt}
\STREAMER
  xml_stream_1 \{\XML \{ \ATTRS ( \UNIT ) \} \} (data_item_identifier);
  xml_stream_2 \{\XML\} (\DATAPOOL);
\END \STREAMER;

\OPERATOR
  \FILESTREAM
    FStream_1 = xml_stream_1;
    FStream_2 = xml_stream_2;
\END \OPERATOR;
\end{alltt}
\end{boxedminipage}

\vspace{0.5cm}

\subsubsection{JSON format command}
\label{sec:stjsoncommand}

The \JSON{} token is used to export datapool variables to json-files or import values from such files into datapool variables. \\
\vspace{0.5cm}

\label{fig:sntx_st_json_command}
\input{diagrams/st_json_format_command}

\index{DATAPOOL@\DATAPOOL!json format command (streamer)}
\index{CYCLE@\CYCLE!json format command (streamer)}
\begin{tabularx}{\textwidth}{l|X}
json format command & description \\
\hline
\verb+st data reference+ & see section \nameref{sec:stvariables} on page \pageref{fig:st_data_reference} \\
\DATAPOOL           & all data items defined in datapool are in- or exported by filestream actions.\\
\CYCLE              & all data items defined in cycle are in- or exported by filestream actions.\\
\end{tabularx}

\input{diagrams/st_json_options}

\index{PROCESS@\PROCESS!stream filter process}
\index{INDENT@\INDENT!json stream option}
\index{HIDDEN@\HIDDEN!json stream option}
\index{TRANSIENT@\TRANSIENT!json stream option}

\begin{tabularx}{\textwidth}{l|X}
json format option & description \\
\PROCESS   & defines a filter process.
             The streams data is processed by the filter process. \\
\INDENT    & defines the indent level of a \JSON{} \STREAM{} (pretty print). \newline
             0 (default): no indentation. \\

\HIDDEN    & write values of hidden variables (see \nameref{dia:dataitemoptions} on page \pageref{dia:dataitemoptions}) \\
{\bfseries !} \HIDDEN  & don't write values of hidden variables (default) \\
\TRANSIENT & write values of transient variables (default, see \nameref{dia:dataitemmoreoption} on page \pageref{dia:dataitemmoreoption}) \\
{\bfseries !} \TRANSIENT & don't write values of transient variables \\
\end{tabularx}

\begin{boxedminipage}[t]{\linewidth}
\begin{alltt}
\STREAMER
  json_stream \{
    \JSON
  , \INDENT = 4
  , \HIDDEN
  , !\TRANSIENT
  \} (data_item_identifier);
\END \STREAMER;

\end{alltt}
\end{boxedminipage}

%%%%%%%%%%%%%%%%%%%%%%%%%%%%%%%%%%%%%%%%%%%%%%%%%%%%%%%%%%%%%%%%%%%%%%%%%%%%%
%%%                  Referencing Data Variables                           %%%
%%%%%%%%%%%%%%%%%%%%%%%%%%%%%%%%%%%%%%%%%%%%%%%%%%%%%%%%%%%%%%%%%%%%%%%%%%%%%
\newpage
\subsubsection{Referencing data variables (Data item)}
\label{sec:stvariables}
\label{sec:refvars}
Data items are declared in section \nameref{sec:datapool} on page \pageref{sec:datapool}. \\
A stream is used to read or write to or from data items, referencing them by their identifier. \\
\vspace{1cm}
%%\label{fig:sntx_st_dataitem}

\input{diagrams/st_advanced_data_reference}
\label{fig:st_advanced_data_reference}
\index{data item!streamer}

\input{diagrams/st_data_reference}
\label{fig:st_data_reference}

\index{  @Signs / Characters!. (dot)!streamer structure-item}
\index{VAR@\VAR!streamer}
\begin{tabularx}{\textwidth}{l|X}
data item         & description \\
\hline
{\bfseries ID\_DATAVARIABLE} & data item identifier previously defined in section \nameref{sec:dpitem} page \pageref{sec:dpitem}\\
{\bfseries ID\_DATASET}      & data set identifier previously defined in section \nameref{sec:dpset} page \pageref{sec:dpset}\\
\verb+identifier+ & data item identifier previously defined in section \nameref{sec:dpitem} page \pageref{sec:dpitem}\\
\verb+st field indizes+ & index (list) (see page \pageref{par:st_field_indizes}) \\
\VAR(string-item) & String dataitem contains the identifier of a data item.
                         By changing the contents of string-item at runtime, specified datapool item is referenced. \\
\verb+.+          & structure item separator \\
                  & Structures are explained in section \nameref{sec:dpstruct}
                  page \pageref{sec:dpstruct}\\
\end{tabularx}

\input{diagrams/st_field_indizes}
\label{par:st_field_indizes}

\begin{tabularx}{\textwidth}{l|X}
index             & description \\
\hline
{\bfseries INT\_CONSTANT} & a constant index \\
\verb+#+ (or empty) & the index of the array items (this is called pre-indexed) \\
\verb+#aZ9+       & \verb+#+ may be followed by any number or letter sign. \\
                  & This is used as an identifier when having more then one array-index. \\
                  & Must be previously declared in stream-declaration format-command (section \nameref{fig:st_format_command} page \pageref{fig:st_format_command}). \\
{\bfseries ID\_INDEX} & an index identifier (see section \nameref{sec:uiindex} on page \pageref{sec:uiindex} \\
\end{tabularx}
\vspace{0.5cm}

If a repetition group has no index as token
(is not pre-indexed), the column index of the matrix items
is assumed to be continuous and is beginning with 0.

Pre-indexed repetition groups ignore newline characters on input.
Their index must be the first element in the group.
\index{newline}

On input operations a string constant within the stream description is
considered as a search token.


\vspace{1cm}


%%%%%%%%%%%%%%%%%%%%%%%%%%%%%%%%%%%%%%%%%%%%%%%%%%%%%%%%%%%%%%%%%%%%%%%%%%%%%
%%%                                Examples                               %%%
%%%%%%%%%%%%%%%%%%%%%%%%%%%%%%%%%%%%%%%%%%%%%%%%%%%%%%%%%%%%%%%%%%%%%%%%%%%%%
\subsubsection{Examples}
\label{sec:stexamples}


\begin{boxedminipage}[t]{\linewidth}
\begin{verbatim}
STREAMER
  file_in(
    asm_type:-20,asm_descriptor:-35,doc_id:-12, \n,
    pz, rS, rR, lSigmaS*1e3, lSigmaR*1e3, lh*1e3, \n, \n,
    (psiMag[], iMag[], \n), \n, un, fn, \n,    klSamples
  );

  in(
    klSamples, un, fn, pz, rS, rR, lSigmaS, lSigmaR, lh, \n
  );

  out(
     ( speed[], i1[], i2[], torque[], cos[], \n )
  );

  report_out(
    \n, DATE, \n,
    \n, "Induction Motor Calculation Report", \n,
        "----------------------------------", \n,
    \n, "Machine Parameters:", \n,
    "N/(1/min): ", (speed[]), \n  );
END STREAMER;
\end{verbatim}
\end{boxedminipage}


Files structured like the following example can be read
with the above declared stream \verb+file_in+


\begin{boxedminipage}[t]{\linewidth}
\begin{verbatim}
1LA5 163-2CA        3000 1/min, 380V/50Hz              CHAX12345
 4 0.094 0.0847  1.05 0.78 28.4

       348.     18.17
       435.     22.71
       522.     27.35
      1305.    179.28
      1392.    215.73
      1479.    249.70
      1566.    292.42
      1653.    333.05
      1740.    372.93
      1827.    410.90

380 50
10
\end{verbatim}
\end{boxedminipage}


A repetition group within an input stream to Mathematica is realized as
a list. It is not necessary to define repetition groups within an output
stream from Mathematica, as for \INTENS{} every data item is stored
as a list.
%(see the example in appendix \nameref{sec:appendix})

%%%%%%%%%%%%%%%%%%%%%%%%%%%%%%%%%%%%%%%%%%%%%%%%%%%%%%%%%%%%%%%%%%%%%%%%%%%%%
%%%                             MATRIX                                    %%%
%%%%%%%%%%%%%%%%%%%%%%%%%%%%%%%%%%%%%%%%%%%%%%%%%%%%%%%%%%%%%%%%%%%%%%%%%%%%%
\newpage
\subsubsection{Matrix}
\label{sec:stmatrix}
\index{MATRIX@\MATRIX!example}
\index{MATLAB@\MATLAB!streamer matrix}

The \MATRIX{} token is used mainly with MATLAB
interfaces as shown in the following example:


\begin{boxedminipage}[t]{\linewidth}
\begin{verbatim}
  matlab_out
    ( Luser,ueuser
    , MATRIX Lin
    , Linmin ,Linmax
    , MATRIX ue[1,101]
    , U2effmax ,U2effnenn ,U2effmin
    );
\end{verbatim}
\end{boxedminipage}


The matrix is a multi-dimensional structure who's values are transferred in row-major order:


\begin{boxedminipage}[t]{\linewidth}
\begin{verbatim}
<dimension> <dim1> <dim2> <dim3> ...
<M001> <M002> <M003> ... <M010> <M011> <M012> ... <M020> <M021> ...
\end{verbatim}
\end{boxedminipage}


The first value indicate the number of dimensions followed by the number of items in each dimension.

Example of a 3x3 matrix:


\begin{boxedminipage}[t]{\linewidth}
\begin{verbatim}
2 3 3
0 1 2 1 0 1 2 2 0
\end{verbatim}
\end{boxedminipage}


gives the following matrix:\\[2ex]



\begin{tabular}{ccc}
0 & 1 & 2 \\
1 & 0 & 1 \\
2 & 2 & 0 \\
\end{tabular}

\input{diagrams/st_matrix_option}

\index{DELIMITER@\DELIMITER!matrix option (streamer)}
\begin{tabularx}{\textwidth}{l|X}
matrix option & description \\
\hline
\DELIMITER    & character used to separate strings. (Default: '\textbackslash0')
                    (can only be used if the item is a string). \\
\end{tabularx}


%%%%%%%%%%%%%%%%%%%%%%%%%%%%%%%%%%%%%%%%%%%%%%%%%%%%%%%%%%%%%%%%%%%%%%%%%%%%%
%%%                 How to send data through a stream                     %%%
%%%%%%%%%%%%%%%%%%%%%%%%%%%%%%%%%%%%%%%%%%%%%%%%%%%%%%%%%%%%%%%%%%%%%%%%%%%%%
\newpage
\subsubsection{How to send data through a stream}
\label{sec:sthowto}
\index{standard input!streamer}
\index{standard output!streamer}
\index{Streamer!how to}
First you need to define a process which generates data. This may be a unix-command like cat that
reads from standard input and writes to standard output channel
(see section \nameref{sec:opprocess} page \pageref{sec:opprocess}). \\
Each stream may be declared as input or as output by the processgroup definition
(section \nameref{sec:opprocessgroup} page \pageref{sec:opprocessgroup}).
\begin{itemize}
\item Input-stream \\
Read from datapool (data-item) write to standard output. \\
\item Processgroup \\
Pipe standard output to standard input of process. \\
\item Process \\
Execute process (eg. cat MyFile.dat). \\
\item Processgroup \\
Pipe standard output of process to standard input. \\
\item Output-stream \\
Read from standard input write to datapool (data-item). \\
\end{itemize}



\begin{boxedminipage}[t]{\linewidth}
\begin{verbatim}
DATAPOOL
  STRING
    data_item_output[15]
  ;
END DATAPOOL;

STREAMER
  streamer_id_out (#, data_item_output[#], EOLN) ;
END STREAMER;

OPERATOR
  PROCESS
    myfile_p : BATCH {
      "cat MyFile.dat"
    }
  ;
  PROCESSGROUP
    load_myfile { "Load" } (
      streamer_id_out = myfile_p (  );
    )
  ;
END OPERATOR;
\end{verbatim}
\end{boxedminipage}

\vspace{0.5cm}

The unix-command "cat MyFile.dat" reads the file MyFile.dat
and sends its contents to standard output.

Contents of "MyFile.dat":

\begin{boxedminipage}[t]{\linewidth}
\begin{alltt}
0 Each
1 word
2 of
3 this
4 file
5 will
6 be
7 stored
8 in
9 data_item_output
10 after
11 the
12 processgroup
13 is
14 executed
\end{alltt}
\end{boxedminipage}

\vspace{0.5cm}

Executing the process group load\_myfile by pressing the button
"Load" sets the values as follows:

\begin{boxedminipage}[t]{\linewidth}
\begin{verbatim}
data_item_output[0]  <-- "Each"
data_item_output[1]  <-- "word"
data_item_output[2]  <-- "of"
data_item_output[3]  <-- "this"
data_item_output[4]  <-- "file"
data_item_output[5]  <-- "will"
data_item_output[6]  <-- "be"
data_item_output[7]  <-- "stored"
data_item_output[8]  <-- "in"
data_item_output[9]  <-- "data_item_output"
data_item_output[10] <-- "after"
data_item_output[11] <-- "the"
data_item_output[12] <-- "processgroup"
data_item_output[13] <-- "is"
data_item_output[14] <-- "executed"
\end{verbatim}
\end{boxedminipage}

\vspace{0.5cm}

This is one possibility of displaying data\_items:

\begin{boxedminipage}[t]{\linewidth}
\begin{verbatim}
UI_MANAGER
  FIELDGROUP
    fieldgroup_identifier (
      "The first word in MyFile.dat is:" data_item_output[0],
      "The last word in MyFile.dat is:" data_item_output[14]
    )
  ;
  FORM
    form_identifier {MAIN} (
      (fieldgroup_identifier)
    )
  ;
END UI_MANAGER;
\end{verbatim}
\end{boxedminipage}
