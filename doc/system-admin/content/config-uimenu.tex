%%%%%%%%%%%%%%%%%%%%%%%%%%%%%%%%%%%%%%%%%%%%%%%%%%%%%%%%%%%%%%%%%%%%%%%%%%%%%
%%%                                 Menu                                  %%%
%%%%%%%%%%%%%%%%%%%%%%%%%%%%%%%%%%%%%%%%%%%%%%%%%%%%%%%%%%%%%%%%%%%%%%%%%%%%%
\subsubsection{Menu}
\label{sec:uimenu}
\INTENS{} automatically generates a number of menus to display forms,
to open, save and print files.
These menus can be rearranged to create a specific menu structure.
%%\index{menu!open} \index{menu!save} \index{menu!print}

\input{diagrams/ui_menu_list}

\index{MENU@\MENU!UI\_MANAGER!file-menu}
\index{MENU@\MENU!UI\_MANAGER!OPEN}
\index{MENU@\MENU!UI\_MANAGER!SAVE}
\index{MENU@\MENU!UI\_MANAGER!PRINT}
\index{MENU@\MENU!UI\_MANAGER!EXPLICIT}
\index{MENU@\MENU!UI\_MANAGER!PLOT2D}
\index{MENU@\MENU!UI\_MANAGER!NAVIGATOR}
\index{MENU@\MENU!UI\_MANAGER!LIST}
\index{MENU@\MENU!UI\_MANAGER!THERMO}
\index{MENU@\MENU!UI\_MANAGER!DATASTRUCTURE}
\index{MENU@\MENU!UI\_MANAGER}
\index{OPEN@\OPEN!UI\_MANAGER menu}
\index{SAVE@\SAVE!UI\_MANAGER menu}
\index{PRINT@\PRINT!ui\_manager menu}
\index{EXPLICIT@\EXPLICIT!ui\_manager menu}
\index{PLOT2D@\PLOTTWOD!ui\_manager menu}
\begin{tabularx}{\textwidth}{l|X}
MENU Syntax     & description \\
\hline
\OPEN           & References the Open submenu in the menu {\bfseries File}.\\
\SAVE           & References the Save submenu the menu {\bfseries File}.\\
\PRINT          & References Print submenu in the menu {\bfseries File}.\\
\EXPLICIT       & Only show these explicit entries (hide automatic entries). \\
ui form element identifier    & References the right click menu of a \NAVIGATOR, \LIST{} or \THERMO{}. \\
{\bfseries ID\_DATASTRUCTURE}  & References the right click menu of a {\bfseries DATA\-STRUCTURE}. \\
\PLOTTWOD         & References the right click menu of all \PLOTTWOD.  \\
\end{tabularx}

\input{diagrams/menu_menu_declaration}
\input{diagrams/menu_option_list}
\label{sec:uimenuid}

%%(not implemented in intens4) \index{MENU@\MENU!UI\_MANAGER!HELPTEXT}
\index{MENU@\MENU!UI\_MANAGER!FORM}
%%(not implemented in intens4) \index{HELPTEXT@\HELPTEXT!ui\_manager menu option}
\index{FORM@\FORM!ui\_manager menu option}

\begin{tabularx}{\textwidth}{l|X}
MENU options  & description \\
\hline
string        & Define a menu label (default is the menu id) \\
%%(not implemented in intens4) \HELPTEXT     & Displays defined helptext (string) when moving the mouse pointer over the menu label.\\
\FORM         & Places the menu in the form referenced by identifier.\\
\end{tabularx}

The user can display a form by activating the corresponding menu button.
\INTENS{} creates
the necessary buttons automatically in the form menu of the main window.
This menu structure can be specified with the following syntax. 

\input{diagrams/ui_form_menu_button_list}
\label{key:menu_form}
\input{diagrams/menu_button_label}
\input{diagrams/menu_submenu_declaration}

\index{MENU@\MENU!UI\_MANAGER!form-menu}
\index{MENU@\MENU!UI\_MANAGER!submenu}
\index{MENU@\MENU!UI\_MANAGER!FORM}
\index{MENU@\MENU!UI\_MANAGER!TOGGLE}
\index{MENU@\MENU!UI\_MANAGER!FUNC}
\index{MENU@\MENU!UI\_MANAGER!SEPARATOR}
\index{FORM@\FORM!ui\_manager forms menu}
\index{TOGGLE@\TOGGLE!ui\_manager forms menu}
\index{FUNC@\FUNC!ui\_manager forms menu}
\index{SEPARATOR@\SEPARATOR!ui\_manager forms  menu}
\begin{tabularx}{\textwidth}{l|X}
MENU Syntax       & description \\ 
\hline
\MENU             & creates a new submenu.\\
\FORM             & creates a menu button within the submenu containing the name of the form to be displayed with this button.\\
\TOGGLE           & The menu entry is created as a toggle, which opens and closes the form.\\
\FUNC             & calls a function defined in section \nameref{sec:functions} page \pageref{sec:functions}.\\
\SEPARATOR        & creates a separator line.\\
\end{tabularx}
\vspace{0.5cm}

Example:


\begin{boxedminipage}[t]{\linewidth}
\begin{alltt}
\MENU 
   "Results" (
      \FORM Form_1,
      \FORM Form_2,
      \MENU "Plots" (
         \FORM curve_form
      )
    );
\end{alltt}
\end{boxedminipage}

\vspace{1cm}

The file menus \OPEN, \SAVE{} and \PRINT{} can be rearranged as well.

\label{uifilemenu}
\input{diagrams/ui_file_menu_button_list}

\index{MENU@\MENU!UI\_MANAGER!file-menu}
\index{MENU@\MENU!UI\_MANAGER!submenu}
\index{MENU@\MENU!UI\_MANAGER!SEPARATOR}
\index{MENU@\MENU!UI\_MANAGER!STD\_WINDOW}
\index{MENU@\MENU!UI\_MANAGER!LOG\_WINDOW}
\index{MENU@\MENU!UI\_MANAGER!FUNC}
\index{SEPARATOR@\SEPARATOR!ui\_manager file  menu}
\index{STD\_WINDOW@\STDWINDOW!ui\_manager file menu}
\index{LOG\_WINDOW@\LOGWINDOW!ui\_manager file menu}
\index{FUNC@\FUNC!ui\_manager file menu}
\begin{tabularx}{\textwidth}{l|X}
MENU Syntax       & description \\
\hline
\MENU             & creates a new submenu.\\
\verb+identifier+ & creates a file menu button within the corresponding submenu. Must be a previously declared FILESTREAM or plotdiagram identifier.\\
\SEPARATOR        & creates a separator line.\\
\STDWINDOW        & references the standard window and sets the menu label.\\
\LOGWINDOW        & references the log window and sets the menu label.\\
\FUNC             & calls a function defined in section \nameref{sec:functions} page \pageref{sec:functions}.\\
\end{tabularx}

\label{uiprintmenu}
\input{diagrams/ui_print_menu_button_list}

\index{MENU@\MENU!UI\_MANAGER!PRINT}
\index{MENU@\MENU!UI\_MANAGER!submenu}
\index{MENU@\MENU!UI\_MANAGER!SEPARATOR}
\index{MENU@\MENU!UI\_MANAGER!STD\_WINDOW}
\index{MENU@\MENU!UI\_MANAGER!LOG\_WINDOW}
\index{SEPARATOR@\SEPARATOR!ui\_manager print  menu}
\index{STD\_WINDOW@\STDWINDOW!ui\_manager print menu}
\index{LOG\_WINDOW@\LOGWINDOW!ui\_manager print menu}
\begin{tabularx}{\textwidth}{l|X}
MENU Syntax              & description \\
\hline
\SEPARATOR               & creates a separator line.\\
\verb+menu button label+ & creates a print menu button for a previously
                           declared element.\\
\STDWINDOW               & references the standard window and sets the menu label.\\
\LOGWINDOW               & references the log window and sets the menu label.\\
\MENU                    & creates a new submenu.\\
\end{tabularx}
\vspace{0.5cm}

\STDWINDOW{} and \LOGWINDOW{} are only shown in the print menu if
their \MENU{} option is TRUE (see page \pageref{sec:uistdlogwindow}).


Example of a print menu:
\index{PRINT@\PRINT!ui\_manager menu!example}

\label{example:uimanagermenu}

\begin{boxedminipage}[t]{\linewidth}
\begin{alltt}
\MENU 
   \PRINT (
      \MENU Windows (
         \STDWINDOW = "Standard",
         \LOGWINDOW = "Log"
      ),
      \MENU Plots (
         Plot1 = "First Plot",
         Plot2 = "Second Plot"
      )
   );
\end{alltt}
\end{boxedminipage}


\input{diagrams/ui_form_element_menu_button_list}

\index{MENU@\MENU!UI\_MANAGER!FUNC}
\index{MENU@\MENU!UI\_MANAGER!SEPARATOR}
\index{FUNC@\FUNC!ui\_manager list menu}
\index{SEPARATOR@\SEPARATOR!ui\_manager list menu}
\begin{tabularx}{\textwidth}{l|X}
MENU Syntax     & description \\
\hline
\SEPARATOR   & creates a separator line.\\
\FUNC        & calls a function defined in section \nameref{sec:functions} page \pageref{sec:functions}.\\
\end{tabularx}


\label{uistructuremenu}
\input{diagrams/ui_structure_menu_option}
\input{diagrams/ui_structure_drop_menu_option}
\input{diagrams/ui_structure_menu_button_list}

\index{MENU@\MENU!UI\_MANAGER!FUNC}
\index{MENU@\MENU!UI\_MANAGER!SEPARATOR}
\index{FUNC@\FUNC!ui\_manager structure menu}
\index{SEPARATOR@\SEPARATOR!ui\_manager structure menu}
\index{DROP@\DROP!ui\_manager structure menu option}
\begin{tabularx}{\textwidth}{l|X}
MENU Syntax     & description \\
\hline
\INDEX       & you can define multiple menus for the same structure for the \DIAGRAM{} \NAVIGATOR{}
               (See section \nameref{sec:uinavigatorDiagram} page \pageref{sec:uinavigatorDiagram}) \newline
               Use \INDEX{} option to declare which menu you are defining. \newline
               defaults to 0. \\
\DROP        & create a DROP menu. The menu is opened when an item is dropped to this item. \\
\SEPARATOR   & creates a separator line.\\
\FUNC        & calls a function defined in section \nameref{sec:functions} page \pageref{sec:functions}.\\
\end{tabularx}

\input{diagrams/ui_plot2d_menu_entry_list}
\label{uiplot2dmenu}

All \PLOTTWOD{}s have the same right click menu. The default menu contains all possible
entries. If that is too much, the desired entries can be defined. And the order can be changed.
As this is valid for all \PLOTTWOD, it can only be done once. \\

These are the possible entries: \\
\begin{tabularx}{\textwidth}{l|X}
string       & {\bfseries menu entry} description \\
\hline
Redraw       & {\bfseries Redraw}
               Redraw the plot. \\
Zoom         & {\bfseries UI Mode}
               Select a UI mode. \newline
               See paragraph \nameref{par:uiplot2duimode}
               in section \nameref{sec:uiplot2d}
               on page \pageref{par:uiplot2duimode}. \\
Annotation   & {\bfseries Show X-Annotation-Labels ...}
               Show or hide the X-Annotation labels. (only in plots with X-Annotations). \\
Reset        & {\bfseries Zoom Out}
               Reset the zoom. \\
Print        & {\bfseries Print ...}
               Print the plot. \\
Config       & {\bfseries Configuration ...}
               Open the configuration dialog where you can choose what to plot where. \\
Cycles       & {\bfseries Select cycles ...}
               Open the dialog to select what cycles to plot.
               The menu entry is only active when more than one cycles are present. \\
Scale        & {\bfseries Scale ...}
               Open the scale dialog to scale the axes. \\
Logarithmic  & {\bfseries Logarithmic}
               Submenu to select what axes should be scaled logarithmic. \\
Style        & {\bfseries Style Y1, Style Y2}
               Two submenus to select the style for the graphs on the respective y-axis. \\
Copy         & {\bfseries Copy}
               Copy the plot to the clipboard. \\
Fullscreen   & {\bfseries Fullscreen}
               Open a new window with only this \PLOTTWOD. \\
Property     & {\bfseries Property}
               Show Plot2D Properties window. \\
\end{tabularx}
