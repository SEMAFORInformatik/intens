%%%%%%%%%%%%%%%%%%%%%%%%%%%%%%%%%%%%%%%%%%%%%%%%%%%%%%%%%%%%%%%%%%%%%%%%%%%%%
%%%                            TIMER                                     %%%
%%%%%%%%%%%%%%%%%%%%%%%%%%%%%%%%%%%%%%%%%%%%%%%%%%%%%%%%%%%%%%%%%%%%%%%%%%%%%
\subsubsection{Timer}
\label{sec:optimer}
Timers are used to start a function periodically. They are defined
in the OPERATOR. They are started or stopped using timer function statements
(see timer\_statement on page \pageref{dia:timerstatement}). \\[2ex]

See \nameref{sec:opexamples:messagequeue:publishsubscribe}
on page \pageref{sec:opexamples:messagequeue:publishsubscribe}. \\[2ex]


\input{diagrams/op_timer_declaration_list}
\index{TIMER@\TIMER}

\begin{tabularx}{\textwidth}{l|X}
timer            & description \\
\hline
\verb+identifier+ & Identifies the Timer. \\
\end{tabularx}

\input{diagrams/op_timer_option}
\index{FUNC@\FUNC!timer option}
\index{MAX\_PENDING\_FUNCTIONS@\MAXPENDINGFUNCTIONS!timer option}

\begin{tabularx}{\textwidth}{l|X}
timer options & description \\
\hline
\FUNC         & Defines the function that will be called repeatedly by the timer.
               (see also section \nameref{sec:functions} on page \pageref{sec:functions}) \\
\MAXPENDINGFUNCTIONS & Defines the maximal number of pending functions. This includes
                a possibly running function, the functions waiting to be executed (after
                the running function) and the \FUNC{} to be started by the timer. \newline
                When this number exceeds \MAXPENDINGFUNCTIONS, \FUNC{} is not started. \newline
                The timer continues and \FUNC{} may be started after another \PERIOD. \newline
                0 (default): no maximum, \FUNC{} is always started. \newline
                1: \FUNC{} is started when no function is running or waiting. \newline
                2: \FUNC{} is started when a function is running or not, but when no function is waiting. \newline
                n: \FUNC{} is started when at most n - 2 functions are waiting. \\
\end{tabularx}
