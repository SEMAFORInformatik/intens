\newpage
\subsubsection{Scale Factors}
%              -------------
\label{sec:scale}
%%%%%%%%%%%%%%%%%%%%%%%%%%%%%%%%%%%%%%%%%%%%%%%%%%%%%%%%%%%%%%%%%%%%%%%%%%%%%
%%%                               Scale                                  %%%
%%%%%%%%%%%%%%%%%%%%%%%%%%%%%%%%%%%%%%%%%%%%%%%%%%%%%%%%%%%%%%%%%%%%%%%%%%%%%
\index{Scale factors}
Scale factors are used in conjunction with datapool items to display their
value divided or multiplied by a factor.

The default value of a scale factor is NaN (not a number).
It is used when a datapool item is declared to be the scale factor, and that
item has no value.

You can change the default value of all scale factors to be 1.0 by starting
Intens with the commandline option \texttt{--}defaultScaleFactor1. Invalid scale factors then
behave as if they were not declared. \\

NOTE: starting with Rel 5.3.2 a unit manager can be used as an alternative
(see section \nameref{sec:unitmanager} page \pageref{sec:unitmanager}).

\input{diagrams/scale_factor}

\index{  @Signs / Characters!* (asterisk)!scale factor}
\index{  @Signs / Characters!/ (slash)!scale factor}
\index{  @Signs / Characters!- (hyphen)!scale factor}
\begin{tabularx}{\textwidth}{l|X}
scale                       & Description \\
\hline
\verb+real or int value+   & any numeric value or expression as shown in the example below. \\
\verb+temp data reference+ & references a numeric data item declared in the datapool
                             (section \nameref{sec:tempdatareference} page \pageref{sec:tempdatareference}). \\
\end{tabularx}

%%%%%%%%%%%%%%%%%%%%%%%%%%%%%%%%%%%%%%%%%%%%%%%%%%%%%%%%%%%%%%%%%%%%%%%%%%%%%
%%%                              Examples                                 %%%
%%%%%%%%%%%%%%%%%%%%%%%%%%%%%%%%%%%%%%%%%%%%%%%%%%%%%%%%%%%%%%%%%%%%%%%%%%%%%
%%\subsubsection{Examples}
Example:

\begin{boxedminipage}[t]{\linewidth}
\begin{intens}
DATAPOOL
  REAL data, factor;
END DATAPOOL;

UIMANAGER
  FIELDGROUP fg_identifier (
    "This fieldgroup shows following values:",
    " 20000: " data * 1e3,
    " 10: " data / 2,
    " 40: " data * factor,
    " 10: " data / factor  );
END UIMANAGER

FUNCTIONS
  FUNC INIT {
    data = 20; factor = 2;
  };
END FUNCTIONS;
\end{intens}
\end{boxedminipage}
