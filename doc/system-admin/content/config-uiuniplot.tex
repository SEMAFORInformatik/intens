%%%%%%%%%%%%%%%%%%%%%%%%%%%%%%%%%%%%%%%%%%%%%%%%%%%%%%%%%%%%%%%%%%%%%%%%%%%%%
%%%                                UNIPLOT                                %%%
%%%%%%%%%%%%%%%%%%%%%%%%%%%%%%%%%%%%%%%%%%%%%%%%%%%%%%%%%%%%%%%%%%%%%%%%%%%%%
\subsubsection{Uniplot}
\label{sec:uiuniplot}
\UNIPLOT{} is a ABB standard plot format with
\index{Plot!Uniplot}
a special set of plot vectors. It is mainly used within existing
Fortran programs. An example is given in figure \nameref{fig:uniplot}.

\input{diagrams/ui_uniplot_list}
\index{UNIPLOT@\UNIPLOT}
\index{PROCESSGROUP@\PROCESSGROUP!Uniplot}


\begin{boxedminipage}[t]{\linewidth}
\begin{intens}
DESCRIPTION "Example UNIPLOT";
  ... STUFF HERE ...
UI_MANAGER
  UNIPLOT
    mech_plot{ "Plot Mech" };
  FORM
    Form_Uniplot {"Plot Mech", HELPKEY "Mech_Plot", HIDECYCLE}
      ( ( mech_plot ) );
END UI_MANAGER;

OPERATOR
  PROCESS  plot_mech_proc : BATCH {"plotproc"};
  PROCESSGROUP
    mech_prog {"Mech"}(output_stream {DISPLAY=NONE},
                       mech_plot[mechuni] = plot_mech_proc( input_stream );
                     );
END OPERATOR;
  ... STUFF HERE ...
END.
\end{intens}
\end{boxedminipage}


%\newpage

\begin{figure}[h]
   \begin{center}
      \includegraphics[width=0.8\linewidth]{grab_uniplot}
   \end{center}
\caption{example of a UNIPLOT diagram}
  \label{fig:uniplot}
\end{figure}
