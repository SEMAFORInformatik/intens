\subsubsection{Resource}
%              ---------
\label{sec:resource}
%%%%%%%%%%%%%%%%%%%%%%%%%%%%%%%%%%%%%%%%%%%%%%%%%%%%%%%%%%%%%%%%%%%%%%%%%%%%%
%%%                            Resource                                   %%%
%%%%%%%%%%%%%%%%%%%%%%%%%%%%%%%%%%%%%%%%%%%%%%%%%%%%%%%%%%%%%%%%%%%%%%%%%%%%%
\index{RESOURCE@\RESOURCE}

\INTENS{} can use the value of an environment variable or of a resource property
set in a resource file (.ini, provided with \texttt{--}resfile, section [Resource]).

\INTENS{} reads the \RESOURCE{} at parse time. If neither environment variable nor
resource property is found, a parser error is thrown and the application does not start.

The type of a \RESOURCE{} is \INTEGER, \REAL{} or \STRING.
The type is based on the value of the resource:
\begin{itemize}
\item ``123'' is an \INTEGER
\item ``3.14'' is a \REAL{}
\item ``123abc'' is a \STRING
\end{itemize}

Make sure the type matches the context of the \RESOURCE: a parser error is thrown
when \RESOURCE{} is i.E. an \INTEGER{} in a context where a \STRING{} is needed.

Example:

\begin{boxedminipage}[t]{\linewidth}
\begin{intens}

MESSAGE_QUEUE mq_request{
  REQUEST
, HOST=RESOURCE("REPLY_HOST")
, PORT_REQUEST=RESOURCE("REPLY_PORT")
};

\end{intens}
\end{boxedminipage}

With a start script that includes the lines
\begin{verbatim}
export REPLY_HOST="localhost"
export REPLY_PORT=12345
\end{verbatim}
\INTENS{} replaces \verb+RESOURCE("REPLY_HOST")+ with the \STRING{} ``localhost'' and
\verb+RESOURCE("REPLY_PORT")+ with the \INTEGER{} 12345.
