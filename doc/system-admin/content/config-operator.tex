\newpage
\subsection{OPERATOR}
%           --------
\label{sec:operator}
%%%%%%%%%%%%%%%%%%%%%%%%%%%%%%%%%%%%%%%%%%%%%%%%%%%%%%%%%%%%%%%%%%%%%%%%%%%%%
%%%                             Description                               %%%
%%%%%%%%%%%%%%%%%%%%%%%%%%%%%%%%%%%%%%%%%%%%%%%%%%%%%%%%%%%%%%%%%%%%%%%%%%%%%
\subsubsection{Description}
\label{sec:opdescription}
The Operator communicates with the operating system and the
  external calculation programs using the pipe mechanism or the
  MathLink \index{Mathlink} protocol. Several (one to many) calculation programs
  are combined to a process group that can be started and stopped
  by the user activating a push button. Data can also be read from
  and saved to files.

\input{diagrams/operator_description}
\input{diagrams/op_element_declaration}
\index{OPERATOR@\OPERATOR}

%%%%%%%%%%%%%%%%%%%%%%%%%%%%%%%%%%%%%%%%%%%%%%%%%%%%%%%%%%%%%%%%%%%%%%%%%%%%%
%%%                               Process                                 %%%
%%%%%%%%%%%%%%%%%%%%%%%%%%%%%%%%%%%%%%%%%%%%%%%%%%%%%%%%%%%%%%%%%%%%%%%%%%%%%
%%\newpage
\subsubsection{Process}
\label{sec:opprocess}
Each external program has to be declared as \PROCESS{} to be included in one or more \PROCESSGROUP s.

\input{diagrams/op_process_declaration_list}
\index{PROCESS@\PROCESS}

%% \includegraphics[width=\linewidth]{xfig/sntx_op_process}
%% \vspace{1cm}

A calculation process must have a type and a unique name (identifier).
\vspace{0.5cm}

\index{FIFO!batch process}
\index{BATCH@\BATCH!operator process}
\index{DAEMON@\DAEMON!operator process}
\index{MATLAB@\MATLAB!batch process}
\begin{tabularx}{\textwidth}{l|X}
PROCESS types      & description \\
\hline
\verb+string+      & The external program and its arguments.\\
\verb+FIFO+        & identifier of named pipe. \\
\BATCH             & This is a executable batch program that reads from standard input
                     and writes to standard output. The string defines the
                     (preferably complete) path name.\\
\DAEMON            & Starts the executable batch program described in string. User must care about
                      ending/closing of daemon-process. No datatransfer through streams are possible. \\
\MATLAB            & This is a {\bfseries Matlab}-function that is included in a Matlab file. The
                     process name is identical to the function name.\\
\end{tabularx}

\vspace{0.5cm}
Example:


\begin{boxedminipage}[t]{\linewidth}
\begin{alltt}
  \PROCESS
     mech_proc : \BATCH \{
        "/tracomo/OSF1-V4.0-alpha/cmech"
       ," ", "1"
       ," ", mechres
       ," ", mechdyn
       ," ", "/home/tar/intens/tracomo/tp/who"
       \}
 ;
\end{alltt}
\end{boxedminipage}


%%%%%%%%%%%%%%%%%%%%%%%%%%%%%%%%%%%%%%%%%%%%%%%%%%%%%%%%%%%%%%%%%%%%%%%%%%%%%
%%%                            Process Group                              %%%
%%%%%%%%%%%%%%%%%%%%%%%%%%%%%%%%%%%%%%%%%%%%%%%%%%%%%%%%%%%%%%%%%%%%%%%%%%%%%
\newpage
\subsubsection{Process Group}
\label{sec:opprocessgroup}

\input{diagrams/op_processgroup_declaration_list}
\index{PROCESSGROUP@\PROCESSGROUP}

A calculation \PROCESS{} must be included in a \PROCESSGROUP{} which may contain any number of processes.
The input and output formats of each process is defined by previously declared streams
(or plot groups of type \UNIPLOT).
Unless explicitly specified, the IO channels used are standard input, standard output and standard error.
\index{UNIPLOT@\UNIPLOT!processgroup io format}
\index{standard input!processgroup}
\index{standard output!processgroup}
\index{standard error!processgroup}

\label{sntxdiagram:processgroupoptions}
\input{diagrams/op_processgroup_option_list}
\index{FORM@\FORM!processgroup option}
\index{NONE@\NONE!processgroup option FORM=NONE}
%%(not implemented in intens4) \index{HELPTEXT@\HELPTEXT!processgroup option}
\index{NO\_LOG@\NOLOG!processgroup option}
\index{SILENT@\SILENT!processgroup option}
\index{HIDDEN@\HIDDEN!processgroup option}
\index{UI\_UPDATE@\UIUPDATE!processgroup option}
\index{NO\_DEPENDENCIES@\NODEPENDENCIES!processgroup option}
\index{AUTOCLEAR\_DEPENDENCIES@\AUTOCLEARDEPENDENCIES!processgroup option}

\begin{tabularx}{\textwidth}{l|X}
process group options & description\\
\hline
\verb+string+         & defines the label of the push button, that starts the \PROCESSGROUP.
                        The default value is the name (\verb+identifier+) of the process group.\\
\FORM                 & creates the push button in the button bar of the form \verb+identifier+.
                        The push button is created in the main window if this option is missing.\\
\FORM = \NONE         & This \PROCESSGROUP{} will NOT have an associated button.\\
%%(not implemented in intens4) \HELPTEXT             & defines a text that is shown when the cursor is resting on the corresponding field.
%%(not implemented in intens4)                         (if menu 'options', menuitem 'helptext messages' is checked) \\
\UIUPDATE (n)         & Updates the user interface after n lines received while processing
                        a previously defined process. \\
\NOLOG                & Does not print log-messages (i.E. ABORT : PROCESSGROUP) to the \LOGWINDOW{}. \\
\SILENT               & The busy cursor is shown instead of the Process Dialog. \\
\HIDDEN               & \PROCESSGROUP{} is not added to the process-menu. \\
\NODEPENDENCIES       & No dependencies between input and output are added.
                        (see paragraph \nameref{par:stdependency} on page \pageref{par:stdependency}) \\
\AUTOCLEARDEPENDENCIES  & Dependencies are cleared without user confirmation.
                        (see paragraph \nameref{par:stdependency} on page \pageref{par:stdependency}) \\
\end{tabularx}

\input{diagrams/op_process_statement}

\begin{tabularx}{\textwidth}{l|X}
process statement & description\\
\hline
\verb+ID_PROCESS+ & Must be a previously declared {\bfseries process-identifier}
                                (section \nameref{sec:opprocess} page \pageref{sec:opprocess}).\\
\end{tabularx}

\input{diagrams/op_input_stream}
\input{diagrams/op_output_stream_list}
\input{diagrams/op_output_stream}
\input{diagrams/op_output_stream_option}

\label{opprocessgroupoutputstreamoption}
\index{DISPLAY@\DISPLAY!processgroup option}
\index{STD\_WINDOW@\STDWINDOW!processgroup option}
\index{NONE@\NONE!processgroup option DISPLAY=NONE}
\begin{tabularx}{\textwidth}{l|X}
output stream options  & description \\
\hline
\DISPLAY          & Output that is read by \INTENS{} from  the
                         calculation process can be directed to a user defined text window
                         (default is \STDWINDOW). \\
\DISPLAY=\NONE    & The output can be suppressed by setting \DISPLAY{} to \NONE.\\
\FORMAT           & TODO \newline
                    \ASCII{} TODO \newline
                    \BINARY{} TODO\\

\end{tabularx}

\input{diagrams/op_stream_parameter}
\index{  @Signs / Characters!\texttt{"!} (exclamation mark)!streamer validation check}
\index{  @Signs / Characters!\# (hash)!streamer index repesent.}
\index{  @Signs / Characters!: (colon)!streamer skip width}
\index{MATRIX@\MATRIX!streamer format command}
\index{Scale factors!streamer}
\index{DATASET\_TEXT@\DATASETTEXT!streamer format command}
\index{SET@\SET!dataset\_text streamer format command}
\index{STRING\_DATE@\STRINGDATE!streamer format command}
\index{STRING\_TIME@\STRINGTIME!streamer format command}
\index{STRING\_DATETIME@\STRINGDATETIME!streamer format command}
\index{STRING\_VALUE@\STRINGVALUE!streamer format command}
\index{EOLN@\EOLN!streamer format command}

Instead of declaring a \STREAM{} in the \STREAMER{} section and using it, a stream
can be defined here (without an identifier, it cannot be used elsewhere). The options
are described in section \nameref{fig:st_format_command} on page \pageref{fig:st_format_command}.
\vspace{1cm}

\index{FIFO!process stream}
The FIFO (named pipe) can be specified either as identifier or string:\\[2ex]
\begin{tabularx}{\textwidth}{l|X}
fifo name         & description \\
\hline
\verb+identifier+ & The fifo name is generated by appending the current process number
                    to the identifier. The fifo is created in the directory \verb+/tmp+
                    (and deleted after exiting \INTENS{}). Each \INTENS{} application has thus
                    unique fifos. The same identifier can be used as argument for the batch process.\\
\verb+string+     & The string defines the fifo name that \INTENS{} will create at startup
                    and delete after exiting.\\
\end{tabularx}
\vspace{0.5cm}

Example:

\begin{boxedminipage}[t]{\linewidth}
\begin{alltt}
  \PROCESSGROUP
     inmo_prog \{"Inmo"\} (
        jobinfo_stream \{\DISPLAY=\NONE\} =
           seq_proc( dummy_in_stream );

        plot_result_stream \{\DISPLAY=\NONE\} =
           mech_inmo_proc( mech_in_stream );

        dummy_out_stream \{\DISPLAY=Listing_Inmo\},
        res_inmo_stream[inmores] =
           inmo_proc( inmo_in_stream );
     )
 ;
\end{alltt}
\end{boxedminipage}


%%%%%%%%%%%%%%%%%%%%%%%%%%%%%%%%%%%%%%%%%%%%%%%%%%%%%%%%%%%%%%%%%%%%%%%%%%%%%
%%%                            Reportstream                               %%%
%%%%%%%%%%%%%%%%%%%%%%%%%%%%%%%%%%%%%%%%%%%%%%%%%%%%%%%%%%%%%%%%%%%%%%%%%%%%%
\subsubsection{Reportstreams}
%           ---------
\label{sec:reportstreams}
Report streams are used to send streams to the script ReportConv.py, which
transform the contents to the requested format (Postscript, PDF, etc.)
\label{sec:opreportstreams}
\input{diagrams/op_reportstream_declaration_list}
\index{REPORTSTREAM@\REPORTSTREAM}

\begin{tabularx}{\textwidth}{l|X}
reportstream      & description \\
\hline
\verb+identifier+ & Identifies the Reportstream. \\
\verb+stream id+  & References the stream defined in section STREAMER
                                      (\nameref{sec:streamer} page \pageref{sec:streamer}). \\
\end{tabularx}

\input{diagrams/op_reportstream_option}

\index{LATEX@\LATEX!reportstream option}
\index{FILTER@\FILTER!reportstream option}
\index{XML@\XML!reportstream option}
\index{STYLESHEET@\STYLESHEET!reportstream option}
\index{TEMPLATE@\TEMPLATE!reportstream option}
\index{DIRNAME@\DIRNAME!reportstream option}
\begin{tabularx}{\textwidth}{l|X}
reportstream options & description \\ \hline
\verb+label string+  & defines the label of the push button that opens the printer selection box\\
\LATEX               & The stream contains \LaTeX  commands.
                      (See \nameref{sec:oplatexreport} on page \pageref{sec:oplatexreport}.) \\
\FILTER              & The stream is piped to a unix-process. (Works similar to filestream-option
                              \PROCESS{}, see page \pageref{sec:opfilestreams}.) \\
\XML                 & TODO \\
\STYLESHEET          & TODO \\
\TEMPLATE            & TODO \\
\DIRNAME             & TODO \\
\end{tabularx}
\vspace{0.5cm}

Example:


\begin{boxedminipage}[t]{\linewidth}
\begin{alltt}
  \REPORTSTREAM
      print_motor_16 = motor_stream_16 \{
         "Motor V1.6"
        ,\LATEX
        \}
 ;
\end{alltt}
\end{boxedminipage}


%%%%%%%%%%%%%%%%%%%%%%%%%%%%%%%%%%%%%%%%%%%%%%%%%%%%%%%%%%%%%%%%%%%%%%%%%%%%%
%%%                            Latexreport                                %%%
%%%%%%%%%%%%%%%%%%%%%%%%%%%%%%%%%%%%%%%%%%%%%%%%%%%%%%%%%%%%%%%%%%%%%%%%%%%%%
\subsubsection{Latexreports}
Latexreports are used to create a pdf file using latex. Data as defined in a stream,
possibly processed by the program provided with the option \FILTER{}, is written to
a temporary .tex file. LaTeX then creates a pdf file from that .tex file.

\label{sec:oplatexreport}
\input{diagrams/op_latexreport_declaration_list}
\index{LATEXREPORT@\LATEXREPORT}

\begin{tabularx}{\textwidth}{l|X}
latexreport       & description \\
\hline
\verb+identifier+ & Identifies the Latexreport. \\
\verb+stream id+  & References the stream defined in section STREAMER
                                      (\nameref{sec:streamer} page \pageref{sec:streamer}). \\
\end{tabularx}

\input{diagrams/op_latexreport_option}

\index{FILTER@\FILTER!latexreport option}
\begin{tabularx}{\textwidth}{l|X}
latexreport options & description \\ \hline
\verb+string+      & Defines the label of the push button that opens the printer selection box. \\
\FILTER             & The stream data is piped to a batch process. (Works similar to filestream-option
                      \PROCESS{} on page \pageref{sec:opfilestreams}.) \\
\HIDDEN             & Don't add the \LATEXREPORT{} to the {\bfseries File Print} menu. \\
\end{tabularx}
\vspace{0.5cm}

Example:


\begin{boxedminipage}[t]{\linewidth}
\begin{alltt}
  \LATEXREPORT
      print_motor_16 = motor_stream_16 \{
         "Motor V1.6"
        \}
 ;
\end{alltt}
\end{boxedminipage}


%%%%%%%%%%%%%%%%%%%%%%%%%%%%%%%%%%%%%%%%%%%%%%%%%%%%%%%%%%%%%%%%%%%%%%%%%%%%%
%%%                             Filestream                                %%%
%%%%%%%%%%%%%%%%%%%%%%%%%%%%%%%%%%%%%%%%%%%%%%%%%%%%%%%%%%%%%%%%%%%%%%%%%%%%%
\subsubsection{Filestreams}
File streams are used for loading and saving streams to and from files. \\
\label{sec:opfilestreams}

\input{diagrams/op_filestream_declaration_list}
\index{FILESTREAM@\FILESTREAM!operator filestream}

\begin{tabularx}{\textwidth}{l|X}
filestream         & description \\
\hline
\verb+identifier+  & Identifies the filestream \\
\verb+stream id+   & References the stream defined in section STREAMER
                                (\nameref{sec:streamer} page \pageref{sec:streamer}) \\
\end{tabularx}

\input{diagrams/op_filestream_option}

\index{standard input!filestream}
\index{standard output!filestream}
\index{FILTER@\FILTER!filestream option}
\index{PROCESS@\PROCESS!filestream option}
\index{READONLY@\READONLY!filestream option}
\index{WRITEONLY@\WRITEONLY!filestream option}
\index{RESET@\RESET!filestream option}
\index{NO\_LOG@\NOLOG!filestream option}
\begin{tabularx}{\textwidth}{l|X}
filestream options & Bedeutung \\
\hline
\verb+string+     & defines the label of the push button that opens the file selection dialog\\
\FILTER            & defines the file filter in the file selection dialog (default is \verb+*+)\\
\PROCESS           & defines the executable batch program that is called for a reading or writing
                     operation. The program must read from standard input
                     and write to standard output. This program can be used for format conversions.\\
\READONLY          & The stream is used for reading files only.
                     It is not added to the {\bfseries File Save} menu. \\
\WRITEONLY         & The stream is used for writing files only.
                     It is not added to the {\bfseries File Open} menu. \\
\RESET             & A read operation resets the modify status of the datapool items and
                                          deletes the associated results.\\
\NOLOG             & Does not print log-messages to the \LOGWINDOW. \\
\EXTENSION         & Defines extension text, which will be automaticly added to the filename. \\
\HIDDEN            & Don't add the \FILESTREAM{} to the {\bfseries File Open} and {\bfseries File Save} menus. \\
\DIRNAME           & Defines the default directory for this filestream. \\
\end{tabularx}
\vspace{0.5cm}

Example:


\begin{boxedminipage}[t]{\linewidth}
\begin{alltt}
  \FILESTREAM
      get_motor_16 = open_motor_stream_16 \{
         "Motor V1.6"
        ,\PROCESS="/home/tar/intens/tracomo/tp/inmo/pmotor"
        ,\READONLY, \RESET, \FILTER="*.inm"
        \}
 ;
\end{alltt}
\end{boxedminipage}


%%%%%%%%%%%%%%%%%%%%%%%%%%%%%%%%%%%%%%%%%%%%%%%%%%%%%%%%%%%%%%%%%%%%%%%%%%%%%
%%%                                Packages                               %%%
%%%%%%%%%%%%%%%%%%%%%%%%%%%%%%%%%%%%%%%%%%%%%%%%%%%%%%%%%%%%%%%%%%%%%%%%%%%%%
%%\newpage

%%%%%%%%%%%%%%%%%%%%%%%%%%%%%%%%%%%%%%%%%%%%%%%%%%%%%%%%%%%%%%%%%%%%%%%%%%%%%
%%%                                Tasks                                  %%%
%%%%%%%%%%%%%%%%%%%%%%%%%%%%%%%%%%%%%%%%%%%%%%%%%%%%%%%%%%%%%%%%%%%%%%%%%%%%%
\newpage
\subsubsection{Tasks}
\label{sec:optasks}
Tasks provide a process control feature:
the execution of user defined process groups can be programmed using
a C-like syntax
(see section \nameref{sec:functions} on page \pageref{sec:functions}).

\input{diagrams/job_task_list}
\index{TASK@\TASK}

\begin{tabularx}{\textwidth}{l|X}
task              & description\\
\hline
\verb+identifier+ & Name of the task\\
\verb+statement+  & Task statements have the same syntax as function statements.
                        (see section \nameref{sec:functions} on page \pageref{sec:functions})\\
\end{tabularx}

\input{diagrams/job_task_options}

\index{FORM@\FORM!task option}
\index{NONE@\NONE!task option FORM=NONE}
%%(not implemented in intens4) \index{HELPTEXT@\HELPTEXT!task option}
\index{HIDDEN@\HIDDEN!task option}
\index{NO\_LOG@\NOLOG!task option}
\index{SILENT@\SILENT!task option}
\begin{tabularx}{\textwidth}{l|X}
task options & description\\
\hline
\verb+string+         & defines the label of the push button, that starts the TASK.
                        The default value is the name (\verb+identifier+) of the task.\\
\FORM                 & creates the push button in the button bar of the form \verb+identifier+.
                        The push button is created in the main window if this option is missing.\\
\FORM=\NONE           & If the identifier is \NONE{} no button will be created at all.\\
\HIDDEN               & Does not create a menu-entry in process-menu. \\
\NOLOG                & Does not print log-messages (i.E. BEGIN : TASK, END : TASK, ABORT : TASK)
                        to the  \LOGWINDOW{}
                        (except \PRINT{}, \SETERROR{} and \ABORT{} messages). \\
\SILENT               & The busy cursor is shown instead of the Process Dialog. \\
\end{tabularx}
\vspace{0.5cm}

Example:


\begin{boxedminipage}[t]{\linewidth}
\begin{alltt}
\OPERATOR
 \TASK
    task_identifier \{"button label"\}(
      \RUN processgroup_id_1;
      \RUN processgroup_id_2;
   )
  ;
END \OPERATOR ;
\end{alltt}
\end{boxedminipage}


%%%%%%%%%%%%%%%%%%%%%%%%%%%%%%%%%%%%%%%%%%%%%%%%%%%%%%%%%%%%%%%%%%%%%%%%%%%%%
%%%                                Menu                                   %%%
%%%%%%%%%%%%%%%%%%%%%%%%%%%%%%%%%%%%%%%%%%%%%%%%%%%%%%%%%%%%%%%%%%%%%%%%%%%%%
\newpage
\subsubsection{Menu}
\label{sec:opmenu}
\index{MENU@\MENU!OPERATOR}
\input{diagrams/op_menu_list}
\input{diagrams/op_menu_button_list}

The menu buttons for the declared process groups and tasks
can be explicitly created and appended in different menus using the
\MENU{} declaration.
\vspace{0.5cm}

\index{MENU@\MENU!OPERATOR!PROCESSGROUP}
\index{MENU@\MENU!OPERATOR!TASK}
\index{PROCESS@\PROCESS!operator menu}
\index{SEPARATOR@\SEPARATOR!operator menu}
\begin{tabularx}{\textwidth}{l|X}
MENU Syntax     & description \\
\hline
\MENU           & creates a submenu.\\
\PROCESS        & creates a menu button within that submenu containing the
                  name of the process group or task to be started with this button.\\
\SEPARATOR      & creates a separator line\\
\end{tabularx}
\vspace{0.5cm}

Example:


\begin{boxedminipage}[t]{\linewidth}
\begin{alltt}
\OPERATOR
 \MENU
    Process \{\FORM = Form_1, "Calculations" \}(
      \PROCESS Process_1,
      \PROCESS Process_2,
      \SEPARATOR,
      \MENU "more Processes" (
         \PROCESS Process_3
      )
   )
  ;
END \OPERATOR ;
\end{alltt}
\end{boxedminipage}
