\newpage
\subsubsection{Identifier}
%              ----------
\label{sec:identifier}
%%%%%%%%%%%%%%%%%%%%%%%%%%%%%%%%%%%%%%%%%%%%%%%%%%%%%%%%%%%%%%%%%%%%%%%%%%%%%
%%%                             Identifier                                %%%
%%%%%%%%%%%%%%%%%%%%%%%%%%%%%%%%%%%%%%%%%%%%%%%%%%%%%%%%%%%%%%%%%%%%%%%%%%%%%
\index{identifier}
Identifiers are used in almost every section of this manual.
Their first character must be a letter: \\

\input{diagrams/IDENTIFIER}

\index{  @Signs / Characters!\# (hash)!data item identifier}
\index{  @Signs / Characters!\_ (underline)!data item identifier}
\begin{tabularx}{\textwidth}{l|X}
Identifier & Description \\
\hline
letter    & alphabetic characters  (a..z, A..Z) \\
digit     & digits (0..9) \\
\#        & \# character (not a wildcard!) may be used as abbreviation of number \\
\_        & underline character \\
\end{tabularx}

\input{diagrams/identifier}
\input{diagrams/data_identifier}
\input{diagrams/job_function_pointer}

%%%%%%%%%%%%%%%%%%%%%%%%%%%%%%%%%%%%%%%%%%%%%%%%%%%%%%%%%%%%%%%%%%%%%%%%%%%%%
%%%                              Examples                                 %%%
%%%%%%%%%%%%%%%%%%%%%%%%%%%%%%%%%%%%%%%%%%%%%%%%%%%%%%%%%%%%%%%%%%%%%%%%%%%%%
%%\subsubsection{Examples}
\label{sec:stringexamples}
\vspace{1cm}

Example:


\begin{boxedminipage}[t]{\linewidth}
\begin{verbatim}
data_item_identifier
speed
fieldgroup_1
main_form
Superusers
StreamTitle
Phone_#
\end{verbatim}
\end{boxedminipage}
