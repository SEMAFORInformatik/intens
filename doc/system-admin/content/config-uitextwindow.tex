%%%%%%%%%%%%%%%%%%%%%%%%%%%%%%%%%%%%%%%%%%%%%%%%%%%%%%%%%%%%%%%%%%%%%%%%%%%%%
%%%                             Text Window                               %%%
%%%%%%%%%%%%%%%%%%%%%%%%%%%%%%%%%%%%%%%%%%%%%%%%%%%%%%%%%%%%%%%%%%%%%%%%%%%%%
\subsubsection{Text-Window}
\label{sec:uitextwindow}
All output that is created by external programs is displayed
in scrolled text windows. Unless otherwise specified the 'standard output'
is printed in the text window called \STDWINDOW{} while the 'standard error'
\index{standard output!ui\_manager text window}
\index{standard error!ui\_manager text window}
\index{STD\_WINDOW@\STDWINDOW!ui\_manager text window}
\index{LOG\_WINDOW@\LOGWINDOW!ui\_manager text window}
is linked to the \LOGWINDOW. These two windows are usually placed in the
main window. They can easily be moved to any other form, as described in: \\
example folder on page \pageref{example:uimanagerfolder},
section form on page \pageref{dia:uiformcontainerelement}. \\
\vspace{0.5cm}

Additional text windows can be created by using the \TEXTWINDOW{} declaration. \\
To print out data in such a window,
you have to refer to the text window identifier in the processgroup-option
\DISPLAY{} explained in section OPERATOR on page \pageref{opprocessgroupoutputstreamoption}.\\

\input{diagrams/ui_text_window_list}
\input{diagrams/ui_text_window_option_list}
\index{TEXT\_WINDOW@\TEXTWINDOW}

\index{WRAP@\WRAP!text window option}
\index{SIZE@\SIZE!text window option}
\index{FORTRAN@\FORTRAN!text window option}
\index{FILTER@\FILTER!text window option}
\index{  @Signs / Characters!* (asterisk)!textwindow size}
\begin{tabularx}{\textwidth}{l|X}
option        & description \\
\hline
identifier    & Window identifier (appears in the save-menu also) \\
\WRAP         & wrap the words, if the line is too long\\
\SIZE         & defines the displayed \verb+height+ and \verb+width+ of the text window as number
                of characters. The text window itself will always be sized according
                to the other fieldgroups within the same form.\\
\FORTRAN      & The text-window expects data in Fortran printfile-format.\\
\FILTER       & The option declares an external program (string), which preprocesses the text
                before sending it to the printer (usually an ASCII-to-PostScript converter,
                default is a2ps) \\
\end{tabularx}
\vspace{1cm}

Example:


\begin{boxedminipage}[t]{\linewidth}
\begin{alltt}
  \TEXTWINDOW Listing_Tcmo \{
    \SIZE=36*148
   ,\FORTRAN
   ,\FILTER="/usr/local/bin/a2ps -nP -ns -nL -nH -1 -l -8 -F6.8"
  \};
\end{alltt}
\end{boxedminipage}
\vspace{1cm}

\paragraph{\STDWINDOW{} and \LOGWINDOW{} options}
\label{sec:uistdlogwindow}
~\\[0.5cm]

\input{diagrams/ui_std_window_options}

\begin{tabularx}{\textwidth}{l|X}
option        & description \\
\hline
\MENU         & Show the window in the print menu. (Default is \FALSE). \\
              & If the window is shown, the menu label can be renamed as shown in : \\
              & example of ui\_manager menu on page \pageref{example:uimanagermenu} \\
\end{tabularx}
\vspace{1cm}

\begin{boxedminipage}[t]{\linewidth}
\begin{alltt}
  \STDWINDOW \{ \MENU=\TRUE \};
\end{alltt}
\end{boxedminipage}
