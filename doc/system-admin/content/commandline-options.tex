\newpage
\section{Commandline options}
\label{sec:cloptions}
%        -------------------
%%%%%%%%%%%%%%%%%%%%%%%%%%%%%%%%%%%%%%%%%%%%%%%%%%%%%%%%%%%%%%%%%%%%%%%%%%%%%
%%%                            Commandline options                        %%%
%%%%%%%%%%%%%%%%%%%%%%%%%%%%%%%%%%%%%%%%%%%%%%%%%%%%%%%%%%%%%%%%%%%%%%%%%%%%%
An INTENS application is started calling intens followed by the name of the
description file: \\
\verb+intens description.des+ \\
In addition, INTENS knows the following options.
The list of INTENS commandline options can be printed using the command: \\
\verb+intens --help+

\begin{description}
\item[\texttt{--}geometry <argument>] defines the geometry of the main window (on linux only). \\
  <argument> is <width>x<height>[+<xOffset>+<yOffset>]
%
%\item[\texttt{--}name <argument>]
%  TODO
%
%\item[\texttt{--}display <argument>]
%  TODO
%
%\item[\texttt{--}mathnode <argument>]
%  TODO
%
%\item[\texttt{--}mathport <argument>]
%  TODO
%
%\item[\texttt{--}matlabnode <argument>]
%  TODO
%
%\item[\texttt{--}fontname <argument>]
%  TODO
%
%\item[\texttt{--}background <argument>]
%  TODO
%
%\item[\texttt{--}xfontlist <argument>]
%  TODO
%
\item[\texttt{--}logconfig <argument>] is obsolete.
  It used to be the configuration filename for log4cxx.
%
\item[\texttt{--}localeDomain <argument>]
  INTENS uses gettext for translation. There are two locale domains:
  \begin{description}
  \item[INTENS] itself is translated using the locale domain name ``intens''
  and the locale directory ``<INTENS\_HOME>/share/locale''.
  The mo file for the german translation is
  ``<INTENS\_HOME>/share/locale/de/LC\_MESSAGES/intens.mo''.
  \item[The Application] is translated using a second locale domain.
  The default locale domain name for the application is the name
  of the description file, without the extension. You can, however,
  provide the locale domain name using this commandline option.
  This is needed i.E. when you want multiple applications to use the
  same translation files. \\
  The locale directory is ``<APPHOME>/share/locale''.
  The mo file for the german translation is
  ``<APPHOME>/share/locale/de/LC\_MESSAGES/<localeDomainName>.mo''.
  \end{description}
%
\item[\texttt{--}init <argument>] defines the filename of a datapool init file.
  That file is read at startup to initialize the datapool. \\
  The file has one value per line:
  \begin{description}
  \item[Scalar] speed 0.0
  \item[List] \# speed 1 1.0
  \item[Matrix] \# \# speed 1 2 2.0
  \item[Struct] points[3].x 3.0
  \end{description}
%
\item[\texttt{--}notitle] hides the titlebar at startup.
  The titlebar shows
  \begin{itemize}
  \item a title (apptitle.text from the resource file)
  \item a subtitle (appsubtitle.text from the resource file)
  \item a left bitmap
  \item a right bitmap
  \end{itemize}
  (see Minimal Example on page \pageref{fig:minimalExample}).
  It is shown at startup unless this option is given.
  It can be shown and hidden later using the menu Options > Titlebar.
%
\item[\texttt{--}shortMainTitle]
  The operating systems title bar
  (Maintitle in Minimal Example on page \pageref{fig:minimalExample})
  of the main window shows
  \begin{itemize}
  \item the application title (string after DESCRIPTION)
  \item the path of INTENS
  \item the hostname
  \item the INTENS version
  \end{itemize}
  With this option, however, it only shows the application title.
%
\item[\texttt{--}toolbar]
  obsolete
%
\item[\texttt{--}undo] enables undo/redo in INTENS. \\
  The Edit menu is added with two entries:
  \begin{description}
  \item[Undo Ctrl-Z] Use this menu or the keyboard shortcut Ctrl-Z to undo a changed value.
  \item[Redo Ctrl-Y] Use this menu or the keyboard shortcut Ctrl-Y to undo an undo.
  \end{description}
  You can undo at most five changes.
%
%\item[\texttt{--}detailGrid]
%  TODO
%
\item[\texttt{--}helpmsg] enables help messages, shown in the Statusbar.
  The help message mode can be changed later using the menu Options > Help messages. \\
  INTENS can show a help message of the content under the mouse-pointer. The help message is defined
  using \HELPTEXT.
  INTENS knows three types of help messages:
  \begin{description}
  \item[Statusbar] shows the help message in the Statusbar
  (see Minimal Example on page \pageref{fig:minimalExample}).
  \item[Tooltip] shows the help message as a tooltip.
  \item[Disabled] does not show help messages.
  \end{description}
%
\item[\texttt{--}resfile <argument>]
  INTENS reads resources (Colors, fonts, plot line styles, etc.) from the file
  provided with this option.
%
\item[\texttt{--}createRes <argument>] asks INTENS to write a default resource file.
  <argument> is the filename. \\
  No description file is needed.
%
%\item[\texttt{--}listFonts]
%  TODO
%
\item[\texttt{--}qtGuiStyle <argument>] sets the Qt Style. \\
  Use \texttt{--}listQtGuiStyles to get a list of available styles.
%
\item[\texttt{--}listQtGuiStyles] prints the available Qt Styles. \\
  No description file is needed.
%
\item[\texttt{--}maxoptions <argument>] sets the maximal number of options shown in a \COMBOBOX.
  The default value is 25.
%
\item[\texttt{--}maxlines <argument>] sets the maximal number of lines of the \LOGWINDOW{} and the \STDWINDOW{}.
  The default value is 500.
  When the content exeed this maximum, the oldest 20\% lines are deleted.
%
\item[\texttt{--}toolTipDuration <argument>] sets the maximal number of seconds a help message is shown.
  The default value is 10.
%
\item[\texttt{--}leftIcon <argument>] defines the name of the left bitmap in the titlebar.
  The default value is ``semafor'' and shows the Semafor logo.
  See Minimal Example on page \pageref{fig:minimalExample}.
%
\item[\texttt{--}rightIcon <argument>] defines the name of the right bitmap in the titlebar.
  The default value is ``'' and shows no right bitmap in the titlebar.
  See Minimal Example on page \pageref{fig:minimalExample}.
%
\item[\texttt{--}startupImage <argument>] defines the name of the splash screen bitmap.
  When this option is given and the bitmap is found, it is shown as a splash screen during application startup.
%
\item[\texttt{--}rolefile <argument>]
  TODO
%
\item[\texttt{--}includePath <argument>] defines the path where INCLUDE files are searched for.
  Multiple paths are separated by ':'. \\
  An \INTENS{} description file can be split into multiple files. These files are included using \\
  INCLUDE file.inc (see \nameref{sec:opexamples:messagequeue:publishsubscribe}
  on page \pageref{sec:opexamples:messagequeue:publishsubscribe}). \\
  The INCLUDE statement can include the path, a subpath or no path at all. \\
  INCLUDE files are (also) searched in the current directory:
  \begin{itemize}
  \item without this option
  \item when <argument> is an empty string
  \item when it is included in <argument> (as '.' or '': foo:, :foo, foo:., .:foo, foo::bar, foo:.:bar)
  \end{itemize}

  Examples (- -includePath include:/tmp/include): \\
  \begin{description}
  \item[INCLUDE common.inc] (filename only):
  \begin{itemize}
  \item ./include/common.inc
  \item /tmp/include/common.inc
  \end{itemize}
  \item[INCLUDE etc/common.inc] (filename with subpath):
  \begin{itemize}
  \item ./include/etc/common.inc
  \item /tmp/include/etc/common.inc
  \end{itemize}
  \end{description}
%
\item[\texttt{--}disableSVGSupport]
  TODO
%
\item[\texttt{--}pspreviewer <argument>]
  TODO
%
\item[\texttt{--}printerConfig <argument>]
  TODO
%
%\item[\texttt{--}create]
%  TODO
%
%\item[\texttt{--}xml <argument>]
%  TODO
%
%\item[\texttt{--}jsb <argument>]
%  TODO
%
%\item[\texttt{--}cpp]
%  TODO
%
%\item[\texttt{--}xmlPath <argument>]
%  TODO
%
\item[\texttt{--}apprunPath <argument>]
  TODO
%
\item[\texttt{--}dbdriver <argument>]
  TODO
%
\item[\texttt{--}dbautologon]
  TODO
%
\item[\texttt{--}corba <argument>]
  TODO
%
\item[\texttt{--}debug <argument>]
  TODO
%
\item[\texttt{--}persistfile <filename>]
  INTENS parses the description file, writes information about persistent components
  into <filename> (using XML) and exits.
%
\item[\texttt{--}persistfileREST <filename>]
  Same as \texttt{--}persistfile, but \LABEL{} and \HELPTEXT{} are not written
  to <filename>. This allows the usage of html tags in \LABEL{} and \HELPTEXT{}.
%
\item[\texttt{--}log4cplusPropertiesFile <argument>] sets the filename of the log4cplus properties file.
  The default is APPHOME/config/log4cplus.properties.
%
\item[\texttt{--}startToken <argument>]
  TODO
%
\item[\texttt{--}reflistfile <argument>]
  TODO
%
\item[\texttt{--}helpdir <argument>]
  TODO
%
\item[\texttt{--}version] prints the INTENS version and exits. \\
  No description file is needed.
%
\item[\texttt{--}help] prints the list of INTENS commandline options and exists. \\
  No description file is needed.
%
\item[\texttt{--}whichGui] prints the GUI type (QT) and exists. \\
  No description file is needed.
%
\item[\texttt{--}withWheelEvent] enables the mouse wheel event for:
  \begin{description}
  \item[Number] increase/decrease a number using the mouse wheel
  (unless \texttt{--}withoutArrowKeys is given) \\
  The mouse wheel event can be enabled for single numeric data items using the datapool option \WHEELEVENT
  (see \nameref{dia:dataitemmoreoption} on page \pageref{dia:dataitemmoreoption}).
  \item[\COMBOBOX] select different entries using the mouse wheel
  \item[\INDEX] increase/decrease a GUI \INDEX{} using the mouse wheel
  \end{description}
%
\item[\texttt{--}withInputStructFunc] enables the call of a struct (or parent) function
  when an input variable is changed that does not have a function. \\
  See example \nameref{fuexample3} on page \pageref{fuexample3}.
%
\item[\texttt{--}withoutArrowKeys] disables the ``arrow keys''. \\
  When enabled (default), you can use the up/down keys to increment/decrement
  the digit of a number before the cursor.
%
\item[\texttt{--}defaultScaleFactor1] sets the default scale factor to 1.0.
  See section \nameref{sec:scale} on page \pageref{sec:scale}.
%
\item[\texttt{--}withoutEditableComboBox] disables the completer of \COMBOBOX{}es. \\
  The completer makes it possible to type some characters into the \COMBOBOX{} and the options
  are filtered to the values that start with the typed characters.
%
\item[\texttt{--}withoutTextPopupMenu] disables the right click menu of multiline text input fields.
%
\item[\texttt{--}ORBIIOPVersion <argument>]
  TODO
%
\item[\texttt{--}ORBInitRef <argument>]
  TODO
%
\item[\texttt{--}test <argument>] enables TestMode:
  \begin{itemize}
  \item The given function is called after the INIT function.
        End the function with the \EXIT{} statement to close the application automatically.
  \item The QUIT function is not run when the application is closed.
  \item \MESSAGEBOX{}es are not shown. Their content is printed to stderr.
  \end{itemize}
%
\item[\texttt{--}replyPort <argument>]
  TODO
%
\item[\texttt{--}sendMessageQueueWithMetadata]
  TODO
%
\item[\texttt{--}defaultMessageQueueDependencies <argument>]
  A \MESSAGEQUEUE{} \REQUEST{} may define dependencies between input (\REQUEST) and output (\RESPONSE) \STREAM{}s
  (see paragraph \nameref{par:stdependency} on page \pageref{par:stdependency}). \\
  Each \STREAM{} can have an explicit option \DEPENDENCIES{} or \NODEPENDENCIES{}
  (see \nameref{dia:jobmessagequeueoption} on page \pageref{dia:jobmessagequeueoption} or
  \nameref{dia:jobpluginoption} on page \pageref{dia:jobpluginoption}). \\
  Without such an option, the default \MESSAGEQUEUE{} dependencies are used. \\
  This commandline options defines that default:
  \begin{itemize}
  \item 1 or ``true'' (=default): The \STREAM{} is used with dependencies
  \item 0 or ``false'': The \STREAM{} is not used with dependencies
  \end{itemize}
\end{description}
